\documentclass[a4paper, 12pt]{article}

\usepackage{cmap}
\usepackage[T2A]{fontenc}
\usepackage[english, russian]{babel}
\usepackage[utf8]{inputenc}
\usepackage[left=2cm,right=1.5cm,top=2cm,bottom=2cm]{geometry}
% \usepackage{mathtext}
\usepackage{amsmath}
\usepackage{amssymb}
\usepackage{etoolbox}
\usepackage{amsthm}
\usepackage{booktabs}
\usepackage{hyperref}
\usepackage{xcolor}
% \usepackage{nicematrix}
\usepackage{graphicx}
% \usepackage{tikz}
% \usepackage{parskip}

% Цвета для гиперссылок
\definecolor{linkcolor}{HTML}{225ae2} % цвет ссылок
\definecolor{urlcolor}{HTML}{225ae2} % цвет гиперссылок
\hypersetup{
    pdfstartview=FitH, 
    linkcolor=linkcolor,
    urlcolor=urlcolor,
    colorlinks=true
}

\addto\captionsenglish{% Replace "english" with the language you use
  \renewcommand{\contentsname}%
    {Содержание}%
}

\parindent = 1cm
\DeclareMathOperator{\End}{\mathrm{End}_{\mathbb{K}}}

\newtheorem*{definition}{Определение}
\newtheorem*{theorem}{\hspace*{\parindent}Теорема}
\theoremstyle{definition}
\newtheorem*{prof}{\hspace*{\parindent}Доказательство}
\newtheorem*{consequense}{Следствие}
\newtheorem*{lemma}{Лемма}
\newtheorem*{subtheorem}{Утверждение}
\newtheorem*{remark}{Замечание}


\usepackage{titlesec}
\titleformat{\section}{\LARGE \bfseries}{\thesection}{1em}{}
\titleformat{\subsection}{\Large\bfseries}{\thesubsection}{1em}{}
\titleformat{\subsubsection}{\large\bfseries}{\thesubsubsection}{1em}{}


\begin{document}

\fontsize{14pt}{20pt}\selectfont

\begin{titlepage}
\newpage

\begin{center}
Московский государственный университет\\
имени М.В.Ломоносова
\end{center}

\begin{center}
Механико-математический факультет \\
Кафедра высшей алгебры \\ 
\end{center}

\vspace{15em}

\begin{center}
\textsc{\textbf{Квантовые симметрии \linebreak в алгебре тройных чисел}}
\end{center}

\vspace{15em}



%\vbox{%
%  \hfill%
%    \vbox{%
%        \hbox{Выполнил студент:}%
%        \hbox{Зазовский Л.С.}%
%        \hbox{211 группа}%
%        \hbox{Научный руководитель:}%
%        \hbox{проф. Гордиенко А.С.}%
%    }%
%}

\begin{flushright}
\begin{minipage}{.30\textwidth}
Выполнил студент:\\
\vbox{%
    \hfill%
    \vbox{%
        \hbox{211 группы}%
        \hbox{Зазовский Л.С.      }%
    }%
}
Научный руководитель:\\
\vbox{%
    \hfill%
    \vbox{%
        \hbox{проф. Гордиенко А.С.}%
    }%
}
\end{minipage}
\end{flushright}



\vspace{\fill}

\begin{center}
Москва \\2025
\end{center}

\end{titlepage}

    \newpage
    \tableofcontents
    \fontsize{14pt}{20pt}\selectfont
    \newpage

\section{Конечно двойственная алгебра Хопфа}

\subsection{Структура конечного двойственного пространства}

Пусть у нас задано векторное пространство $V$ над полем $\mathbb{K}$. 

Известно, что если $V$ конечномерное, тогда $V \cong V^*$ и существует канонический изоморфизм между $V$ и $V^{**}$. Оказывается, что похожий результат верен в бесконечномерных пространствах. 

\begin{definition} 
Конечное двойственное пространство $V^\circ$"--- это подпространство $V^*$ такое, что для любой линейной функции $\alpha \in V^\circ$ её ядро имеет конечную коразмерность.
\end{definition}

В случае конечномерных пространств $V^\circ = V^*$.

\begin{definition}
    Пусть в $V$ выбран базис $(e_i)_{i \in \Lambda}$. Тогда 
    двойственная система $(\varepsilon_i)_{i \in \Lambda}$ в $V^\circ$ "--- это система функций  таких, что для всех $i, j \in \Lambda$ верно соотношение $\varepsilon_i(e_j) = \delta_{ij}$, где $\delta_{ij}$ "--- символ Кронекера.
\end{definition}

В случае конечномерных пространств двойственная система окажется в точности системой координатных функций. 

\begin{theorem}
    Двойственная система является базисом в $V^\circ$
\end{theorem}

\begin{prof}
    
    Сначала докажем, что двойственная система полна в $V^\circ$. Рассмотрим некоторую линейную функцию $f$ из $V^\circ$.
    Тогда существует векторное подпространство $W$ такое, что $V$, раскладывается в прямую сумму ${V = W \oplus \mathrm{Ker} (f)}$.

    Рассмотрим проектор $\pi$ на подпространство $W$ вдоль $\mathrm{Ker}(f)$. Нетрудно, заметить что $\mathrm{Ker}(\pi) = \mathrm{Ker}(f)$ и $\mathrm{Im}(\pi) = W$. Значит $W \cong V/\mathrm{Ker}(f)$.
    Отсюда следует, что $W$ имеет конечную размерность, обозначим её как $m$. Выберем базис $a_1, \dots, a_m$ в $W$.
    Для любого вектора $v$ верно разложение $v = \lambda_1a_1+ \dots +\lambda_ma_m + u$, где $u$ некоторый вектор из $\mathrm{Ker}(f)$. Тогда $f(v) =  f(\lambda_1a_1+ \dots +\lambda_ma_m + u) = \lambda_1f(a_1)+ \dots +\lambda_mf(a_m)$. 
    Мы получили, что значение $f$ на произвольном векторе является линейной комбинацией значений на $(a_i)_{i = 1}^m$

    Так как $(e_i)_{i \in \Lambda}$ "--- базис, значит все $(a_j)_{j = 1}^m$ выражаются с помощью конечной линейной комбинации $(e_i)_{i \in \Lambda}$.
    Отсюда понятно, что значение $f$ на произвольном векторе является конечной линейной комбинацией значений на конечном подмножестве базисных векторов $(e_i)_{i \in \Lambda}$. Обозначим это подмножество $e_{i_1}, \dots, e_{i_n}$.
    Тогда можно выразить $f = f(e_{i_1})\varepsilon_{i_1} + \dots + f(e_{i_n})\varepsilon_{i_n}$.
    А это значит, что двойственная система полна.

    Теперь докажем линейную независимость двойственной системы. Рассмотрим линейную комбинацию некоторой конечной подсистемы $\lambda_{i_1}\varepsilon_{i_1} + \dots + \lambda_{i_n}\varepsilon_{i_n} = 0$. Подставим $e_{i_k}$, получим $\lambda_{i_k}\varepsilon_{i_k}(e_{i_k}) = 0$, а значит $\lambda_{i_k} = 0$. 
    Взяв все возможные $k$ от $1$ до $n$ получим, что все коэффициенты равны нулю, что означает линейную независимость этой подсистемы. Следовательно все конечные подсистемы линейно независимы, а значит и двойственная система линейно независима. \qed

\end{prof}

\begin{remark}
    В силу доказанной выше теоремы будем называть двойственную систему "--- конечно двойственным базисом.
\end{remark}
Для полного соответствия аналогичным результатам в случае конечномерных пространств остатётся доказать две нижеследующие теоремы.

\begin{theorem}
    Пусть задано векторное пространство $V$ над полем $\mathbb{K}$. Тогда $V \cong V^\circ$
\end{theorem}

\begin{prof}
    Выберем в $V$ базис $(e_i)_{i \in \Lambda}$ и соответсвующий ему конечно двойственный базис $(\varepsilon_i)_{i \in \Lambda}$
    Пусть $\varphi: V \to V^\circ$ "--- линейное отображение заданное на базисе $\varphi(e_i) := \varepsilon_i$. Очевидно, что $\varphi$ сюръективно.

    Докажем инъективность. Пусть $\varphi(a) = \varphi(b)$ для некоторых $a, b \in V$. Тогда $\varphi(a-b) = 0$. Разложим $a-b$ по базису $V$: $a-b = \lambda_{i_1}e_{i_1} + \dots + \lambda_{i_n}e_{i_n}$.
    Тогда $\varphi(\lambda_{i_1}e_{i_1} + \dots + \lambda_{i_n}e_{i_n}) = \lambda_{i_1}\varepsilon_{i_1} + \dots + \lambda_{i_n}\varepsilon_{i_n} = 0$. В силу линейной независимости $(\varepsilon_{i_k})_{k=1}^n$, получаем $\lambda_{i_1} =\dots=\lambda_{i_n} = 0$. Значит $a-b =0$, следовательно $\varphi$ инъективно. Тогда $\varphi$ изоморфизм. \qed
\end{prof}

\begin{theorem}
    Существует канонический изоморфизм $\varphi: V \to V^{\circ\circ}$.
\end{theorem}

\begin{prof}
    Определим $\varphi: V \to V^{\circ\circ}$, как $\varphi(u)(\alpha):= \alpha(u)$, для любых $u \in V$ и $\alpha \in V^\circ$. 
    Докажем корректность определения, а именно тот факт, что $\varphi(u) \in V^{\circ\circ}$.
    Ядро $\varphi(u)$ это множество линейных функций $\alpha$ таких, что $\alpha(u) = 0$.
    Выберем базис $(e_i)_{i \in \Lambda}$ в $V$ и соответствующий ему конечно двойственный базис $(\varepsilon_i)_{i \in \Lambda}$ в $V^\circ$. Пусть $u = \lambda_{i_1}e_{i_1}+ \dots +\lambda_{i_m}e_{i_m}$, тогда $\varepsilon_j(u) = \lambda_j$, если для некоторого $k$ верно $j=i_k$, и 
    $\varepsilon_j(u)= 0$ иначе. Отсюда получаем что $V^\circ/\mathrm{Ker}(\varphi(u)) \cong \langle\varepsilon_{i_1}, \dots , \varepsilon_{i_m} \rangle$, откуда следует, что $\mathrm{Ker}(\varphi(u))$ имеет конечную коразмерность.

    Теперь докажем, что $\varphi$ "--- изоморфизм.
    Пусть $\varphi(a)(\alpha) = \varphi(b)(\alpha)$ для любой функции $\alpha \in V^\circ$. 
    Тогда для любого $i \in \Lambda$ верно $\varphi(a)(\varepsilon_i)= \varphi(b)(\varepsilon_i)$, иначе говоря $\varepsilon_i(a) = \varepsilon_i(b)$ для любого  $i \in \Lambda$. Это значит что $a$ и $b$ имеют одинаковые разложения по базису $(e_i)_{i \in \Lambda}$. Получаем, что $a = b$, следовательно $\varphi$ инъективен.

    Пусть $f \in V^{\circ\circ}$. Выберем в $V^{\circ\circ}$ базис $(\epsilon_i)_{i \in \Lambda}$ конечно двойственный к $(\varepsilon_i)_{i \in \Lambda}$. 
    Разложим функцию $f$ по $(\epsilon_i)_{i \in \Lambda}$: $f = \lambda_{i_1}\epsilon_{i_1} + \dots + \lambda_{i_k}\epsilon_{i_k}$. 
    Рассмотрим вектор $u \in V$, такой что $u = \lambda_{i_1}e_{i_1} + \dots + \lambda_{i_k}e_{i_k}$ и докажем, что $\varphi(u) = f$.
    
    $f(\varepsilon_j) = \lambda_j$, если для некоторого $k$ верно, что $j = i_k$, иначе $f(\varepsilon_j) = 0$. $\varphi(u)(\varepsilon_j) = \varepsilon_j(u) = \lambda_j$, если для некоторого $k$ верно, что $j = i_k$, иначе $\varphi(u)(\varepsilon_j) = \varepsilon_j(u) = 0$.
    Значит $f$ и $\varphi(u)$ совпадают на базисе, следовательно $f = \varphi(u)$. Тем самым мы доказали, что $\varphi$ сюръективен. \qed
    
\end{prof}

Для удобства рассуждений введём ещё одно понятие.

\begin{definition}

Пусть задано векторное пространство $V$ над полем $\mathbb{K}$ и система $K \subseteq V^*$.
Единицей линейной функции $\alpha \in K$ по системе $K$~--- будем называть вектор $v\in V$, такой что $\alpha(v) = 1$ и для любой линейной функции $\beta \in K$ отличной от $\alpha$ верно $\beta(v) = 0$
\end{definition}
Будем обозначать единицу линейной функции $\alpha$ как $\alpha^{(1)}$.
Из рассуждений выше следует, что для любой функции из конечно двойственного базиса существует единица линейной функции.

\newpage

\subsection{Структура конечно двойственной алгебры Хопфа}

Пусть задана $A$ "--- алгебра с единицей над полем $\mathbb{K}$. Рассмотрим подмножество линейных функций $A^\circ \subseteq A^*$, такое что в ядре любой функции $\alpha \in A^\circ$ содержится идеал конечной коразмерности. Тогда $\mu^*(A^\circ) \subseteq A^\circ \otimes A^\circ$, а значит можно определить конечную двойственную коалгебру.

\begin{definition}
    Конечная двойственная коалгебра $A^\circ$ "--- это коалгебра $(A^\circ, \mu^\circ, u^\circ)$, где $\mu^\circ$ и $u^\circ$ это ограничения $\mu^*$ и $u^*$ на $A^\circ$ соответственно.
\end{definition}

Как множество $A^\circ$ является подпространством в конечно двойственном пространстве к векторному пространству $A$. 
Выберем в $A^\circ$ базис $(\varepsilon_i)_{i \in \Lambda}$ и дополним его до базиса всего конечно двойственного пространства.
Рассмотрим конечно двойственный базис в $A$, в нём существует подсистема $(e_i)_{i \in \Lambda}$ из единиц линейных функций составляющих базис $A^\circ$. Она обладает следующим свойством $\varepsilon_i(e_j) = \delta_{ij}$, где $\delta_{ij}$ "--- символ Кронекера.

Пусть задана $H$ "--- алгебра Хопфа над полем $\mathbb{K}$. Рассмотрим $H^\circ$ "--- конечную двойственную коалгебру. Как множество $H^\circ$ является подалгеброй в $H^*$, поэтому $H^\circ$ является алгеброй Хопфа.

\begin{definition}
    Пусть задана $H$ "--- алгебра Хопфа над полем $\mathbb{K}$. $H^\circ$ называют конечной двойственной алгеброй Хопфа.
\end{definition}

Здесь и далее мы пользуемся обозначениями Свидлера и для удобства опускаем знак суммы: $\bigtriangleup a = a_{(1)} \otimes a_{(2)}$. 
Выпишем некоторые полезные соотношения, следующие из двойственности. Для любых $\alpha, \beta \in H^\circ$ и $h, g \in H$ верно:
\begin{eqnarray}
    \varepsilon\alpha = \alpha\varepsilon = \alpha \label{str1}\\
    (\alpha\beta)(a) = \alpha(a_{(1)})\beta(a_{(2)}) \label{str2}\\
    u^\circ(\alpha) = \alpha(1_H) \label{str3}\\
    \bigtriangleup\alpha(a \otimes b) = \alpha(ab) \label{str4}
\end{eqnarray}

\newpage

\section{Характеризация \texorpdfstring{$H$}{H}-модульной структуры}

Пусть задана $A$ "--- конечномерная $H$-модульная алгебра с 1 над полем $\mathbb{K}$, где $H$ "--- алгебра Хопфа.

Отображение $\psi: H \otimes A \to A$, такое что $\psi(h \otimes a) := ha$ для всех $h \in H$ и $a \in a$, называется $H$-модульной структурой. Определим гомоморфизм алгебр $\zeta: H \to \End(A)$ равенством $\zeta(h)(a) := \psi (h \otimes a)$.

Выбрав базис в $A$ такой, что его первым элементом будет единица алгебры $A$, отождествим $\End(A)$ с $M_n(\mathbb{K})$.
Тогда существуют такие $(\alpha_{ij})_{ij} \in H^\circ$, что \[\zeta(h) = 
\begin{pmatrix}
    \varepsilon(h) && \alpha_{12}(h) && \dots  && \alpha_{1n}(h)\\
    0              && \alpha_{22}(h) && \dots  && \alpha_{2n}(h)\\
    \vdots         && \vdots         && \ddots && \vdots          \\
    0              && \alpha_{n2}(h) && \dots  && \alpha_{nn}(h)\\
\end{pmatrix}
\]

Для удобства положим $\alpha_{11} = \varepsilon$ и $\alpha_{21} = \dots = \alpha_{n1}  = 0$ и обозначим за $\mathrm{K}$ базис $\langle\varepsilon, (\alpha_{ij})_{ij}\rangle$.

Будем называть $\zeta(h)$ "--- матрицей модульной структуры $\psi$ . При так заданном отображении $\zeta$, верно $\zeta(H) = \mathrm{cosupp} \ \psi$. Кроме того, рассмотрев ранг матрицы модульной структуры, получаем $\mathrm{dim} \ \zeta(H) = \mathrm{dim}\  \langle K \rangle$. 

Нетрудно заметить, что $\mathrm{Ker}(\zeta)$ совпадает с аннулятором $\langle \mathrm{K} \rangle$.
Из того, что $\mathrm{Ker}(\zeta)$ "--- идеал конечной коразмерности, следует что в ядре любой функции из $\langle\mathrm{K}\rangle$ содержится идеал конечной коразмерности, а значит $\langle\mathrm{K}\rangle \subseteq H^\circ$. 

Так как $\zeta$ гомоморфизм алгебр, для любых $a, b \in H$ верно $\zeta(ab)=\zeta(a)\zeta(b)$. 
Значит для всех $1 \leq i, j \leq n$  верно $$
    \alpha_{ij}(ab) = \sum\limits_{k=1}^{n}\alpha_{ik}(a)\alpha_{kj}(b) \eqno(*)
$$. 

Дополним $K$ до базиса $(\varepsilon_i)_{i \in \Lambda}$ в $H^\circ$, тогда существует система $(e_i)_{i \in \Lambda}$ в $H$, такая что $\varepsilon_i(e_j) = \delta_{ij}$, где $\delta_{ij}$ "--- символ Кронекера. 
Разложим $\bigtriangleup\alpha_{ij}$ по базису $(\varepsilon_k \otimes \varepsilon_l)_{k, l \in \Lambda}$. Рассматривая $\bigtriangleup\alpha_{ij}$ от всех возможных $e_k \otimes e_l$ и используя (\ref{str4}) и $(*)$, получим 
$$\bigtriangleup\alpha_{ij} = \sum\limits_{k=1}^{n}\alpha_{ik} \otimes \alpha_{kj}$$


\newpage

\section{Классификация модульных структур}

\begin{theorem}
	Пусть $\psi:H \otimes A\ \to \ A $~-- структура $H$-модульной алгебры с $1$ на $A = \mathbb{K}[x]/(x^3)$, где $H$~--- некоторая алгебра Хопфа, $\mathrm{char} \ \mathbb{K} \neq 3$ и в поле существует примитивный корень степени 3. Выберем базис в $A: \bar 1, \bar x, \bar x^2$ и отождествим $\End(A)$ с $M_3(\mathbb{K})$. Предположим, что коноситель действия это подалгебра верхнетреугольных матриц. Тогда $\psi$ эквивалентно одной из следующих модульных структур над $A$:
    
    \begin{enumerate}
        \item действие поля $\mathbb{K}$ на алгебре $A$ умножением на скаляры; \label{scalar}
        
        \item действие групповой алгебры $\mathbb{K}\langle c\rangle_2$, заданное равенством 
        ${c\bar{x} = -\bar{x}}$; \label{eps=b^2}
        
        \item действие групповой алгебры $\mathbb{K}\langle c \rangle_3$, заданное равенством $c\bar{x} = \xi \bar{x}$,
        
        где $\xi$ "--- примитивный корень из единицы степени $3$;\label{diagonal} 
        
        \item $H_9(\xi)$-действие, заданное равенствами $c\bar{x}=\xi\bar{x},\ v\bar{x}= \bar{1}$, 
        
        где $\xi$ "--- примитивный корень из единицы степени $3$. \label{all}
        
    \end{enumerate}
\end{theorem}


\begin{prof}
    Определим отображение $\zeta:H \to \End(A)$ из равенства $\zeta(h)(a) := \psi(h \otimes a)$.
    Выберем в $A$ базис $\bar 1, \bar x, \bar x^2$ и отождествим $\End(A)$ с $M_3(\mathbb{K})$. Тогда $\zeta(h)$ будет матрицей модульной структуры. Из условий теоремы следует, что матрица модульной структуры верхнетреугольная.
    Для удобства изложения переобозначим её нижеследующим образом: \[
    \zeta(h) = 
    \begin{pmatrix}
        \varepsilon(h) & \alpha(h) & \tau(h)\\
         0 & \beta(h) & \theta(h) \\
         0 & 0 & \varphi(h)
    \end{pmatrix}    
    \]
    Из того, что $A$ "--- $H$-модульная алгебра с 1, следует 
    $$h(\bar{x}^3)=(h_{(1)}\bar{x})(h_{(2)}\bar{x}^2) = 0 $$ 
    $$h(\bar{x}^2)=(h_{(1)}\bar{x})(h_{(2)}\bar{x})$$
    Используя определение $\zeta(h)$, получаем следующее
    $$h\bar x= \alpha(h)\bar1 + \beta(h) \bar x$$
    $$h\bar x^2 = \tau(h)\bar1 + \theta(h) \bar x + \varphi(h) \bar x^2$$
    Тогда, используя $\bar x^3 = 0$, получим следующие соотношения
    \[
    (h_{(1)}\bar{x})(h_{(2)}\bar{x}) = \left( \alpha(h_{(1)}) \bar{1} + \beta(h_{(1)}) \right)
    \left( \alpha(h_{(2)}) \bar{1} + \beta(h_{(2)}) \right)=
    \]
    \[
     = \alpha^2(h)\bar{1} + (\alpha\beta + \beta\alpha)(h) \bar{x} + \beta^2(h) \bar{x}^2
    \]
    $$h\bar x^2 = (h_{(1)}\bar{x})(h_{(2)}\bar{x}) = \tau(h)\bar1 + \theta(h) \bar x + \varphi(h) \bar x^2 = \alpha^2(h)\bar{1} + (\alpha\beta + \beta\alpha)(h) \bar{x} + \beta^2(h) \bar{x}^2$$
    А значит можно выразить  
    \begin{align} \label{repl}
        \tau &= \alpha^2 \notag \\
        \theta &= \alpha\beta+\beta\alpha \\
        \varphi &=\beta^2 \notag
    \end{align}

    \noindent Кроме того
    \begin{multline*}
        (h_{(1)}\bar{x})(h_{(2)}\bar{x}^2) = 
        \left( \alpha(h_{(1)}) \bar{1} + \beta(h_{(1)}) \bar{x} \right)
        \left( \tau(h_{(2)}) \bar{1} + \theta(h_{(2)}) \bar{x} + \varphi(h_{(2)}) \bar{x}^2 \right)=
        \\
        = (\alpha\tau)(h) \bar{1}
        +(\alpha\theta + \beta\tau)(h)\bar{x}+(\alpha\varphi + \beta\theta)(h)\bar{x}^2
        =0
    \end{multline*}
    Отсюда получаем
    \begin{align*}
        &\alpha \tau = 0\\
        &\alpha \theta + \beta \tau = 0\\
        &\alpha \varphi + \beta \theta = 0
    \end{align*}
    Воспользовавшись (\ref{repl}) получим
    \begin{align}
        &\alpha^3 = 0 \label{1}\\
        &\alpha^2\beta+\alpha\beta\alpha + \beta\alpha^2 = 0 \label{2}\\
        &\alpha\beta^2+\beta\alpha\beta+\beta^2\alpha = 0 \label{3}
    \end{align}
    
    Из соотношений (\ref{repl}) выразим $\tau,\ \theta,\ \varphi$ и перепишем матрицу модульной структуры, используя эти выражения:
    \[\zeta(h) = 
    \begin{pmatrix}
        \varepsilon(h) & \alpha(h) & \alpha^2(h)\\
         0 & \beta(h) & (\alpha\beta + \beta\alpha)(h) \\
         0 & 0 & \beta^2(h)
    \end{pmatrix}\]

    \noindentИз доказанного в предыдущем разделе \[\mathrm{dim} \ \zeta(H) = \mathrm{dim} \langle\varepsilon,\ \alpha,\ \beta,\ \alpha^2,\ \alpha\beta+\beta\alpha,\ \beta^2 \rangle \leq 6\]

    \noindentЕсли $\mathrm{dim}\ \zeta(H) = 6$, тогда $\zeta(H)$~-- подалгебра всех верхнетреугольных матриц. Докажем, что в таком случае модульная структура $\psi$ эквивалентна модульной структуре \ref{all}.

    Пусть $\psi_1:H_9(\xi) \otimes A \to A$ "--- модульная структура \ref{all}.
    Определим гоморфизм алгебр $\zeta_1:H_9(\xi) \to M_3(\mathbb{K})$ с помощью соотношения ${\zeta_1(h)(a)=\psi_1(h \otimes a)}$.
    Тогда мы получим \[
    \zeta_1(c) = 
    \begin{pmatrix}
        1 & 0 & 0\\
        0 & \xi & 0\\
        0 & 0 & \xi^2\\
    \end{pmatrix},\ 
    \zeta_1(v) = 
    \begin{pmatrix}
        0 & 1 & 0\\
        0 & 0 & 1+\xi\\
        0 & 0 & 0\\
    \end{pmatrix}
    \]

    Выпишем образ базиса $(c^kv^l)_{0 \leq k,l \leq 2}$ в $\zeta(H)$:
    \begin{eqnarray*}
    &\zeta(1) = \begin{pmatrix}
    1 & 0 & 0\\
    0 & 1 & 0\\
    0 & 0 & 1\\
    \end{pmatrix},\
    \zeta(c) = \begin{pmatrix}
    1 & 0 & 0\\
    0 & \xi & 0\\
    0 & 0 & \xi^2\\
    \end{pmatrix},\
    \zeta(c^2) = \begin{pmatrix}
    1 & 0 & 0\\
    0 & \xi^2 & 0\\
    0 & 0 & \xi\\
    \end{pmatrix}\\
    &\zeta(v) = \begin{pmatrix}
    0 & 1 & 0\\
    0 & 0 & 1+\xi\\
    0 & 0 & 0\\
    \end{pmatrix},\
    \zeta(cv) = \begin{pmatrix}
    0 & 1 & 0\\
    0 & 0 & \xi+\xi^2\\
    0 & 0 & 0\\
    \end{pmatrix} 
    \zeta(c^2v) = \begin{pmatrix}
    0 & 1 & 0\\
    0 & 0 & 1+\xi^2\\
    0 & 0 & 0\\
    \end{pmatrix}\\
    &\zeta(v^2) = \begin{pmatrix}
    0 & 0 & 1+\xi\\
    0 & 0 & 0\\
    0 & 0 & 0\\
    \end{pmatrix},\
    \zeta(cv^2) = \begin{pmatrix}
    0 & 0 & 1+\xi\\
    0 & 0 & 0\\
    0 & 0 & 0\\
    \end{pmatrix} 
    \zeta(c^2v^2) = \begin{pmatrix}
    0 & 0 & 1+\xi\\
    0 & 0 & 0\\
    0 & 0 & 0\\
    \end{pmatrix}\\    
    \end{eqnarray*}

    Рассмотрим эти матрицы как вектора в базисе из матричных единиц и запишем их координаты построчно в матрицу: \[ M=
    \begin{pmatrix}
        1 & 0 & 0 & 0 & 1 & 0 & 0 & 0 & 1\\
        1 & 0 & 0 & 0 & \xi & 0 & 0 & 0 & \xi^2\\
        1 & 0 & 0 & 0 & \xi^2 & 0 & 0 & 0 & \xi\\
        0 & 1 & 0 & 0 & 0 & 1+\xi & 0 & 0 & 0\\
        0 & 1 & 0 & 0 & 0 & \xi+\xi^2 & 0 & 0 & 0\\
        0 & 1 & 0 & 0 & 0 & 1+\xi^2 & 0 & 0 & 0\\
        0 & 0 & 1+\xi & 0 & 0 & 0 & 0 & 0 & 0\\
        0 & 0 & 1+\xi & 0 & 0 & 0 & 0 & 0 & 0\\
        0 & 0 & 1+\xi & 0 & 0 & 0 & 0 & 0 & 0\\
    \end{pmatrix}
    \]
    Нетрудно, проверить, что $\mathrm{rank}\ M = 6$, а значит $\dim \mathrm{cosupp} \ \psi_1 = 6$, откуда $\mathrm{cosupp}\ \psi_1$ совпадает c алгеброй верхнетреугольных матриц.
    
    \noindentПусть теперь $\mathrm{dim}\ \zeta(H) \leq 5$\par

    В предыдущем разделе были доказаны разложения для результатов коумножения на элементах матрицы модульной структуры. Выпишем их для наших функций:
    \begin{eqnarray*}
        \bigtriangleup\varepsilon &=& \varepsilon \otimes \varepsilon\\
        \bigtriangleup\alpha &=& \varepsilon \otimes \alpha + \alpha \otimes \beta\\
        \bigtriangleup\beta &=& \beta \otimes \beta\\
        \bigtriangleup\alpha^2 &=& \varepsilon \otimes \alpha^2 + \alpha \otimes (\alpha\beta+\beta\alpha) + \alpha^2 \otimes \beta^2\\
        \bigtriangleup(\alpha\beta+\beta\alpha) &=& \beta \otimes (\alpha\beta+\beta\alpha) + (\alpha\beta+\beta\alpha)\otimes \beta^2\\
        \bigtriangleup\beta^2 &=& \beta^2 \otimes \beta^2\\
    \end{eqnarray*}
    Значит функции $\varepsilon,\ \beta, \beta^2$ являются группоподобными.
    
    Если $\varepsilon$ и $\beta$ линейно зависимы, тогда из того, что $\varepsilon \neq 0$, следует, что $\beta = \lambda\varepsilon$, при этом $1 = \beta(1_H)=\lambda\varepsilon(1_H)=\lambda$, значит $\beta = \varepsilon$.
    Отсюда $\beta^2 = \varepsilon^2 = \varepsilon$. Подставим это в (\ref{3}).
    \[
        \alpha\varepsilon + \varepsilon\alpha \varepsilon + \varepsilon \alpha = 3\alpha = 0
    \]
    Из этого тождества и того, что $\mathrm{char} \ \mathbb{K} \neq 3$, следует что $\alpha = 0$. Тогда наша матриц модульной структуры выглядит следующим образом:\[
    \zeta(h) = \begin{pmatrix}
        \varepsilon(h) & 0 & 0\\
        0 & \varepsilon(h) & 0\\
        0 & 0 & \varepsilon(h)\\
    \end{pmatrix}
    \]
    Значит $\zeta(H)$ - подалгебра скалярных матриц. Такая модульная структура эквивалентна структуре \ref{scalar}.
    
    \textbf{Далее будем полагать линейные функции $\varepsilon,\ \beta$ линейно независимыми.}

    Предположим, что функции $\alpha,\ \varepsilon,\ \beta$ линейно зависимы, тогда существуют $\lambda, \mu$ такие, что $\alpha = \lambda \varepsilon + \mu \beta$. 
    Подставив в равенство $h = 1_H$, получим $0 = \lambda + \mu$.
    Значит \[\alpha = \lambda(\varepsilon - \beta)\]
    В таком случае $\alpha$ коммутирует с $\beta$, а значит, с учётом (\ref{3})
    \[
        3\alpha\beta^2 = 3\lambda(\varepsilon-\beta)\beta^2 = 3\lambda(\beta^2-\beta^3)=0
    \]
    Тогда, с учётом $\mathrm{char} \ \mathbb{K} \neq 3$, получаем, что
    \[
        \lambda(\beta^2-\beta^3) = 0
    \]
    Так как $\beta$ группоподобный, значит существует обратный. Домножив это равенство на квадрат обратного к $\beta$:
    \[
        \lambda(\varepsilon-\beta) = 0
    \]
    А значит $\alpha = 0$. Значит $\alpha^2 = \alpha\beta+\beta\alpha = 0$. 
    
    \textbf{Случай 1.1}
    
    Если $\varepsilon, \beta, \beta^2$ - линейно независимы, тогда $\zeta(H)$ "--- это алгебра диагональных матриц. 
    Докажем, что тогда модульная структуре $\psi$ эквивалентна структуре \ref{diagonal}.
    Пусть $\psi_1: \mathbb{K}\langle c\rangle_3 \otimes A \to A$ "--- модульная структура \ref{diagonal}.
    Определим гоморфизм алгебр $\zeta_1:\mathbb{K}\langle c\rangle_3 \to M_3(\mathbb{K})$ с помощью соотношения ${\zeta_1(h)(a)=\psi_1(h \otimes a)}$. Тогда получаем \[
    \zeta(c) = \begin{pmatrix}
        1 & 0 & 0\\
        0 & \xi & 0\\
        0 & 0 & \xi^2\\
    \end{pmatrix}
    \]
    Выберем в $\mathbb{K}\langle c\rangle_3$  базис $(c^k)_{0 \leq k \leq 2}$ и выпишем образ этого базиса:
    \[
    \zeta_1(1) = \begin{pmatrix}
        1 & 0 & 0\\
        0 & 1 & 0\\
        0 & 0 & 1\\
    \end{pmatrix},\
    \zeta_1(c) = \begin{pmatrix}
        1 & 0 & 0\\
        0 & \xi & 0\\
        0 & 0 & \xi^2\\
    \end{pmatrix},\ 
    \zeta_1(c^2) = \begin{pmatrix}
        1 & 0 & 0\\
        0 & \xi^2 & 0\\
        0 & 0 & \xi\\
    \end{pmatrix}
    \]

    Рассмотрим эти матрицы как вектора в базисе из матричных единиц и запишем их координаты построчно в матрицу: \[
        M = \begin{pmatrix}
            1 & 0 & 0 & 0 & 1 & 0 & 0 & 0 & 1\\
            1 & 0 & 0 & 0 & \xi & 0 & 0 & 0 & \xi^2\\
            1 & 0 & 0 & 0 & \xi^2 & 0 & 0 & 0 & \xi\\
        \end{pmatrix}
    \]
    Нетрудно проверить, что $\mathrm{rank}\ M = 3$, а значит $\dim \mathrm{cosupp}\ \psi_1 = \dim \zeta(H) = 3$, следовательно $\mathrm{cosupp}\ \psi_1$ "--- алгебра диагональных матриц.

    \textbf{Случай 1.2}
    
    $\beta^2$ линейно зависим с $\varepsilon$ и $\beta$. Тогда существуют $\lambda, \mu$ такие, что $\beta^2 = \lambda \varepsilon+\mu \beta$. Подставим $h = 1_H$ и получим $1 = \lambda+\mu$, а значит $\beta^2 = \lambda \varepsilon + (1 - \lambda)\beta$
    
    Из группоподобности $\varepsilon,\ \beta$ и $\beta^2$ получаем 
    \[
    \varepsilon(\varepsilon^{(1)}\beta^{(1)}) = 0,\ \beta(\varepsilon^{(1)}\beta^{(1)}) = 0,\ \beta^2(\varepsilon^{(1)}\beta^{(1)}) = \lambda(1-\lambda)
    \]
    Используя эти результаты, подставим в разложение функции $\beta^2$  $\varepsilon^{(1)}\beta^{(1)}$, где $\varepsilon^{(1)}$ и $\beta^{(1)}$ "--- это единицы соответствующих линейных функций. Получим 
    \[\lambda(1-\lambda) = \lambda*0 + (1-\lambda)*0 = 0\]

    Если $\lambda = 0$, значит $\beta^2 = \beta$. Домножим это равенство на обратный к $\beta$ и получим $\beta = \varepsilon$, что противоречит линейной независимости $\varepsilon$ и $\beta$.

    Следовательно $\lambda = 1$ и $\beta^2 = \beta$. Докажем, что в таком случае модульная структура $\psi$ эквивалентна структуре \ref{eps=b^2}.

    Пусть $\psi_1: \mathbb{K}\langle c\rangle_2 \otimes A \to A$ "--- модульная структура \ref{eps=b^2}.
    Определим гоморфизм алгебр $\zeta_1:\mathbb{K}\langle c\rangle_2 \to M_3(\mathbb{K})$ с помощью соотношения ${\zeta_1(h)(a)=\psi_1(h \otimes a)}$. Тогда получаем \[
    \zeta(c) = \begin{pmatrix}
        1 & 0 & 0\\
        0 & -1 & 0\\
        0 & 0 & 1\\
    \end{pmatrix}
    \]

    Выберем в $\mathbb{K}\langle c\rangle_3$  базис $(c^k)_{0 \leq k \leq 1}$ и выпишем образ этого базиса:
    \[
    \zeta_1(1) = \begin{pmatrix}
        1 & 0 & 0\\
        0 & 1 & 0\\
        0 & 0 & 1\\
    \end{pmatrix},\
    \zeta_1(c) = \begin{pmatrix}
        1 & 0 & 0\\
        0 & -1 & 0\\
        0 & 0 & 1\\
    \end{pmatrix}
    \]

    Рассмотрим эти матрицы как вектора в базисе из матричных единиц и запишем их координаты построчно в матрицу: \[
        M = \begin{pmatrix}
            1 & 0 & 0 & 0 & 1 & 0 & 0 & 0 & 1\\
            1 & 0 & 0 & 0 & -1 & 0 & 0 & 0 & 1\\
        \end{pmatrix}
    \]
    Нетрудно проверить, что $\mathrm{rank}\ M = 2$, а значит $\dim \mathrm{cosupp}\ \psi_1 = \dim \zeta(H) = 2$, следовательно $\mathrm{cosupp}\ \psi_1$ "--- подалгебра диагональных матриц, такая что для любой матрицы $B$, содержащейся в ней, верно $b_{11}=b_{33}$.
    
    \textbf{Далее будем полагать $\varepsilon, \beta$ и $\alpha$ линейно независимыми.}

    \textbf{Случай 2.1}
    
    \noindent Предположим, что $\alpha^2 = \lambda_1\varepsilon + \lambda_2\beta + \lambda_3 \alpha$, тогда подставив $h = 1_H$, получим $0 = \lambda_1+\lambda_2$. Обозначим $\lambda = \lambda_1,\ \mu = \lambda_3$ и перепишем разложение $\alpha^2$ с учётом этого.
    \[\alpha^2 = \lambda(\varepsilon - \beta) + \mu \alpha\]
    Домножив слева и справа на $\alpha$, получим 
    \begin{eqnarray*}
        \lambda(\alpha-\alpha\beta) + \mu\alpha^2 = 0\\
        \lambda(\alpha-\beta\alpha) + \mu\alpha^2 = 0
    \end{eqnarray*}
    Пусть $\lambda \neq 0$, тогда $\alpha\beta = \beta\alpha$, иначе говоря $\alpha$ и $\beta$ коммутируют. С учётом этого и  (\ref{3}) получаем
    \[
    \alpha\beta^2+\beta\alpha\beta+\beta^2\alpha = 3\alpha\beta^2 = 0
    \]

    Тогда домножим это тождество на $\beta^{-2}$ и, учитывая что $\mathrm{char}\ \mathbb{K} \neq 3$ выведем, что $\alpha = 0$. Это противоречит линейной независимости $\varepsilon, \alpha$ и $\beta$.

    Значит $\lambda = 0$, иначе говоря $\alpha^2 = \mu\alpha$. Тогда из (\ref{1}) следует 
    \[
    \alpha^3 = \mu\alpha^2 = \mu^2\alpha = 0
    \]
    Значит $\mu = 0$. 

    \textbf{Случай 2.1.1}

    Функции $\varepsilon,\ \alpha,\ \beta$ линейно независимы, $\alpha^2 =0$ и функция $\alpha\beta+\beta\alpha$ линейна независима с $\varepsilon,\ \alpha,\ \beta$.
    Тогда \[
    \alpha^2\left(\left(\alpha^{(1)}(\alpha\beta+\beta\alpha)^{(1)}\right)\right) = \alpha(\alpha^{(1)})(\alpha\beta+\beta\alpha)\left((\alpha\beta+\beta\alpha)^{(1)}\right) = 1
    \]
    что невозможно.

    \textbf{Случай 2.1.2}

    Функции $\varepsilon,\ \alpha,\ \beta$ линейно независимы, $\alpha^2 =0$ и функция $\alpha\beta+\beta\alpha$ линейно выражается через $\varepsilon,\ \alpha,\ \beta$.

    Пусть $\alpha\beta + \beta\alpha = \lambda_1\varepsilon+\lambda_2\beta + \mu\alpha$. Подставим $h = 1_H$ и получим, что $0 = \lambda_1 + \lambda_2$. Тогда обозначим $\lambda = \lambda_1,\ \mu = \lambda_3$ и перепишем разложение $\alpha\beta+\beta\alpha$ с помощью этого: \[
    \alpha\beta+\beta\alpha = \lambda(\varepsilon-\beta) + \mu\alpha
    \]
    
    С учётом $\alpha^2=0$ и (\ref{2}) получим \[
    \alpha^2\beta+\alpha\beta\alpha+\beta\alpha^2 = \alpha\beta\alpha = 0
    \]

    Домножим разложение $\alpha\beta+\beta\alpha$ на $\alpha$ с обеих сторон 
    \begin{eqnarray*}
        \lambda(\alpha-\alpha\beta) = 0\\
        \lambda(\alpha-\beta\alpha) = 0
    \end{eqnarray*}
    Пусть $\lambda \neq 0$, тогда $\alpha\beta=\beta\alpha = \alpha$. Из (\ref{3}) и того, что $\mathrm{char}\ \mathbb{K} \neq 3$, получаем $\alpha = 0$, что невозможно.
    Значит $\lambda = 0$, иначе говоря $\alpha\beta+\beta\alpha = \mu\alpha$.

    Тогда 
    \begin{eqnarray*}
    (\alpha\beta+\beta\alpha)(\varepsilon^{(1)}\alpha^{(1)}) = \alpha(\varepsilon^{(1)})(\alpha\beta+\beta\alpha)(\alpha^{(1)}) = 0\\
    \alpha(\varepsilon^{(1)}\alpha^{(1)}) = \varepsilon(\varepsilon^{(1)})\alpha(\alpha^{(1)}) = 1
    \end{eqnarray*}

    Подставим в разложение $\alpha\beta+\beta\alpha$ $\varepsilon^{(1)}\alpha^{(1)}$ и, воспользовавшись результатами выше, получим $0 = \mu$, иначе говоря $\alpha\beta+\beta\alpha = 0$

    Тогда из (\ref{3}) следует \[
    \alpha\beta^2+\beta(\alpha\beta+\beta\alpha) = \alpha\beta^2=0
    \]
    Так как $\beta$ группоподобный, значит существует $\beta^{-1}$. Домножим на квадрат обратного справа и получим  $\alpha = 0$, что противоречит линейной независимости $\varepsilon,\ \beta,\ \alpha$.
    
    \textbf{Случай 2.2}

    \textbf{Далее полагаем, что $\varepsilon, \alpha, \beta, \alpha^2$ линейно независимы.}

    \textbf{Случай 2.2.1}

    Функции $\varepsilon,\ \alpha,\ \alpha^2,\ \beta$ линейно независимы и функция $\alpha\beta+\beta\alpha$ линейно выражается через $\varepsilon,\ \alpha,\ \alpha^2,\ \beta$.
    Тогда существуют $\lambda_i$ такие, что \[
    \alpha\beta+\beta\alpha = \lambda_1\varepsilon+\lambda_2\beta +\lambda_3\alpha+\lambda_4\alpha^2
     \]

    Подставим $h = 1_H$, получим $0 = \lambda_1+\lambda_2$. Тогда обозначим $\lambda = \lambda_1,\ {\mu_1 = \lambda_3,}$ ${\mu_2 = \lambda_4}$ и перепишем разложение $\alpha\beta+\beta\alpha$ с помощью этого: \[
    \alpha\beta+\beta\alpha = \lambda(\varepsilon-\beta)+\mu_1\alpha+\mu_2\alpha^2
    \]

    Рассмотрим (\ref{2}): 
    \[
    \alpha(\alpha\beta+\beta\alpha)+\beta\alpha^2 = 0
    \]
    Значит с учётом $\alpha^3 =0$: \[
    \alpha(\alpha\beta+\beta\alpha)\alpha = (-\beta\alpha^2)\alpha = 0
    \]

    Тогда домножим разложение $\alpha\beta+\beta\alpha$ на $\alpha$ с обеих сторон:\[
    0 = \lambda(\alpha^2-\alpha\beta\alpha)
    \]
    Пусть $\lambda \neq 0$, тогда $\alpha^2 = \alpha\beta\alpha$. Домножим $\alpha\beta+\beta\alpha$ на $\alpha^2$ слева и выпишем, что получится:
    \begin{eqnarray*}
        \alpha^2(\alpha\beta+\beta\alpha) = \alpha^2\beta\alpha= \alpha^3 = 0\\
        (\alpha\beta+\beta\alpha)\alpha^2= \alpha\beta\alpha^2 = \alpha^3 = 0\\
        \alpha^2(\alpha\beta+\beta\alpha) = \lambda(\alpha^2 - \alpha^2\beta)\\
        (\alpha\beta+\beta\alpha)\alpha^2 = \lambda(\alpha^2 - \beta\alpha^2)\\
    \end{eqnarray*}
    Отсюда, с учётом предположения, что $\lambda\neq 0$, следует, что $\alpha^2 = \alpha^2\beta= \beta\alpha^2$. Подставим это в (\ref{2}), используя $\alpha^2 = \alpha\beta\alpha$, тогда
    \[
        \alpha^2\beta+\alpha\beta\alpha+\beta\alpha^2 = 3\alpha^2=0
    \]
    Что невозможно, так как $\mathrm{char}\ \mathbb{K} \neq 3$ и функция $\alpha^2$ линейно независима с $\varepsilon,\ \alpha,\ \beta$. Значит $\lambda = 0$, иначе говоря: \[
    \alpha\beta+\beta\alpha = \mu_1\alpha+\mu_2\alpha^2
    \]

    Подставим разложение $\alpha\beta+\beta\alpha$ в (\ref{2})
    \begin{eqnarray*}
        \alpha(\alpha\beta+\beta\alpha)+\beta\alpha^2 = \mu_1\alpha^2 + \beta\alpha^2 = 0\\
        \alpha^2\beta+(\alpha\beta+\beta\alpha)\alpha = \alpha^2\beta + \mu_1\alpha^2 = 0\\
    \end{eqnarray*}
    Сложим эти два равенства с учётом (\ref{2}) получим \[
    2\mu_1\alpha^2 = \alpha\beta\alpha
    \]
    
    Теперь домножим $\alpha\beta+\beta\alpha$ на $\alpha\beta$ слева :
    \[
        \alpha\beta(\alpha\beta+\beta\alpha) = \alpha(-\alpha\beta^2) = -\alpha^2\beta^2 = -\mu_1^2\alpha^2
    \]
    \[
        \alpha\beta(\mu_1\alpha+\mu_2\alpha^2) = \mu_1\alpha\beta\alpha+\mu_2\alpha\beta\alpha^2=\mu_1\alpha\beta\alpha-\mu_1\mu_2\alpha^3 = \mu_1\alpha\beta\alpha
    \]
    А значит \[
    \mu_1\alpha\beta\alpha = -\mu_1^2\alpha^2
    \]
    С учётом полученного выше соотношения приходит к выводу, что 
    \begin{eqnarray*}
        2\mu_1^2\alpha^2 &=& -\mu_1^2\alpha^2\\
        3\mu_1^2\alpha^2 &=& 0
    \end{eqnarray*}
    Тогда, так как $\mathrm{char}\ \mathbb{K} \neq 3$ и функция $\alpha^2$ линейно независима с $\varepsilon,\ \alpha,\ \beta$, а значит $\alpha^2 \neq 0$, получаем, что $\mu_1 =0 $.
    Тогда \[
    \alpha\beta+\beta\alpha = \mu_2\alpha^2
    \]
    Но если мы рассмотрим
    \begin{multline*}
        \alpha^2 \left(\varepsilon^{(1)}\left(\alpha^2\right)^{(1)} \right) = \varepsilon(\varepsilon^{(1)})\alpha^2 \left(\left(\alpha^2\right)^{(1)}\right) +
        \\
        +\alpha(\varepsilon^{(1)})(\alpha\beta+\beta\alpha)\left(\left(\alpha^2\right)^{(1)}\right) + \alpha^2(\varepsilon^{(1)}) \beta^2 \left(\left(\alpha^2\right)^{(1)}\right) = 1
    \end{multline*}
    \begin{multline*}
        (\alpha\beta+\beta\alpha)\left(\varepsilon^{(1)}\left(\alpha^2\right)^{(1)} \right) = \beta(\varepsilon^{(1)})(\alpha\beta+\beta\alpha)\left(\left(\alpha^2\right)^{(1)}\right)+
        \\
        +(\alpha\beta+\beta\alpha)(\varepsilon^{(1)})\beta^2\left(\left(\alpha^2\right)^{(1)}\right) = 0
    \end{multline*}
    Тогда, если мы подставим в разложение $\alpha\beta+\beta\alpha$ $\varepsilon^{(1)}\left(\alpha^2\right)^{(1)}$, получим \[
    0 = \mu_2*1
    \]
    Значит $\alpha\beta+\beta\alpha = 0$. Отсюда с учётом (\ref{3}) получаем 
    \[
        \beta(\alpha\beta+\beta\alpha)+\alpha\beta^2 = \alpha\beta^2 = 0\\
    \]
    Но тогда домножив на обратный к $\beta^2$ получим $\alpha =0$, что противоречит линейной независимости $\alpha$ с $\varepsilon,\ \alpha^2,\ \beta$.
    
    \textbf{Случай 2.2.2}

    \textbf{Далее полагаем Функции $\varepsilon,\ \alpha,\ \alpha^2,\ \beta,\ \alpha\beta+\beta\alpha$ линейно независимыми.}

    \textbf{Случай 2.2.2.1}
    
    Функции $\varepsilon,\ \alpha,\ \alpha^2,\ \beta,\ \alpha\beta+\beta\alpha$ линейно независимы и функция $\beta^2$ линейно выражается через $\varepsilon,\ \alpha,\ \alpha^2,\ \beta$.
    Тогда существуют $\lambda_i$ такие, что \[
    \beta^2= \lambda_1\varepsilon+\lambda_2\beta +\lambda_3\alpha+\lambda_4\alpha^2 + \lambda_5(\alpha\beta+\beta\alpha)
     \]

    Подставим $h = 1_H$, получим $1 = \lambda_1+\lambda_2$. Тогда обозначим $\lambda = \lambda_1,\ {\mu_1 = \lambda_3,}$ ${\mu_2 = \lambda_4},\ {\mu_3 = \lambda_5}$ и перепишем разложение $\alpha\beta+\beta\alpha$ с помощью этого: 
    \[
    \beta^2 = \lambda\varepsilon + (1-\lambda)\beta+\mu_1\alpha+\mu_2\alpha^2 +\mu_3(\alpha\beta+\beta\alpha)
    \]

    Рассмотрим значения наших функций от $\left(\alpha^{(1)}\right)^2$
    \begin{eqnarray*}
    \varepsilon\left(\left(\alpha^{(1)}\right)^2\right) &=& 
    \varepsilon\left(\alpha^{(1)}\right)\varepsilon\left(\alpha^{(1)}\right) = 0\\
    %
    \alpha\left(\left(\alpha^{(1)}\right)^2\right) &=&
    \varepsilon\left(\alpha^{(1)}\right)\alpha\left(\alpha^{(1)}\right)+\alpha\left(\alpha^{(1)}\right)\beta\left(\alpha^{(1)}\right) = 0\\
    %
    \beta\left(\left(\alpha^{(1)}\right)^2\right) &=&
    \beta\left(\alpha^{(1)}\right)\beta\left(\alpha^{(1)}\right) = 0\\
    %
    \alpha^2\left(\left(\alpha^{(1)}\right)^2\right) &=&
    \varepsilon\left(\alpha^{(1)}\right)\alpha^2\left(\alpha^{(1)}\right)+
    \alpha\left(\alpha^{(1)}\right)(\alpha\beta+\beta\alpha)\left(\alpha^{(1)}\right)+\\
    &+& \alpha^2\left(\alpha^{(1)}\right)\beta\left(\alpha^{(1)}\right) = 0\\
    %
    (\alpha\beta+\beta\alpha)\left(\left(\alpha^{(1)}\right)^2\right) &=&
    \beta\left(\alpha^{(1)}\right)(\alpha\beta+\beta\alpha)\left(\alpha^{(1)}\right)+\\
    &+&(\alpha\beta+\beta\alpha)\left(\alpha^{(1)}\right)\beta^2\left(\alpha^{(1)}\right) = 0\\
    %
    \beta^2\left(\left(\alpha^{(1)}\right)^2\right) &=& \beta^2\left(\alpha^{(1)}\right)\beta^2\left(\alpha^{(1)}\right) = \mu_1^2\\
    \end{eqnarray*}

    Тогда получим \[
    \beta^2\left(\left(\alpha^{(1)}\right)^2\right) = \mu_1^2 = 0
    \]
    Значит $\mu_1 = 0$, иначе говоря \[
    \beta^2 = \lambda\varepsilon+(1-\lambda)\beta + \mu_2\alpha^2 + \mu_3(\alpha\beta+\beta\alpha)
    \]

    Рассмотрим значения наших функций от $\alpha^{(1)}(\alpha\beta+\beta\alpha)^{(1)}$
    \begin{eqnarray*}
    \varepsilon\left(\alpha^{(1)}(\alpha\beta+\beta\alpha)^{(1)}\right) &=& 
    \varepsilon\left(\alpha^{(1)}\right)\varepsilon\left((\alpha\beta+\beta\alpha)^{(1)}\right) = 0\\
    %
    \alpha\left(\alpha^{(1)}(\alpha\beta+\beta\alpha)^{(1)}\right) &=&
    \varepsilon\left(\alpha^{(1)}\right)\alpha\left((\alpha\beta+\beta\alpha)^{(1)}\right)+\\
    &+&\alpha\left(\alpha^{(1)}\right)\beta\left((\alpha\beta+\beta\alpha)^{(1)}\right) = 0\\
    %
    \beta\left(\alpha^{(1)}(\alpha\beta+\beta\alpha)^{(1)}\right) &=&
    \beta\left(\alpha^{(1)}\right)\beta\left((\alpha\beta+\beta\alpha)^{(1)}\right) = 0\\
    %
    \alpha^2\left(\alpha^{(1)}(\alpha\beta+\beta\alpha)^{(1)}\right) &=&
    \varepsilon\left(\alpha^{(1)}\right)\alpha^2\left((\alpha\beta+\beta\alpha)^{(1)}\right)+\\
    &+&
    \alpha\left(\alpha^{(1)}\right)(\alpha\beta+\beta\alpha)\left((\alpha\beta+\beta\alpha)^{(1)}\right)+\\
    &+& \alpha^2\left(\alpha^{(1)}\right)\beta\left((\alpha\beta+\beta\alpha)^{(1)}\right) = 1\\
    %
    (\alpha\beta+\beta\alpha)\left(\alpha^{(1)}(\alpha\beta+\beta\alpha)^{(1)}\right) &=&
    \beta\left(\alpha^{(1)}\right)(\alpha\beta+\beta\alpha)\left((\alpha\beta+\beta\alpha)^{(1)}\right)+\\
    &+&(\alpha\beta+\beta\alpha)\left(\alpha^{(1)}\right)\beta^2\left((\alpha\beta+\beta\alpha)^{(1)}\right) = 0\\
    %
    \beta^2\left(\alpha^{(1)}(\alpha\beta+\beta\alpha)^{(1)}\right) &=& \beta^2\left(\alpha^{(1)}\right)\beta^2\left((\alpha\beta+\beta\alpha)^{(1)}\right) = 0\\
    \end{eqnarray*}

    Значит $\beta^2\left(\alpha^{(1)}(\alpha\beta+\beta\alpha)^{(1)}\right) = \mu_2 = 0$
    Тогда разложения выглядит таким образом:\[
    \beta^2 = \lambda\varepsilon+ (1-\lambda)\beta + \mu_3(\alpha\beta+\alpha\beta)
    \]

    Рассмотрим значения наших функций от $\varepsilon^{(1)}(\alpha\beta+\beta\alpha)^{(1)}$
    \begin{eqnarray*}
    \varepsilon\left(\varepsilon^{(1)}(\alpha\beta+\beta\alpha)^{(1)}\right) &=& 
    \varepsilon\left(\varepsilon^{(1)}\right)\varepsilon\left((\alpha\beta+\beta\alpha)^{(1)}\right) = 0\\
    %
    \alpha\left(\varepsilon^{(1)}(\alpha\beta+\beta\alpha)^{(1)}\right) &=&
    \varepsilon\left(\varepsilon^{(1)}\right)\alpha\left((\alpha\beta+\beta\alpha)^{(1)}\right)+\\
    &+&\alpha\left(\varepsilon^{(1)}\right)\beta\left((\alpha\beta+\beta\alpha)^{(1)}\right) = 0\\
    %
    \beta\left(\varepsilon^{(1)}(\alpha\beta+\beta\alpha)^{(1)}\right) &=&
    \beta\left(\varepsilon^{(1)}\right)\beta\left((\alpha\beta+\beta\alpha)^{(1)}\right) = 0\\
    %
    \alpha^2\left(\varepsilon^{(1)}(\alpha\beta+\beta\alpha)^{(1)}\right) &=&
    \varepsilon\left(\varepsilon^{(1)}\right)\alpha^2\left((\alpha\beta+\beta\alpha)^{(1)}\right)+\\
    &+&
    \alpha\left(\varepsilon^{(1)}\right)(\alpha\beta+\beta\alpha)\left((\alpha\beta+\beta\alpha)^{(1)}\right)+\\
    &+& \alpha^2\left(\varepsilon^{(1)}\right)\beta\left((\alpha\beta+\beta\alpha)^{(1)}\right) = 0\\
    %
    (\alpha\beta+\beta\alpha)\left(\varepsilon^{(1)}(\alpha\beta+\beta\alpha)^{(1)}\right) &=&
    \beta\left(\varepsilon^{(1)}\right)(\alpha\beta+\beta\alpha)\left((\alpha\beta+\beta\alpha)^{(1)}\right)+\\
    &+&(\alpha\beta+\beta\alpha)\left(\varepsilon^{(1)}\right)\beta^2\left((\alpha\beta+\beta\alpha)^{(1)}\right) = 0\\
    %
    \beta^2\left(\varepsilon^{(1)}(\alpha\beta+\beta\alpha)^{(1)}\right) &=& \beta^2\left(\varepsilon^{(1)}\right)\beta^2\left((\alpha\beta+\beta\alpha)^{(1)}\right) = \lambda\mu_3\\
    \end{eqnarray*}

    Значит $\beta^2\left(\varepsilon^{(1)}(\alpha\beta+\beta\alpha)^{(1)}\right) = \lambda\mu_3 = 0$.

    Пусть $\mu_3 \neq 0$, тогда $\lambda = 0$ и $\beta^2 = \beta + \mu_3(\alpha\beta+\beta\alpha)$.
    Домножим на $\beta$ слева и справа и с учётом (\ref{3}) сложим 
    \begin{eqnarray*}
        \beta^3 &=& \beta^2 + \mu_3 \beta(\alpha\beta+\beta\alpha)\\
        \beta^3 &=& \beta^2 + \mu_3 (\alpha\beta+\beta\alpha)\beta\\
        2\beta^3 &=& 2\beta^2 + \mu_3 \beta\alpha\beta\\
    \end{eqnarray*}
    Домножив последнее равенство на обратный к $\beta$ с обеих сторон и получим
    \[
    2\beta = 2\varepsilon+\mu_2\alpha
    \]
    Так как $\varepsilon,\ \alpha,\ \beta$ линейно независимы значит $\mu_3 = 0$. Получаем противоречие с нашим предположением $\mu_3 \neq 0$. Значит $\mu_3 = 0$.

    Тогда $\beta^2 =\lambda\varepsilon + (1-\lambda)\beta$.
    
    Из группоподобности $\varepsilon,\ \beta$ и $\beta^2$ получаем 
    \[
    \varepsilon(\varepsilon^{(1)}\beta^{(1)}) = 0,\ \beta(\varepsilon^{(1)}\beta^{(1)}) = 0,\ \beta^2(\varepsilon^{(1)}\beta^{(1)}) = \lambda(1-\lambda)
    \]
    Используя эти результаты, подставим в разложение функции $\beta^2$  $\varepsilon^{(1)}\beta^{(1)}$, где $\varepsilon^{(1)}$ и $\beta^{(1)}$ "--- это единицы соответствующих линейных функций. Получим 
    \[\lambda(1-\lambda) = \lambda*0 + (1-\lambda)*0 = 0\]

    Если $\lambda = 0$, значит $\beta^2 = \beta$. Домножим это равенство на обратный к $\beta$ и получим $\beta = \varepsilon$, что противоречит линейной независимости $\varepsilon$ и $\beta$.

    Если $\lambda = 1$, тогда $\beta^2 = \varepsilon$.
    Рассмотрим (\ref{3})
    \[
    \beta^2\alpha+ \beta\alpha\beta+\alpha\beta^2 = 2\alpha + \beta\alpha\beta = 0 \eqno(*)
    \]
    Домножим на $\alpha$ с обеих сторон 
    \[
    2\beta\alpha\beta + \beta^2\alpha\beta^2 = 2\beta\alpha\beta + \alpha = 0
    \]
    Выразим из $(*)$ $\beta\alpha\beta$ через $\alpha$ и подставим в соотношение выше, получим
    \[
    -4\alpha+\alpha = -3\alpha = 0
    \]
    Так как $\mathrm{char}\ \mathbb{K}\neq 3$, получаем $\alpha = 0$, что невозможно ввиду линейной независимости $\varepsilon,\ \alpha,\ \beta,\ \alpha^2,\ \alpha\beta+\beta\alpha$.
\end{prof}

\end{document}
