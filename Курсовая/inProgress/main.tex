\documentclass[12pt, reqno, a4paper, oneside, notitlepage]{amsart}

\usepackage{color}
\definecolor{darkblue}{rgb}{0,0,0.75}
\definecolor{darkgreen}{rgb}{0,0.75,0}
\usepackage[%pagebackref,
colorlinks, linkcolor=darkblue,
citecolor=darkgreen, urlcolor=blue, bookmarks=false, breaklinks]{hyperref}

\usepackage{cmap}

\usepackage[utf8]{inputenc}
\usepackage[T2A]{fontenc}
\usepackage[english,russian]{babel}

\usepackage{indentfirst}
\usepackage[left=2.5cm, right=1.5cm, top=2cm, bottom=2cm]{geometry}
\setlength\parindent{5ex}

\usepackage{fancyhdr}
\fancypagestyle{plain}{
	\fancyhf{}
	\chead{\thepage}
	\renewcommand{\headrulewidth}{0pt}}
\linespread{1.25}

\usepackage{tocloft}
\usepackage[ddmmyyyy]{datetime}

\usepackage{titlesec}
\titleformat{\section}[block]{\bfseries\large\centering}{\thesection}{1ex}{}
\titleformat{\subsection}[block]{\bfseries\normalsize\centering}{\thesubsection}{1ex}{}

\usepackage{amssymb, amsmath, amsthm, longtable, bbm}
\usepackage[matrix, arrow, curve]{xy}

\usepackage[geometry]{ifsym}

\renewcommand{\square}{\text{\SmallSquare}}

\binoppenalty=\maxdimen
\relpenalty=\maxdimen

\righthyphenmin=2
\sloppy

\newtheoremstyle{mytheoremstyle} % name
{\topsep}                    % Space above
{\topsep}                    % Space below
{\itshape}                   % Body font
{5ex}                           % Indent amount
{\bfseries}                   % Theorem head font
{.}                          % Punctuation after theorem head
{.5em}                       % Space after theorem head
{}  % Theorem head spec (can be left empty, meaning ‘normal’)

\newtheoremstyle{myremarkstyle} % name
{\topsep}                    % Space above
{\topsep}                    % Space below
{}                   % Body font
{5ex}                           % Indent amount
{\bfseries}                   % Theorem head font
{.}                          % Punctuation after theorem head
{.5em}                       % Space after theorem head
{}  % Theorem head spec (can be left empty, meaning ‘normal’)

\theoremstyle{mytheoremstyle}
\newtheorem{theorem}{Теорема}[section]
\newtheorem{proposition}[theorem]{Предложение}
\newtheorem{corollary}[theorem]{Следствие}
\newtheorem{lemma}[theorem]{Лемма}
\newtheorem*{conjecture}{Гипотеза}

\theoremstyle{myremarkstyle}
\newtheorem{remark}[theorem]{Замечание}
\newtheorem{agreement}[theorem]{Соглашение}
\newtheorem{example}[theorem]{Пример}
\newtheorem*{examples}{Примеры}
\newtheorem*{questions}{Вопросы}
\newtheorem*{question}{Вопрос}
\newtheorem{definition}[theorem]{Определение}
\newtheorem{problem}[theorem]{Задача}

\numberwithin{equation}{section}

\makeatletter
\renewenvironment{proof}[1][\proofname]{\par\indent {\bfseries #1\@addpunct{.} }}{\qed}
\makeatother

\DeclareMathOperator{\Id}{Id}
\DeclareMathOperator{\Der}{Der}
\DeclareMathOperator{\id}{id}
\DeclareMathOperator{\chr}{char}
\DeclareMathOperator{\ad}{ad}
\DeclareMathOperator{\tr}{tr}
\DeclareMathOperator{\GL}{GL}
\DeclareMathOperator{\UT}{UT}
\DeclareMathOperator{\SL}{SL}
\DeclareMathOperator{\diff}{diff}
\DeclareMathOperator{\supp}{supp}
\DeclareMathOperator{\cosupp}{cosupp}
\DeclareMathOperator{\PGL}{PGL}
\DeclareMathOperator{\PSL}{PSL}
\DeclareMathOperator{\height}{ht}
\DeclareMathOperator{\End}{End}
\DeclareMathOperator{\Aut}{Aut}
\DeclareMathOperator{\diag}{diag}
\DeclareMathOperator{\rank}{rank}
\DeclareMathOperator{\length}{length}
\DeclareMathOperator{\sign}{sign}
\DeclareMathOperator{\Alt}{Alt}
\DeclareMathOperator{\Ann}{Ann}
\DeclareMathOperator{\Hom}{Hom}
\DeclareMathOperator{\op}{op}
\DeclareMathOperator{\PIexp}{PIexp}
\newcommand{\hatotimes}{\mathbin{\widehat{\otimes}}}
\DeclareMathOperator{\Ker}{Ker}
\DeclareMathOperator{\Coker}{Coker}
\renewcommand{\Im}{\mathop\mathrm{Im}}
\DeclareMathOperator{\Coim}{Coim}
\DeclareMathOperator{\Ext}{Ext}
\DeclareMathOperator{\Tor}{Tor}

\DeclareMathOperator{\fld}{\mathbbm{k}}
\DeclareMathOperator{\kG}{\mathbbm{k}G}

%%%%%%%%%%%%%%% Fancy symbol \No %%%%%%%%%%%%%%%%%%%%%%%%%%%%%%%%%%%%%%%%%%%%%%

\DeclareRobustCommand{\No}{\ifmmode{\nfss@text{\textnumero}}\else\textnumero\fi} 


%%%%%%%%%%%%%%% Pullbacks van Tim van der Linden %%%%%%%%%%%%%%%%%%%%%%%%%%%%%

\newbox\skewpullbackbox
\setbox\skewpullbackbox=\hbox{\xy 0;<1mm,0mm>: \POS(4,0)\ar@{-} (-4,0) \ar@{-} (8,4)
	\endxy}
\newcommand{\skewpullback}{\copy\skewpullbackbox}

\newbox\skwepullbackbox
\setbox\skwepullbackbox=\hbox{\xy 0;<1mm,0mm>: \POS(16,0)\ar@{-} (10,0) \ar@{-} (12,4)
	\endxy}
\newcommand{\skwepullback}{\copy\skwepullbackbox}

\newbox\ksewpullbackbox
\setbox\ksewpullbackbox=\hbox{\xy 0;<1mm,0mm>: \POS(0,-8)\ar@{-} (0,-4) \ar@{-} (4,-4)
	\endxy}
\newcommand{\ksewpullback}{\copy\ksewpullbackbox}

\newbox\pullbackbox
\setbox\pullbackbox=\hbox{\xy 0;<1mm,0mm>: \POS(4,0)\ar@{-} (0,0) \ar@{-} (4,4)
	\endxy}
\newcommand{\pullback}{\copy\pullbackbox}

\newbox\pullbackabox
\setbox\pullbackabox=\hbox{\xy 0;<1mm,0mm>: \POS(-4,-6)\ar@{-} (-8,-6) \ar@{-} (-4,-2)
	\endxy}
\newcommand{\pullbacka}{\copy\pullbackabox}

\newbox\pullbackbbox
\setbox\pullbackbbox=\hbox{\xy 0;<1mm,0mm>: \POS(-4,-4)\ar@{-} (-8,-4) \ar@{-} (-4,0)
	\endxy}
\newcommand{\pullbackb}{\copy\pullbackbbox}

\newbox\pullbackcbox
\setbox\pullbackcbox=\hbox{\xy 0;<1mm,0mm>: \POS(4,-7)\ar@{-} (0,-7) \ar@{-} (4,-3)
	\endxy}
\newcommand{\pullbackc}{\copy\pullbackcbox}


\newbox\pullbackdbox
\setbox\pullbackdbox=\hbox{\xy 0;<1mm,0mm>:\POS(-10,-6)\ar@{-} (-14,-6) \ar@{-} (-10,-2)
	\endxy}
\newcommand{\pullbackd}{\copy\pullbackdbox}


\newbox\pushoutbox
\setbox\pushoutbox=\hbox{\xy 0;<1mm,0mm>: \POS(0,4)\ar@{-} (0,0) \ar@{-} (4,4)
	\endxy}
\newcommand{\pushout}{\copy\pushoutbox}

\newbox\pushoutabox
\setbox\pushoutabox=\hbox{\xy 0;<1mm,0mm>: \POS(2,8)\ar@{-} (2,4) \ar@{-} (8,8)
	\endxy}
\newcommand{\pushouta}{\copy\pushoutabox}

\begin{document}

{\fontsize{14pt}{18}\selectfont
  
  \thispagestyle{empty}
  \begin{center}
		
		\vfill\vfill \ \\ {%\Large
			МОСКОВСКИЙ ГОСУДАРСТВЕННЫЙ УНИВЕРСИТЕТ \\
			имени М.В.~ЛОМОНОСОВА
			
			\medskip
			
			МЕХАНИКО-МАТЕМАТИЧЕСКИЙ ФАКУЛЬТЕТ

			\medskip
			
			
			КАФЕДРА ВЫСШЕЙ АЛГЕБРЫ

		}



{%\Large
	\vfill {%\Large
		
		КУРСОВАЯ РАБОТА
		
	}
		
		
		\vfill{\Large
			\textbf{Квантовые симметрии \linebreak в алгебре тройных чисел} 
		}
		
			

			\vfill
			\begin{flushright}
				\begin{tabular}{r}
					Выполнил:\\
студент 211 группы \\
Зазовский Леон Станиславович
					\\
					\\
					Научный руководитель: \\
					доктор физико-математических наук, \\
					профессор Гордиенко Алексей Сергеевич
				\end{tabular}
			\end{flushright}
			
		}
		
		\vfill\vfill\vfill\vfill 
		
		Москва
		
		 2025
	\end{center}
}

\newpage

\thispagestyle{empty}
\tableofcontents

\newpage

\pagestyle{plain}
\section{Введение}

Во многих областях математики и физики (см., например, \cite{ArnoldBook, ModernGeometry, MurphyBook, HaagKastler}) находят своё применение \textit{алгебры}, то есть векторные пространства над некоторым полем $\mathbbm{k}$ (например, $\mathbbm{k}$ может быть полем $\mathbb{R}$ вещественных или $\mathbb{C}$ комплексных чисел), в которых задана бинарная операция внутреннего умножения, линейная по каждому аргументу. 

Часто алгебры, встречающиеся в приложениях, наделены некоторой дополнительной структурой или (обобщёнными) симметриями: действием группы автоморфизмами, групповой градуировкой или действием алгебры Ли дифференцированиями (см., например, \cite{PolyakovBook, HaagBook, KakuBook, MajidBook}). Для работы с такими дополнительными структурами оказывается полезным понятие модульной и комодульной алгебры над алгеброй Хопфа. В частности, понятие (ко)модульной алгебры позволяет изучать различные виды дополнительных структур на алгебрах одновременно.

Кроме того, (ко)модульные алгебры естественным образом возникают в геометрии: если некоторая аффинная алгебраическая группа действует морфизмами на аффинном алгебраическом многообразии (например, рассматриваются симметрии некоторой поверхности, заданной алгебраическими уравнениями), то алгебра регулярных функций (множество функций, которые можно определить при помощи многочленов от координат точки,
с операциями сложения, умножения между собой и на скаляры) будет модульной алгеброй над групповой алгеброй этой группы и над универсальной обёртывающей алгебры Ли этой группы и комодульной алгеброй над алгеброй регулярных функций на аффинной алгебраической группе~\cite{Abe}.

Обратим внимание на то, что в классическом случае алгебры регулярных функций на многообразиях коммутативны, так как для умножения функций выполнен перестановочный закон. Однако в новом направлении, которое получило название некоммутативной геометрии, рассматриваются <<некоммутативные пространства>>, т. е. такие пространства, алгебры регулярных функций которых некоммутативны. Поэтому изучение (ко)действий необязательно коммутативных алгебр Хопфа на необязательно коммутативных алгебрах можно интерпретировать как изучение квантовых симметрий некоммутативных пространств. Последние находят своё применение в теоретической физике (см., например, \cite{ConnesMarcolli, Donatsos}).

В данной работе разрабатываются методы для исследования квантовых симметрий на алгебрах, полученных из алгебры многочленов путем факторизования по различным идеалам. Для этого используются понятия эквивалентности модульных структр (введённое в \cite{ASGordienko21ALAgoreJVercruysse}) и структуры коносителя, введённое автором в данной работе. Основная идея исследования заключается в том, чтобы в случае конечномерности алгебры, на которой действует алгебра Хопфа, свести изучение квантовых симметрий к рассмотрению конечномерной подкоалгебры в конечной двойственной к действующей алгебре Хопфа.

\newpage

\section{Основные понятия}
\subsection{Алгебры и коалгебры}

В данном параграфе мы напомним основные понятия, связанные с (ко)алгебрами и алгебрами Хопфа.
Подробнее с этими понятиями можно познакомиться в монографиях~\cite{Danara, Montgomery, Sweedler}.

Прежде всего напомним понятие алгебры над полем в удобной для нас форме, а именно, на языке линейных отображений и коммутативных диаграмм. 

\begin{definition} \label{algebra}
	\textit{Алгеброй над полем $\mathbbm{k}$} называется пара $(A, \mu)$, состоящая из
	векторного пространства $A$ над $\mathbbm{k}$ и линейного отображения
	$\mu \colon A \otimes A \to A$.
\end{definition}

При помощи отображения $\mu$ на векторном пространстве $A$ задаётся операция внутреннего умножения, линейная по каждому аргументу:
$ab := \mu(a\otimes b)$ для всех $a,b\in A$.

Алгебра $(A, \mu)$ называется \textit{ассоциативной}, если
следующая диаграмма коммутативна: 
  \[\xymatrix{ 
	A \otimes A \otimes A \ar[d]_{\mu \otimes \id_A} \ar[rr]^{\id_A \otimes \mu} && A\otimes A
	\ar[d]_\mu \\
	A\otimes A \ar[rr]^\mu && A  
  }\]

  \begin{definition}
  Набор $(A, \mu, u)$ называется \textit{алгеброй с единицей над полем $\mathbbm{k}$}, если $(A, \mu)$ --- алгебра над $\mathbbm{k}$, а $u \colon \mathbbm{k} \to A$ --- линейное отображение и, кроме того, следующие диаграммы коммутативны:
  \[\xymatrix{ 
	  A \otimes \mathbbm{k} \ar[rd]^\sim \ar[rr]^{\id_A \otimes u}&& A\otimes A \ar[ld]^\mu \\
	  &A
  }
  \hspace{2em}
  \text{и}
  \hspace{2em}
  \xymatrix{ 
	  \mathbbm{k} \otimes A \ar[rd]^\sim \ar[rr]^{u \otimes \id_A}&& A\otimes A \ar[ld]^\mu \\
	  &A
  }\]
\end{definition}
\noindent(Здесь мы использовали естественные отждествления $A \otimes \mathbbm{k} \cong \mathbbm{k} \otimes A \cong A$.)

Если ввести обозначение $1_A :=u(1_\mathbbm{k})$, то элемент $1_A \in A$ будет удовлетворять классическому определению единицы в алгебре.

Определение коассоциативной коалгебры с коединицей получается из определения ассоциативной алгебры с единицей формальным обращением стрелок:

\begin{definition} \label{coalgebra}
  Набор $(C,\triangle,\varepsilon)$ называется \textit{коассоциативной коалгеброй с коединицей}, если $C$ --- векторное пространство над полем $\mathbbm{k}$, $\triangle: C \to C \otimes C$ и $\varepsilon : C \to \mathbbm{k}$ являются линейными отображениями и, кроме того, следующие диаграммы коммутативны: \begin{enumerate}
	\item (коассоциативность) \[
		\xymatrix{
		  C \ar[d]^{\triangle} \ar[rr]^{\triangle} && C\otimes C
		  \ar[d]^{\id_C \otimes \triangle} \\
		  C\otimes C \ar[rr]_{\triangle \otimes \id_C} && C \otimes C \otimes C  
		}\]
	\item (наличие коединицы) 
	  \[\xymatrix{
		  &C \ar[rd]^\sim  \ar[ld]_\triangle \\
	  C \otimes C \ar[rr]^{\id_C \otimes \varepsilon}&& C \otimes \mathbbm{k} \\
  }
  \hspace{2em}
  \text{и}
  \hspace{2em}
  \xymatrix{
		  &C \ar[rd]^\sim  \ar[ld]_\triangle \\
	  C \otimes C \ar[rr]^{\varepsilon \otimes \id_C}&& \mathbbm{k} \otimes C\\
  }\]

  \end{enumerate}
\end{definition}
\noindent(Здесь мы использовали естественные отждествления $A \otimes \mathbbm{k} \cong \mathbbm{k} \otimes A \cong A$.)

В рабботе под \textit{коалгебрами} понимаются именно коассоциативные коалгебры с коединицей.

Отображение $\triangle$ называется \textit{коумножение}, а отображение $\verepsilon$ называется \textit{коединицей} коалгебры $(C, \triangle, \varepsilon)$.

В работе используются обозначение М. Свидлера $\triangle c = c_{(1)} \otimes c_{(2)}$, где $c \in C$ и опущен знак суммы.

Если $(C, \triangle, \varepsilon)$ --- коалгебра, то алгебра $(C^*, \triangle^*, \varepsilon^*)$, \textit{двойственная} к $(C, \triangle, \varepsilon)$, 
определяется при помощи ограничения отображения $\triangle^*:\left(C \otimes C)^* \to C^*$, двойственного к $\triangle$, на подпространство
$C^* \otimes C^* \subseteq \left( C \otimes C\right)^*$ и композиции отображения $\varepsilon^*: C^* \to \fld^*$ с естественным отождествлением $\fld^* \cong \fld$. 
(Для удобства отображения $\triangle^*$ и $\varepsilon^*$, изменённые таким образом, обозначаются по-прежнему через $\triangle^*$ и $\varepsilon^*$.)

Если $(A, \mu, u)$ --- конечномерная ассоциативная алгебра с единицей, то, 
используя двойственные отображения $\mu^*: A^* \to \left(A \otimes A\right)^*$ и $u^*: A^* \to \mathbbm{k}^*$ и 
естественные отождествления $\left(A \otimes A\right)^* \cong A^* \otimes A^*$ и $\mathbbm{k}^* \cong \mathbbm{k}$, 
можно определить коалгебру $(A^*, \mu^*, u^*)$ \textit{двойственную} к $(A, \mu, u)$. 
Если алгебра $A$ бесконечномерна, то определить двойственную коалгебру таким образом уже не получается. 
В этом случае рассматривают подмножество $A^\circ \subseteq A^*$ таких линейных функций на алгебре $A$, которые содержат в своём ядре некоторый идеал конечной коразмерности. 
Тогда $\mu^*\left(A^\circ\right) \subseteq A^\circ \otimes A^\circ$. Обозначив через $\mu^\circ$ и $u^\circ$, 
соответственно, ограничения отображений $\mu^*$ и $u^*$ на $A^\circ$, получаем коалгебру $(A^\circ, \mu^\circ, u^\circ)$, \textit{конечную двойственную} к $(A, \mu, u)$.

В дальнейшем коалгебры $(C, \triangle, \varepsilon)$ для краткости обозначаются просто через $C$. Аналогичное соглашение действует и для остальных алгебраических структур.

\subsection{Алгебры Хопфа}

\begin{definition} \label{Hopf algebra}
  Набор $(H, \mu, u, \triangle, \varepsilon, S)$ называется \textit{алгеброй Хопфа} над полем $\mathbbm{k}$, если $(H, \mu, u)$ --- ассоциативная алгебра с единицей, 
  $(H, \triangle, \varepsilon)$ --- коассоциативная коалгебра с коединицей, 
  кроме того, $\triangle$ и $\varepsilon$ являются гомоморфизмами алгебр с единицей, а $S: H \to H$ --- такое линейное отображение, что \[
	\left(Sh_{(1)}\right)h_{(2)} = h_{(1)}\left(Sh_{(2)}\right) = \varepsilon(h)1_H \text{ для всех } h \in H.
  \]
\end{definition}

Отображение $S$ называется \textit{антиподом} алгебры $H$.

\begin{example} \label{field}
  Основное поле $\fld$ с тривиальным коумножением $\triangle$, коединицей $\varepsilon$ и антиподом $S$, заданными равенствами $\triangle(1) = 1 \otimes 1$,
  $\varepsilon(1) = 1$ и $S(1) = 1$ является алгеброй Хопфа.
\end{example}

\begin{example} \label{kG}
  Пусть $G$ --- группа, а $\fld$ --- поле. Рассмотрим алгебру $\kG$ с базисом, состоящим из элементов группы $G$ и умножением, продолженным по линейности с умножения в $G$. 
  Тогда $\kG$ называется \textit{групповой алгеброй} группы $G$. 
  Определим отображения $\triangle: \kG \to \kG \otimes \kG$, $\varepsilon: \kG \to \fld$ и $S: \kG \to \kG$ при помощи 
  формул $\triangle(g) = g \otimes g$, $\varepsilon(g) = 1$, $S(g) = g^{-1}$ при $g \in G$, а затем продолжим их по линейности на всё $\kG$. Тогда $\kG$ является алгеброй Хопфа.\end{example}

\begin{example} \label{H_m^2}
  Пусть $m \geq 2$ --- натуральное число, а $\zeta$ --- примитивный корень $m$-й степени из единицы в поле $\fld$. 
  (Такой корень может существовать в поле $\fld$, только если $\chr \fld \nmid m$.)
  Рассмотрим алгебру $H_{m^2}(\zeta)$ с $1$, порождённую элементами $c$ и $v$, которые удовлетворяют соотношениям $c^m = 1,\ v^m = 0,\ vc = {\zeta}cv$.
  Тогда элементы $\left(c^iv^k\right)_{0 \leq i,k\leq m-1}$ образуют базис алгебры $H_{m^2}(\zeta)$.
  Введём на алгебре $H_{m^2}(\zeta)$ структуру коалгебры при помощи равенств $\triangle(c) = c \otimes c,\ \triangle(v) = c \otimes v + v \otimes 1$,
  $\varepsilon(c) = 1,\ \varepsilon(v) = 0$. Тогда $H_{m^2}(\zeta)$ --- алгебра Хопфа с антиподом $S$, где $S(c) = c^{-1}$ и $S(v) = -c^{-1}v$. Алгебра Хопфа $H_{m^2}(\zeta)$
  называется \textit{алгеброй Тафта}. Алгебра $H_4(-1)$ называется \textit{алгеброй Свидлера}. 
  Так как $(-1)$ --- единственный примитивный корень $2$-й степени из еденицы, мы будем обозначать алгебры Свидлера просто через $H_4$. 
\end{example}

\begin{definition} \label{Hopf group}
  Элемент $h \neq 0$ алгебры Хопфа $H$ называется \textit{группоподобным}, если $\triangle h = h \otimes h$. Группоподобные элементы алгебры Хопфа образуют группу $G(H)$ 
  относительно операции умножения.
\end{definition}


Если $H$ --- алгебра Хопфа, то можно рассмотреть коалгебру $H^\circ$, конечную двойственную к алгебре $H$. Как множество $H^\circ$ является подалгеброй в алгебре $H*$. 
$H^\circ$ также является алгеброй Хопфа и называется \textit{конечной двойственной} алгеброй Хопфа к алгебре Хопфа $H$. Если $H$ конечномерна, то $H^\circ = H^*$, и 
$H^*$ называется, соответственно, алгеброй Хопфа, \textit{двойственной} к $H$.

\subsection{Модули}

Подобно тому, как определение ассоциативной алгебры с единицей было дано на языке линейных отображений, дадим определение модуля над алгеброй.

\begin{definition} \label{module}
  Пусть $M$ --- вектороной пространство, $(A, \mu)$ --- ассоциативная алгебра, а $\psi: A \otimes M \to M$ --- линейное отображение.
  Говорят, что $(M, \psi)$ --- \textit{(левый) модуль} над алгеброй $A$, если коммутативна следующая диаграмма:
  \[\xymatrix{
	  A \otimes A \otimes M \ar[rr]^{\id_A \otimes \psi} \ar[d]_{\mu \otimes \id_M}&& A \otimes M \ar[d]_\psi\\
	  A \otimes M \ar[rr]^\psi && M
	}\]
\end{definiton}

Действие алгебры $A$ на $M$ обозначается через $am:= \psi(a \otimes m)$ для всех $a \in A,\ m \in M$.

\begin{definition} \label{unit module}
  Пусть $(A, \mu, u)$ --- ассоциативная алгебра с единицей над полем $\fld$.
  Говорят, что $(M, \psi)$ --- \textit{унитальный модуль} над $(A, \mu, u)$, если $(M,\psi)$ --- модуль над $(A, \mu)$ и, кроме того, коммутативна диаграмма:
  \[\xymatrix{
	  \fld \otimes M \ar[rd]^\sim \ar[rr]^{u \otimes \id_M} && A \otimes M \ar[ld]^\mu\\
	  & M
   }\]
\end{definition}

\noindent(Здесь мы использовали естественное отждествление $\fld \otimes M \cong M$.)

В дальнейшем под модулями над ассоциативными алгебрами с единицей будут пониматься именно унитальные модули.

\subsection{Модульные алгебры}

Наличие в алгебре Хопфа комуножения и коединицы позволяет формулировать дополнительные условия на действие такой алгебры Хопфа на некоторой алгебре:

\begin{definition} \label{H-module algebra}
  Говорят, что (необязательно ассоциативная) алгебра $A$ над полем $\fld$ являетс \textit{(левой) $H$-модульной алгеброй} для алгебры Хопфа $H$, 
  если $A$ --- это левый $H$-модуль и 
  \begin{equation}
	h(ab) = \left(h_{(1)}a\right)\left(h_{(2)}b\right) \text{ для всех } a,b \in A,\ h \in H
  \end{equation}
\end{definition}

Обозначим через $\psi$ линейное отображение $H \otimes A \to A$, заданное равенством $\psi(h \otimes a) = ha$ для всех $h \in H$ и $a \in A$. 
Отображение $\psi$ называется \textit{$H$-модульной структурой} или \textit{$H$-действием} на $A$.

Мы будем говорить, что $A$ --- \textit{$H$-модульная алгебра с единицей}, если $A$ --- алгебра с единицей $1_A$ и $h1_A = \varepsilon(h)1_A$ для всех $h \in H$.

\begin{example}
  Если $G$ --- группа, то $\kG$-модульная алгебра --- это в точности алгебра, на которой группа $G$ действует автоморфизмами.
\end{example}

\begin{example}
  Если $H_{m^2}(\zeta)$ --- алгебра Тафта, то $H_{m^2}(\zeta)$-модульная алгебра --- это в точности алгебра на которой $c$ действует автоморфизмом степени $m$ и 
  $v$ действует нильпотентным косым дифференцированием.
\end{example}

\subsection{Коноситель}
Следующие понятия были введены в статье \cite{ASGordienko21ALAgoreJVercruysse}, написанной совместно А.С.Гордиенко, Аной Агоре и Йоостом Веркрёйсе.

Пусть $\psi: P \otimes A \to B$ --- линейное отображение для некоторых векторных пространств $P, A, B$ над полем $\fld$.
Назовём подпространство $\cosupp \psi := \psi\left(P \otimes (\text{---})\right)$ \textit{коносителем} отображения $\psi$.
Здесь $\psi(p \otimes (\text{---}))$ рассматривается как линейное отображение $A \to B$ для фиксированного элемента $p \in P$, 
а $\psi\left(P \otimes (\text{---})\right):= \left\{\psi(p \otimes(\text{---})) \mid p \in P\right\}$.
Другими словами, коноситель $\spi$ --- это подпространство всех линейных операторов $A \to B$, отвечающих отображению $\psi$.

\begin{definition} \label{equal module}
  Пусть $\psi_i:P_i \otimes A_i \to B_i$ --- линейные отображения для некоторых векторных пространств $P_i, A_i, B_i$, где $i =1,2$, и пусть
  $\varphi: A_1\  \widetilde{\to}\ A_2$ и $\xi: B_1\  \widetilde{\to}\ B_2$ --- линейные биективные отображения. Будем говорить, что пара $(\varphi, \xi)$
  является \textit{эквивалентностью} отображений $\psi_1$ и $\psi_2$, если
  \[
	\xi(\cosupp \psi_1)\varphi^{-1} = \cosupp \psi_2
  \]
  В этом случае будем говорить, что $\psi_1$ и $\psi_2$ \textit{эквивалентны} при помощи $(\varphi, \xi)$.
\end{definition}

Рассмотрим теперь линейные отображения $\psi_i: P_i \otimes A \to B,\ i = 1,2$, где пространства $A$ и $B$ общие для обоих отображений.
Тогда отображения $\psi_1, \psi_2$ эквивалентны при помощи $(\id_A, \id_A)$, если и только если $\cosupp \psi_1 = \cosupp \psi_2$.

\subsection{Структура коносителя}

Заметим, что для $H$-действия $\psi: H \otimes A \to A$, где $H$ --- алгебра Хопфа, его коноситель $\cosupp \psi$ --- это в точности образ соответствующего гоморфизма алгебр
$H \to \End_{\fld}(A)$. В частности $\cosupp \psi$ --- это подалгебра в $\End_{\fld}(A)$, содержащая тождественный оператор.

В дальнейшем в работе $H$-модульные алгебры с единицей, где $H$ --- алгебра Хопфа, будут подразумеваться конечномерными.

\begin{definition} \label{struct cosupp}
  Пусть $A$ --- $n$-мерная $H$-модульная алгебра, где $H$ --- алгебра Хопфа, тогда выберем в $A$ базис и 
  отождествим $\End_{\fld}(A)$ с алгеброй матриц $\mathrm{M}_n(\fld)$. Определим гоморфизм алгебр $\zeta: H \to \mathrm{M}_n(\fld)$ из равенства 
  $\zeta(h)(a) = \psi(h \otimes a)$ для любых $h \in H,\ a \in A$, где $\psi$ --- $H$-модульная структура на $A$.
  Тогда $\zeta(h)$ имеет вид $\left(\alpha_{ij}(h)\right)_{1 \leq i, j \leq n}$.
  Система линейных функций $\left(\alpha_{ij}(h)\right)_{1 \leq i, j \leq n}$ называется \textit{структурой коносителя}.
\end{definition}

\begin{remark} \label{struct cosupp with 1}
  Если $A$ --- $H$-модульная алгебра с единицей, тогда $\psi(h \otimes 1_A) = \varepsilon(h)1_A$, для всех $h \in H$.
  Тогда для $\zeta(h)(1_A) = \varepsilon(h)1_A$ для всех $h \in H$ или, что то же самое, следующее условие на структуру коносителя
  \[
	\alpha_{11} = \varepsilon,\ \alpha_{21} = \alpha_{31} = \dots = \alpha_{n1} = 0
  \]
\end{remark}

Пусть $A$ --- $n$-мерная $H$-модульная алгебра, где $H$ --- алгебра Хопфа. Выберем базис в $A$ и отождествим $\End_{\fld}(A)$ с алгеброй матриц $\mathrm{M}_n(\fld)$.
Определим гоморфизм алгебр $\zeta: H \to \mathrm{M}_n(\fld)$ из равенства 
$\zeta(h)(a) = \psi(h \otimes a)$ для любых $h \in H,\ a \in A$, где $\psi$ --- $H$-модульная структура на $A$.
Так как $\zeta$ гомоморфизм алгебр, то $\Ker(\zeta)$ --- идеал. Известно, что алгебра $\zeta(H)$ изоморфна алгебре $H/\Ker(\zeta)$. 
Значит $\Ker(\zeta)$ --- идеал конечной коразмерности. 

Для любой линейной функции $\alpha_{ij}$ из структуры коносителя верно, что $\Ker(\zeta) \subseteq \Ker(\alpha_{ij})$, а значит любая функция из структуры коносителя
содержит в своём ядре идеал конечной коразмерности. Тогда все линейные функции из структуры коносителя лежат в конечной двойственной алгебре Хопфа $H^\circ$.

Из того, что $\zeta$ --- гоморфизм алгебр, следует соотношение, что для любых $a,\ b \in H$ и $1 \leq i,j \leq n$ верно:
\begin{equation} \label{product}
  \alpha_{ij}(ab) = \sum\limits_{k=1}^{n}\alpha_{ik}(a)\alpha_{kj}(b) 
\end{equation}
Тогда для любых двух функций  $\alpha,\ \beta$ из структуры коносителя и $a,\ b \in H$ верно
\begin{eqnarray}
    \varepsilon\alpha = \alpha\varepsilon = \alpha \label{str1}\\
    (\alpha\beta)(a) = \alpha(a_{(1)})\beta(a_{(2)}) \label{str2}\\
    u^\circ(\alpha) = \alpha(1_H) \label{str3}\\
    \bigtriangleup\alpha(a \otimes b) = \alpha(ab) \label{str4}
\end{eqnarray}
Определим линейную функцию $\beta_{ij}: H \otimes H \to \fld$ с помощью формулы $\beta_{ij} = \sum\limits_{k=1}^{n}\alpha_{ik} \otimes \alpha_{kj}$
Тогда для любых $a,\ b \in H$, с учётом верно 
\[
  \beta_{ij}(a \otimes b) = \sum\limits_{k=1}^{n}\alpha_{ik}(a)\alpha_{kj}(b) = \alpha_{ij}(ab) = \triangle\alpha_{ij}(a \otimes b)
\]
Значит $\beta_{ij} = \triangle\alpha_{ij}$, иначе говоря
\begin{equation}\label{coproduct}
  \triangle\alpha_{ij} = \sum\limits_{k=1}^{n}\alpha_{ik} \otimes \alpha_{kj}
\end{equation}

Пусть $\alpha_{i_1j_1}(h),\ \dots,\ \alpha_{i_kj_k}(h)$ --- линейно независимая система функций из структуры коносителя.
Обозначим как $M$ пространство порождённое матричными единицами $E_{ij}$ такими, что для любого $1 \leq l \leq k$ верно $E_{ij} \neq E_{i_lj_l}$.
Определим отображение $\hat \zeta: H \to \mathrm{M}_n(\fld)/M$ из равенства $\zeta = \hat \zeta \circ \pi$, 
где $\pi: \mathrm{M}_n(\fld) \to \mathrm{M}_n(\fld)/M$ --- канонический гоморфизм векторных пространств.
Тогда $\hat \zeta(h) = \sum\limits_{t =1}^k\alpha_{i_tj_t}E_{i_tj_t}$. Рассмотрим ранг матрицы линейного отбражения $\hat \zeta$ и получим
$\dim \hat \zeta(H) = \dim \langle \alpha_{i_1j_1},\ \dots,\ \alpha_{i_kj_k} \rangle = k$, в силу линейной независимости.
С учётом $\dim \mathrm{M}_n(\fld) = k$, получаем что для любого $1 \leq t \leq k$ существует $h \in H$ такой, что $\hat \zeta(h) = E_{i_tj_t}$.
Тем самым мы доказали, что для любого $1 \leq t \leq k$ существует $h \in H$ такой, что
\[
  \alpha_{i_lj_l}(h) = \delta^t_l
\]
Такой элемент $h \in H$ для функции $\alpha_{ij}$ будем обозначать $\left(\alpha_{ij}\right)_\varepsilon$ и называть \textit{единицей линейной функции}.

Выберем некоторые $i$ и $j$ и пусть все функции участвующие в (\ref{product}) для $\alpha_{ij}$ и $\alpha_{kl},\ \alpha_{tm}$ линейно независимые, тогда
\begin{equation}\label{podstanovka}
  \alpha_{ij}\left((\alpha_{kl})_\varepsilon(\alpha_{tm})_\varepsilon\right) = \delta^i_k\delta^j_m\delta^l_t
\end{equation}

Докажем полезную лемму 
\begin{lemma} \label{Group_lemma}
  Пусть задана алгебра Хопфа $H$ и группоподобный элемент $h \in H$. Если $h$ имеет конечный порядок как элемент группы, тогда его степени $1,\ h,\ \dots,\ h^{k-1}$ линейно независимы, где $k$ --- порядок элемента $h$. Если же $h$ имеет бесконечный порядок, то любая конечная систмеа из степеней $h$ линейно независима. \label{lem}
\end{lemma}

\begin{proof}
  Доказывать будем по индукции.
  База индукции, что система $1$ линейна независима, очевидна.
  Пусть $1,\ h,\ \dots,\ h^n$ линейно независимы. Предположим, что существуют $\lambda_i$ такие, что $h^{n+1} = \lambda_11+\lambda_2h + \dots + \lambda_{n+1}h^n$. Тогда получим, что  \[
  \bigtriangleup h^{n+1} = \bigtriangleup (\lambda_11+\lambda_2h + \dots + \lambda_{n+1}h^n) = \lambda_1 1 \otimes 1 +\lambda_2 h \otimes h + \dots + \lambda_{n+1} h^n \otimes h^n = h^{n+1} \otimes h^{n+1} =  (\lambda_11+\lambda_2h + \dots + \lambda_{n+1}h^n) \otimes (\lambda_11+\lambda_2h + \dots + \lambda_{n+1}h^n)
  \]
А значит для всех $i$ и $j$ верно, что $\lambda_i\lambda_j = \lambda_i \delta^i_j$, где $\delta^i_j$ --- символ Кронекера. Значит для любого $i$ $\lambda_i$ либо $0$, либо $1$. 
При этом если существуют $i_1,\ i_2$ такие, что $i_1 \neq i_2$ и $\lambda_{i_1} = \lambda_{i_2} = 1$, то $\lambda_{i_1}\lambda_{i_2} \neq 0$. Значит не более одного $\lambda_i$ может быть ненулевым. Так как \[
  \varepsilon(h^{n+1}) = 1 = \varepsilon(\lambda_11+\lambda_2h + \dots + \lambda_{n+1}h^n) = \lambda_1+\lambda_2 + \dots + \lambda_{n+1}
\]
Значит в точности один коэффицент равен 1. Пусть это $\lambda_{t} = 1$. Тогда $h^{n+1} = h^t$, иначе говоря $h^{n+1-t} = 1$. Если $h$ имеет бесконечный порядок получим противоречие. Если же $h$ имеет конечный порядок, то если $n+1 \leq \mathrm{ord}(h)$, тогда этого неверно, а значит они все $h^k$ вплоть до $h^{\mathrm{ord}-1}$ линейно независимы.

\end{proof}

\newpage

\section{Структура некоторых \texorpdfstring{$H$}{H}-модульных алгебр}

Пусть задана алгебра $A = \fld[x]/(x^n)$, где поле $\fld$, такое что $\chr \fld \nmid n$ и $A$ является $H$-модульной алгеброй с единицей, где $H$ --- алгебра Хопфа.

Пусть $\psi$ --- $H$-модульная структура. Выберем базис $1,\ \bar x,\ \dots,\ \bar x^{n-1}$, тогда можно определить $\zeta: H \to \mathrm{M}_n(\fld)$ из равенства
$\zeta(h)(a) = \psi(h \otimes a)$. 
Пусть коноситель $\cosupp \psi$ является подалгеброй верхнетреугольных матриц. Тогда для любого $h \in H$ $\zeta(h)$ верхнетреугольная матрица.
Тогда получаем следующие соотношения на структуру коносителя: для всех $i > j$ верно $\alpha_{ij} = 0$.

Введём обозначение $\mathrm{P}^i_j(\alpha,\ \beta)$ --- сумма всех возможных произведений, таких что $\alpha$  входит $i$ раз, а $\beta$ --- $j$ раз, где $\alpha,\ \beta \in H^\circ$ и положим $\mathrm{P}^0_0(\alpha,\ \beta) = \varepsilon$.

\begin{remark} \label{reqursive}
  Из этого определения несложно понять, что
  \[
	\mathrm{P}^i_j(\alpha,\ \beta) = \alpha\mathrm{P}^{i-1}_j(\alpha,\ \beta)+\beta\mathrm{P}^i_{j-1}(\alpha,\ \beta)
  \]
\end{remark}

Переобозначим линейные функции из матрицы модульной структуры $\alpha_{12}$ и $\alpha_{22}$ как $\alpha$ и $\beta$ соответственно.
Тогда имеет место следующее утверждение.

\begin{proposition}
  Матрица модульной структуры имеет вид 
  \begin{equation}
	\zeta(h) = 
	\begin{pmatrix}
	  \mathrm{P}^0_0(\alpha,\ \beta) && \mathrm{P}^1_0(\alpha,\ \beta) && \mathrm{P}^2_0(\alpha,\ \beta) && \dots  && \mathrm{P}^{n-1}_0(\alpha,\ \beta) \\
	  0 && \mathrm{P}^0_1(\alpha,\ \beta) && \mathrm{P}^1_1(\alpha,\ \beta) && \dots  && \mathrm{P}^{n-2}_1(\alpha,\ \beta) \\
	  \vdots && 0 && \mathrm{P}^0_2(\alpha,\ \beta) && \dots  && \mathrm{P}^{n-3}_2(\alpha,\ \beta)\\
	  \\
	  \vdots &&  \vdots && \ddots && \ddots && \vdots \\
	  \\
	  0 && 0 && \dots && 0 && \mathrm{P}^0_{n-1}(\alpha,\ \beta) \\ \label{matrixPol}
	\end{pmatrix}
  \end{equation}
\end{proposition}

\begin{proof}
  Доказывать будем индукцией по номеру столбца. 
  База очевидна, ведь по определению $\mathrm{P}^0_0(\alpha,\ \beta) = \varepsilon$, $\mathrm{P}^1_0(\alpha,\ \beta) = \alpha$ и $\mathrm{P}^0_1(\alpha,\ \beta) = \beta$.
  Пусть для столбца с номером $k$ это верно, значит
  \[
	\zeta(h)(\bar x^k) = \sum\limits_{t = 0}^k \mathrm{P}^{k-t}_t(\alpha,\ \beta)(h)(\bar x^t) 
  \]
  Тогда так как $A$ --- $H$-модульная алгебра с $1$, значит
  \[
	\zeta(h)(\bar x^{k+1}) = \left(\zeta(h_{(1)})(\bar x)\right) \left(\zeta(h_{(2)})(\bar x^k)\right)
  \]
  Приравняем коэффиценты при соответствующих степенях $\bar x$. Тогда для всех $i$ от $0$ до $k+1$ 
  \[
	\alpha_{ik+1}(h) = \alpha(h_{(1)})\mathrm{P}^{k-i}_i(\alpha,\ \beta)(h_{(2)})+\beta(h_{(2)})\mathrm{P}^{k+1-i}_{i-1}(\alpha,\ \beta)(h_{(1)})
  \]
  \[
	\alpha_{ik+1} = \alpha\mathrm{P}^{k-i}_i(\alpha,\ \beta)+\beta\mathrm{P}^{k+1-i}_{i-1}(\alpha,\ \beta)
  \]
  Из сделанного замечания следует, что $\alpha_{ik+1} = \mathrm{P}^{k+1-i}_{i}(\alpha,\ \beta)$
  Тем самым шаг индукции доказан.
\end{proof}

Тогда соотношение (\ref{podstanovka}) преобразуется в 
\begin{equation} \label{pods*}
  \mathrm{P}^i_j(\alpha,\ \beta)\left(\left(\mathrm{P}^k_l(\alpha,\ \beta)\right)_\varepsilon\left(\mathrm{P}^t_m(\alpha,\ \beta)\right)_\varepsilon\right) = 
  \delta^j_l\delta^{i+j}_{t+m}\delta^{k+l}_m
  %\delta_{j+1 l+1}\delta_{i+j+1 t+m+1}\delta_{k+l+1 m+1}
\end{equation}

С учётом $\bar x^n = 0$ получаем соотношения
\[
  \zeta(h)(\bar x^n) =  \left(\zeta(h_{(1)})(\bar x)\right) \left(\zeta(h_{(2)})(\bar x^{n-1})\right) = 0
\]
Приравняем все коэффиценты при степенях $\bar x$ к нулю. Тогда для всех $i$ от $0$ до $n-1$ 
\[
  \alpha(h_{(1)})\mathrm{P}^{n-1-i}_i(\alpha,\ \beta)(h_{(2)})+\beta(h_{(2)})\mathrm{P}^{n-i}_{i-1}(\alpha,\ \beta)(h_{(1)}) = 0
\]
\[
  \alpha\mathrm{P}^{n-1-i}_i(\alpha,\ \beta)+\beta\mathrm{P}^{n-i}_{i-1}(\alpha,\ \beta) = 0
\]
Из замечания (\ref{reqursive}) следует, что для всех $i$ от $0$ до $n-1$ 
\begin{equation} \label{null}
  \mathrm{P}^{n-i}_{i}(\alpha,\ \beta) = 0
\end{equation}

Из (\ref{coproduct}) следует, что $\varepsilon,\ \beta,\ \dots,\ \beta^{n-1}$ группоподобные, а значит для них верна лемма (\ref{Group_lemma}).

Обозначим систему $\mathrm{P}^t_p(\alpha,\ \beta),\ \dots,\ \mathrm{P}^t_{p+l}(\alpha,\ \beta)$ как $\mathrm{D}^t_p(l)$.
В свою очередь систему $\bigcup\limits_{i=0}^t D^{t+i}_p(l-i)$ обозначим как 
$M^t_p(l)$

%\begin{theorem}\label{theorem independent}
%  Пусть $k$ $P^i_j(\alpha,\ \beta)$
%\end{theorem}

Теперь мы можем написать основную теорему, которую мы позже докажем.

\begin{theorem}\label{main}
  Пусть функции $\mathrm{K}^0_0(n-1) \cup \{\alpha\}$ линейно независимы. 
  Тогда $\zeta(h)$ блочно диагональная где на диагонали стоят 
  $\beta^{(q-1)(k+1)}M^0_{q(k+1)}(k+1)$.
\end{theorem}

Перед тем, как доказать эту теорему, выведем две необходимые леммы.
\begin{lemma} \label{support1}
  Пусть система $\bigcup\limits_{i=0}^k\mathrm{D}^i_0(l-i)$ линейно независима. 
  Тогда $\mathrm{P}^{k+1}_0(\alpha,\ \beta)$, где $k >0$, 
  линейно независима с $\bigcup\limits_{i=0}^k\mathrm{D}^i_0(l-i)$.
\end{lemma}

\begin{proof}
  Предположим противное, тогда существуют $\lambda^i_j$ такие, что 
  \[
	\mathrm{P}^{k+1}_0(\alpha,\ \beta) = \lambda^i_j\mathrm{P}^{i}_j(\alpha,\ \beta)
  \]
 где $\mathrm{P}^{i}_j(\alpha,\ \beta)\in\bigcup\limits_{i=0}^k\mathrm{D}^i_0(l-i)$.

  Пусть $m \neq 0$ и $\mathrm{P}^{m}_0(\alpha,\ \beta) \in \bigcup\limits_{i=0}^k\mathrm{D}^i_0(l-i)$.
  Тогда \[
	\mathrm{P}^{i}_j(\alpha,\ \beta)\left(\left(\mathrm{P}^m_0(\alpha,\ \beta)\right)_\varepsilon\left(\mathrm{P}^0_m(\alpha,\ \beta)\right)_\varepsilon\right)  = 
	\delta^j_0\delta^i_m
  \]
  Значит  \[
	\mathrm{P}^{k+1}_0(\alpha,\ \beta)\left(\left(\mathrm{P}^m_0(\alpha,\ \beta)\right)_\varepsilon\left(\mathrm{P}^0_m(\alpha,\ \beta)\right)_\varepsilon\right) = \lambda^m_0 = 0
  \]
  Теперь пусть $l \neq 0$, $l$ и $m$ не равны одновременно $k+1$ и $0$ соответственно и $\mathrm{P}^{m}_l(\alpha,\ \beta) \in \bigcup\limits_{i=0}^k\mathrm{D}^i_0(l-i)$.
  Тогда \[
	\mathrm{P}^{i}_j(\alpha,\ \beta)\left(\left(\mathrm{P}^0_l(\alpha,\ \beta)\right)_\varepsilon\left(\mathrm{P}^m_l(\alpha,\ \beta)\right)_\varepsilon\right)  = \delta^j_l\delta^i_m
  \]
  Значит  \[
	\mathrm{P}^{k+1}_0(\alpha,\ \beta)\left(\left(\mathrm{P}^0_l(\alpha,\ \beta)\right)_\varepsilon\left(\mathrm{P}^m_l(\alpha,\ \beta)\right)_\varepsilon\right) = \lambda^m_l = 0
  \]
  Отсюда следует, что 
  \[
	\mathrm{P}^{k+1}_0(\alpha,\ \beta) = \lambda^0_0\mathrm{P}^0_0(\alpha,\ \beta) + \lambda^0_{k+1}\mathrm{P}^0_{k+1}(\alpha,\ \beta)
  \]
  Далее рассмотрим 
  \[
	\mathrm{P}^{k+1}_0(\alpha,\ \beta)\left(\left(\mathrm{P}^1_0(\alpha,\ \beta)\right)_\varepsilon\left(\mathrm{P}^k_1(\alpha,\ \beta)\right)_\varepsilon\right) = 1
  \]
  Однако 
  \begin{eqnarray*}
	\mathrm{P}^0_0(\alpha,\ \beta)\left(\left(\mathrm{P}^1_0(\alpha,\ \beta)\right)_\varepsilon\left(\mathrm{P}^k_1(\alpha,\ \beta)\right)_\varepsilon\right) = 0\\
	\mathrm{P}^0_{k+1}(\alpha,\ \beta)\left(\left(\mathrm{P}^1_0(\alpha,\ \beta)\right)_\varepsilon\left(\mathrm{P}^k_1(\alpha,\ \beta)\right)_\varepsilon\right) = 0\\
  \end{eqnarray*}
  Полученное противоречие и доказывает нашу лемму.
\end{proof}

\begin{lemma} \label{support2}
  Пусть система $\bigcup\limits_{i=0}^k\mathrm{D}^i_0(l-i)$ 
  объединенная с системой $D^{k+1}_0(p)$ линейно независима и $p+2$ не делит $n$. 
  Тогда $\mathrm{P}^{k+1}_{p+1}(\alpha,\ \beta)$, 
  где $k>0$ и $p+1< l-1$ линейно независима с $\bigcup\limits_{i=0}^k\mathrm{D}^i_0(l-i)$, 
  объединенной с системой $\mathrm{P}^{k+1}_0,\ \dots,\ \mathrm{P}^{k+1}_p$.
\end{lemma}

\begin{proof}
  Предположим противное, тогда существуют $\lambda^i_j$ такие, что 
  \[
	\mathrm{P}^{k+1}_{p+1}(\alpha,\ \beta) = \lambda^i_j\mathrm{P}^i_j(\alpha,\ \beta)
  \]
  где $\mathrm{P}^{i}_j(\alpha,\ \beta)$ принадлежат 
  или $\bigcup\limits_{i=0}^k\mathrm{D}^i_0(l-i)$, или  $D^{k+1}_0(p)$ .

  Пусть $m \neq 0$ и $\mathrm{P}^m_{p+1}(\alpha,\ \beta)$ принадлежат 
  или $\bigcup\limits_{i=0}^k\mathrm{D}^i_0(l-i)$, или  $D^{k+1}_0(p)$ .
  Тогда \[
	\mathrm{P}^{i}_j(\alpha,\ \beta)\left(
	\left(\mathrm{P}^m_{p+1}(\alpha,\ \beta)\right)_\varepsilon
	\left(\mathrm{P}^0_{m+p+1}(\alpha,\ \beta)\right)_\varepsilon\right)  = 
	\delta^j_{p+1}\delta^i_m
  \]
  Значит  \[
  \mathrm{P}^{k+1}_{p+1}(\alpha,\ \beta)\left(
  \left(\mathrm{P}^m_0(\alpha,\ \beta)\right)_\varepsilon
  \left(\mathrm{P}^0_m(\alpha,\ \beta)\right)_\varepsilon\right) = 
  \lambda^m_{p+1} = 0
  \]
  Теперь пусть $l \neq p+1$, $l$ и $m$ не равны одновременно 
  $p+k+2$ и $0$ соответственно и 
  $\mathrm{P}^{m}_l(\alpha,\ \beta)$ принадлежат 
  или $\bigcup\limits_{i=0}^k\mathrm{D}^i_0(l-i)$, или  $D^{k+1}_0(p)$ .

  Тогда \[
	\mathrm{P}^{i}_j(\alpha,\ \beta)\left(\left(\mathrm{P}^0_l(\alpha,\ \beta)\right)_\varepsilon\left(\mathrm{P}^m_l(\alpha,\ \beta)\right)_\varepsilon\right)  = \delta^j_l\delta^i_m
  \]
  Значит  \[
	\mathrm{P}^{k+1}_{p+1}(\alpha,\ \beta)\left(\left(\mathrm{P}^0_l(\alpha,\ \beta)\right)_\varepsilon\left(\mathrm{P}^m_l(\alpha,\ \beta)\right)_\varepsilon\right) = \lambda^m_l = 0
  \]
  Отсюда следует, что 
  \[
	\mathrm{P}^{k+1}_{p+1}(\alpha,\ \beta) = \lambda^0_{p+1}\mathrm{P}^0_{p+1}(\alpha,\ \beta) + \lambda^0_{p+k+2}\mathrm{P}^0_{p+k+2}(\alpha,\ \beta)
  \]
  Если $k > 0$, то рассмотрим 
  \[
	\mathrm{P}^{k+1}_{p+1}(\alpha,\ \beta)\left(\left(\mathrm{P}^1_0(\alpha,\ \beta)\right)_\varepsilon\left(\mathrm{P}^k_1(\alpha,\ \beta)\right)_\varepsilon\right) = 1
  \]
  Однако 
  \begin{eqnarray*}
	\mathrm{P}^0_0(\alpha,\ \beta)\left(\left(\mathrm{P}^1_0(\alpha,\ \beta)\right)_\varepsilon\left(\mathrm{P}^k_1(\alpha,\ \beta)\right)_\varepsilon\right) = 0\\
	\mathrm{P}^0_{k+1}(\alpha,\ \beta)\left(\left(\mathrm{P}^1_0(\alpha,\ \beta)\right)_\varepsilon\left(\mathrm{P}^k_1(\alpha,\ \beta)\right)_\varepsilon\right) = 0\\
  \end{eqnarray*}
  Получили противоречие и доказали.

  Если $k = 0$, тогда скажем, что так как $\mathrm{P}^0_t(\alpha,\ \beta)$ это степень $\beta$, то $\mathrm{P}^{1}_{p+1}(\alpha,\ \beta)$ коммутирует с $\beta$.
  
  Если $p+1 = n-2$, тогда 
  \[
	\mathrm{P}^1_{n-1}(\alpha,\ \beta) = \beta\mathrm{P}^1_{p+1}+\alpha\beta^{n-1} = 0
  \]
  Домножим на $\beta^{-p-2}$ и получим нетривиальную линейную комбинацию $\varepsilon,\ \beta,\ \alpha$, что противоречит условиям леммы, значит осталось доказать, для случая $p+1 < n-2$

  Для начала докажем формулу 
  \begin{equation}\label{sup1}
	\mathrm{P}^{1}_{p+2+l}(\alpha,\ \beta) = \mathrm{P}^1_l(\alpha,\ \beta)\beta^{p+2} + \mathrm{P}^1_{p+1}(\alpha,\ \beta)\beta^{l+1}
  \end{equation}
  При $l = 0$
  \begin{eqnarray*}
	\mathrm{P}^1_{p+2}(\alpha,\ \beta) &=& \beta\mathrm{P}^1_{p+1}(\alpha,\ \beta) + \alpha\beta^{p+2} = \\
	  &=& \mathrm{P}^1_0(\alpha,\ \beta)\beta^{p+2} + \mathrm{P}^1_{p+1}(\alpha,\ \beta)\beta
  \end{eqnarray*}

  База индукции доказана.
  Пусть верно при $l$, докажем, что верно при $l+1$
  \begin{eqnarray*}
	\mathrm{P}^{1}_{p+2+l+1}(\alpha,\ \beta)&=& \beta\mathrm{P}^1_{p+1+l}(\alpha,\ \beta) + \alpha\beta^{p+2+l+1} = \\
	&=& \beta(\mathrm{P}^1_l(\alpha,\ \beta)\beta^{p+2} + \mathrm{P}^1_{p+1}(\alpha,\ \beta)\beta^{l+1}) + \alpha\beta^{p+2+l+1} = \\
	&=& (\beta\mathrm{P}^1_l(\alpha,\ \beta) + \alpha\beta^{l+1})\beta^{p+2} + \beta\mathrm{P}^1_{p+1}(\alpha,\ \beta)\beta^{l+1} = \mathrm{P}^1_{l+1}(\alpha,\ \beta)\beta^{p+2} + \mathrm{P}^1_{p+1}(\alpha,\ \beta)\beta^{l+2}
  \end{eqnarray*}

  Тем самым мы доказали (\ref{sup1}).
  Тогда зная $\mathrm{P}^1_{n-1}(\alpha,\ \beta) = 0$, подставим в формулу $l = n-p-3$
  \[
  \mathrm{P}^{1}_{n-1}(\alpha,\ \beta) &=& \mathrm{P}^1_{n-2-(p+1)}(\alpha,\ \beta)\beta^{p+2} + \mathrm{P}^1_{p+1}(\alpha,\ \beta)\beta^{n-2 -p} = \\
  \mathrm{P}^1_{n-2-(p+1)}(\alpha,\ \beta)\beta^{p+2} + (\lambda^0_{p+1}\beta^{p+1}+ \lambda^0_{p+2}\beta^{p+2})\beta^{n-2 -p} = 0
  \]
  Тогда можно выразить 
  \[
	\mathrm{P}^1_{n-2-(p+1)}(\alpha,\ \beta) = -(\lambda^0_{p+1}\varepsilon+ \lambda^0_{p+2}\beta)\beta^{n-2-(p+1)}
  \]
  Благодаря этому и формуле (\ref{sup1}) мы можем получить ещё одну формулу
  \begin{equation}\label{sup2}
	\mathrm{P}^1_{n-1-q(p+2)}(\alpha,\ \beta) = -q(\lambda^0_{p+1}\varepsilon+ \lambda^0_{p+2}\beta)\beta^{n-1-q(p+2)}
  \end{equation}

  База индукции при $q = 1$ была доказана выше.
  Пусть теперь формула верна при $q$, докажем её истинность при $q+1$ пользуясь формулой (\ref{sup1}) при $l = n-1-(q+1)(p+2)$
  \begin{eqnarray*}
  \mathrm{P}^1_{n-1-q(p+2)}(\alpha,\ \beta) = \mathrm{P}^1_{n-1-(q+1)(p+2)}(\alpha,\ \beta)\beta^{p+2} + \mathrm{P}^1_{p+1}(\alpha,\ \beta)\beta^{n-(q+1)(p+2)}\\
	-q(\lambda^0_{p+1}\varepsilon+ \lambda^0_{p+2}\beta)\beta^{n-1-q(p+2)} = \mathrm{P}^1_{n-1-(q+1)(p+2)}(\alpha,\ \beta)\beta^{p+2} + \mathrm{P}^1_{p+1}(\alpha,\ \beta)\beta^{n-(q+1)(p+2)}\\
	-(q+1)(\lambda^0_{p+1}\varepsilon+ \lambda^0_{p+2}\beta)\beta^{n-1-(q+1)(p+2)} = \mathrm{P}^1_{n-1-(q+1)(p+2)}(\alpha,\ \beta)\beta^{p+2}
  \end{eqnarray*}
  Таким образом мы доказали, что формула (\ref{sup2}) верна.

  Повторно применяя формулу (\ref{sup2}) для всё больших $q$, найдётся момент, такой, что $n-1-(q+1)(p+2) < 0 \leq n-1-q(p+2)$. Обозначим такое значение $q$ как $q_0$.
  Тогда $0 \leq n-1-q_0(p+2) \leq p$. Действительно, первое неравенство следует из определения $q_0$.
  Второе неравенство верно, так как если $p+2 \leq n-1-q_0(p+2)$, то $0 \leq n-1-(q_0+1)(p+2)$, что противоречит выбору $q_0$. 
  Если же $n-1-q_0(p+2) = p+1$, тогда $n = (q_0+1)(p+2)$, но по условию $p+2$ не делит $n$, что невозможно.

  Тогда из неравенства следует, что $\mathrm{P}^1_{n-1-q_0(p+2)}(\alpha,\ \beta) \in K^0_0(l)$, но это противоречит линейной независимости в условии леммы.
\end{proof}

Из доказательства этой леммы следует, что $\mathrm{P}^1_{p+1}(\alpha,\ \beta) = 0$, 
если и $p+2$ делит $n$.

\begin{lemma}\label{support up}
  Пусть $\mathrm{P}^1_{p+1}(\alpha,\ \beta) = 0$
  Тогда $\mathrm{P}^t_{p+2-t}(\alpha,\ \beta) = 0$, 
\end{lemma}

\begin{proof}
  Из того, что $\mathrm{P}^1_{p+2}(\alpha,\ \beta) = 
  \beta\mathrm{P}^1_{p+1}(\alpha,\ \beta) + \alpha\beta^{p+2} = 
  \beta^{p+2}\alpha + \mathrm{P}^1_{p+1}(\alpha,\ \beta)\beta$

  Отсюда нетрудно получить, что $\alpha\beta^{p+2}=\beta^{p+2}\alpha$.
  Кроме того из формулы (\ref{sup1}), следует, что
  $\mathrm{P}^1_{p+2+l} = \beta^{p+2}\mathrm{P}^1_l$.

  Будем доказывать, что $\mathrm{P}^t_{p+2-t}(\alpha,\ \beta) = 0$ и 
  $\mathrm{P}^q_{p+2+l} = \beta^{p+2}\mathrm{P}^q_l$, по индукции.
  База индукции при $t=1$ верна по условию.
  Пусть $\mathrm{P}^q_{p+2-q}(\alpha,\ \beta) = 0$ и 
  $\mathrm{P}^q_{p+2+l} = \beta^{p+2}\mathrm{P}^q_l$, для всех $q$ от $1$ до $t$,
  тогда докажем, 
  \begin{equation}\label{sup1 up}
	\mathrm{P}^{t+1}_{p+1-t+l}(\alpha,\ \beta) = 
	  \beta^l\mathrm{P}^{t+1}_{p+1-t}(\alpha,\ \beta) + 
	  \beta^{p+2}\mathrm{P}^{t+1}_{l-1-t}(\alpha,\ \beta)
  \end{equation}
  Где положим $P^i_{(-t)} = 0$, для всех $t < 0$

  Докажем базу индукции.
  \begin{eqnarray*}
	\mathrm{P}^{t+1}_{p+2-t}(\alpha,\ \beta) =
	\alpha\mathrm{P}^t_{p+2-t}(\alpha,\ \beta) +
	\beta\mathrm{P}^{t+1}_{p+1-t}(\alpha,\ \beta) =\\
	= \beta\mathrm{P}^{t+1}_{p+1-t}(\alpha,\ \beta) +
	\alpha\beta^{p+2}\mathrm{P}^t_{-t}(\alpha,\ \beta)
  \end{eqnarray*}
  Что соответствует формуле (\ref{sup1 up}) при $l=1$.

  Теперь пусть верно при $l$ докажем при $l+1$.
  \begin{eqnarray*}
  \mathrm{P}^{t+1}_{p+1-t+l+1}(\alpha,\ \beta) = 
	\beta\mathrm{P}^{t+1}_{p-t+l+1}(\alpha,\ \beta) + 
	\alpha\mathrm{P}^t_{p+l-t+2}(\alpha,\ \beta) = \\
	 = \beta\left(\beta^l\mathrm{P}^{t+1}_{p+1-t}(\alpha,\ \beta) + 
	\beta^{p+2}\mathrm{P}^{t+1}_{l-1-t}(\alpha,\ \beta)\right) +
	\alpha\beta^{p+2}\mathrm{P}^t_{l-t}(\alpha,\ \beta) =\\
	= \beta^{l+1}\mathrm{P}^{t+1}_{p+1-t}(\alpha,\ \beta) + 
	\beta^{p+2}(\alpha\mathrm{P}^t_{l-t}(\alpha,\ \beta) + 
	\beta\mathrm{P}^{t+1}_{l-1-t}(\alpha,\ \beta))
  \end{eqnarray*}

  Тем самым формула (\ref{sup1 up}) доказана.
  С помощью неё докажем следующую формулу 
  \begin{equation}\label{sup2 up}
	-q\beta^{n-(q+1)(p+2)}\mathrm{P}^{t+1}_{p+1-t}(\alpha,\ \beta) = 
	\mathrm{P}^{t+1}_{n-t-1-q(p+2)}(\alpha,\ \beta)
  \end{equation}

  Базу индукции при $q = 1$, докажем подставив в формулу (\ref{sup1 up})
  $l = n - (p+2)$.
  \begin{eqnarray*}
	\mathrm{P}^{t+1}_{n-t-1}(\alpha,\ \beta)= 
	\beta^{n-(p+2)}\mathrm{P}^{t+1}_{p+1-t}(\alpha,\ \beta) + 
	\beta^{p+2} \mathrm{P}^{t+1}_{n-t-1-(p+2)}(\alpha,\ \beta)\\
	-\beta^{n-2(p+2)}\mathrm{P}^{t+1}_{p+1-t}(\alpha,\ \beta) =
	\mathrm{P}^{t+1}_{n-t-1-(p+2)}(\alpha,\ \beta)
  \end{eqnarray*}
  База доказана.

  Докажем шаг индукции, подставив в формулу (\ref{sup1 up}) $l = n -(q+1)(p+2)$.
  \begin{eqnarray*}
	\mathrm{P}^{t+1}_{n-t-1-q(p+2)}(\alpha,\ \beta) = 
	\beta^{n-(q+1)(p+2)}\mathrm{P}^{t+1}_{p+1-t}(\alpha,\ \beta) +
	\beta^{p+2}\mathrm{P}^{t+1}_{n-1-t-(q+1)(p+2)}(\alpha,\ \beta)\\
	-(q+1)\beta^{n-(q+2)(p+2)}\mathrm{P}^{t+1}_{p+1-t}(\alpha,\ \beta) =
	\mathrm{P}^{t+1}_{n-t-1-(q+1)(p+2)}(\alpha,\ \beta)
  \end{eqnarray*}

  Тем самым доказан формула (\ref{sup2 up}).
  Под ставим в формулу (\ref{sup2 up}) $q = q_0$, 
  где $q_0$ определенно из соотношения $n = (q_0+1)(p+2)$.

  Тогда получим
  \[
	-q_0\beta^{n-(q_0+1)(p+2)}\mathrm{P}^{t+1}_{p+1-t}(\alpha,\ \beta) =
	\mathrm{P}^{t+1}_{n-t-1-q_0(p+2)}(\alpha,\ \beta)
  \]
  \[
	(1+q_0)\mathrm{P}^{t+1}_{p+1-t}(\alpha,\ \beta) = 0
  \]
	Так как $\char \fld \nmid n$, значит $1+q_0 \neq 0$.
	Тем самым $\mathrm{P}^{t+1}_{p+1-t}(\alpha,\ \beta) = 0$.
	Если мы подставим это в формулу (\ref{sup1 up}), то получим
	\[
	  \mathrm{P}^{t+1}_{p+1-t+l}(\alpha,\ \beta) = 
	  \beta^{p+2}\mathrm{P}^{t+1}_{l-1-t}(\alpha,\ \beta)
	\]
	Тем самым оба утверждения индукции доказаны, а значит и вся лемма.
\end{proof}

Теперь можем доказать теорему (\ref{main})

\begin{proof}[Доказательство теоремы \ref{main}]
  Если для некоторого $k$, такого что $k+1$ делит $n$ $\mathrm{P}^1_k$, 
  то из леммы (\ref{support up}) матрица становиться блочно диагональной указанного вида
  Из лемм (\ref{support1}) и (\ref{support2}) следует, что 
  либо левый верхний блок разбивается ещё на блоки меньшего размера, 
  либо линейно независим.
  Из его линейной независимости следует линейная независимость объединения всех блоков.
  Действительно, применяя шаги из лемм (\ref{support1}) и (\ref{support2})
  мы получим доказательство этого факта.
\end{proof}

Пусть в $\fld$ существует примитивные корни $\xi$ степени $n$ и $\theta$ степени $k$,
где $k$ --- размер блока.
Тогда модульная структура $psi$ эквивалентна 
$\fld\langlea\rangle_n \otimes H_{k^2}(\theta)$-модульной структуре, где это действие задается равенством 
\[
  (a \otimes 1)\bar x = \xi \bar x,\
  (1 \otimes c)\bar x = \capa \bar x,\
  (1 \otimes v)\bar x = \bar 1
\]

\newpage
\section{Классификация модульных структур}

\begin{theorem}
	Пусть $\psi:H \otimes A\ \to \ A $~-- структура $H$-модульной алгебры с $1$ на $A = \mathbb{K}[x]/(x^3)$, где $H$~--- некоторая алгебра Хопфа, $\mathrm{char} \ \mathbb{K} \neq 3$ и в поле существует примитивный корень степени 3. Выберем базис в $A: \bar 1, \bar x, \bar x^2$ и отождествим $\End(A)$ с $M_3(\mathbb{K})$. Предположим, что коноситель действия это подалгебра верхнетреугольных матриц. Тогда $\psi$ эквивалентно одной из следующих модульных структур над $A$:
    
    \begin{enumerate}
        \item действие поля $\mathbb{K}$ на алгебре $A$ умножением на скаляры; \label{scalar}
        
        \item действие групповой алгебры $\mathbb{K}\langle c\rangle_2$, заданное равенством 
        ${c\bar{x} = -\bar{x}}$; \label{eps=b^2}
        
        \item действие групповой алгебры $\mathbb{K}\langle c \rangle_3$, заданное равенством $c\bar{x} = \xi \bar{x}$,
        
        где $\xi$ "--- примитивный корень из единицы степени $3$;\label{diagonal} 
        
        \item $H_9(\xi)$-действие, заданное равенствами $c\bar{x}=\xi\bar{x},\ v\bar{x}= \bar{1}$, 
        
        где $\xi$ "--- примитивный корень из единицы степени $3$. \label{all}
        
    \end{enumerate}
\end{theorem}


\begin{proof}
  Обозначим за $\zeta$ гоморфизм алгебр $H \to M_3(\fld)$. Тогда $\zeta(h)$ будет иметь вид (\ref{matrix}), но для удобства работы подставим вместо $\mathrm{P}^i_j(\alpha,\ \beta)$ сами выражения для $\alpha$ и $\beta$
    \[\zeta(h) = 
    \begin{pmatrix}
        \varepsilon(h) & \alpha(h) & \alpha^2(h)\\
         0 & \beta(h) & (\alpha\beta + \beta\alpha)(h) \\
         0 & 0 & \beta^2(h)
    \end{pmatrix}\]
  
  Тогда соотношения (\ref{null}) примут вид
    \begin{align}
        &\alpha^3 = 0 \label{1}\\
        &\alpha^2\beta+\alpha\beta\alpha + \beta\alpha^2 = 0 \label{2}\\
        &\alpha\beta^2+\beta\alpha\beta+\beta^2\alpha = 0 \label{3}
    \end{align}
    
    В предыдущем разделе были доказаны разложения для результатов коумножения на элементах матрицы модульной структуры. Выпишем их для наших функций:
    \begin{eqnarray*}
        \bigtriangleup\varepsilon &=& \varepsilon \otimes \varepsilon\\
		\bigtriangleup\beta &=& \beta \otimes \beta\\
		\bigtriangleup\beta^2 &=& \beta^2 \otimes \beta^2\\
        \bigtriangleup\alpha &=& \varepsilon \otimes \alpha + \alpha \otimes \beta\\
        \bigtriangleup(\alpha\beta+\beta\alpha) &=& \beta \otimes (\alpha\beta+\beta\alpha) + (\alpha\beta+\beta\alpha)\otimes \beta^2\\
        \bigtriangleup\alpha^2 &=& \varepsilon \otimes \alpha^2 + \alpha \otimes (\alpha\beta+\beta\alpha) + \alpha^2 \otimes \beta^2\\
    \end{eqnarray*}
	Значит функции $\varepsilon,\ \beta, \beta^2$ являются группоподобными. Тогда пользуясь леммой (\ref{lem}) можем сказать, что либо $\beta = \varepsilon$, 
	либо $\beta^2 = \varepsilon$ и $\varepsilon,\ \beta$ линейно независимы, либо $\varepsilon,\ \beta,\ \beta^2$ линейно независимы.

	Если $\beta = \varepsilon$, тогда из (\ref{3}) 
	\[
	  3\alpha = 0
	\]
	Так как $\chr \mathbb{K} \neq 3$, то $\alpha = 0$, а значит наша матрица модульной структуры выглядит следующим образом:\[
    \zeta(h) = \begin{pmatrix}
        \varepsilon(h) & 0 & 0\\
        0 & \varepsilon(h) & 0\\
        0 & 0 & \varepsilon(h)\\
    \end{pmatrix}
    \]
    Тогда $\zeta(H)$ --- подалгебра скалярных матриц. Такая модульная структура эквивалентна структуре \ref{scalar}.

	Пусть $\beta^2 = \varepsilon$ и $\varepsilon,\ \beta$ линейно независимы.
	Тогда из (\ref{3}) следует, что
	\[
	  2\alpha +\beta\alpha\beta = 0
	\]
	А значит \[
	  \alpha = \beta^2\alpha\beta^2 = \beta(\beta\alpha\beta)\beta = \beta (-2\alpha)\beta = -2\beta\alpha\beta = 4\alpha
	\]
	То есть $3\alpha = 0$
	Так как $\chr \mathbb{K} \neq 3$, то $\alpha = 0$, а значит наша матрица модульной структуры выглядит следующим образом:\[
    \zeta(h) = \begin{pmatrix}
        \varepsilon(h) & 0 & 0\\
        0 & \beta(h) & 0\\
        0 & 0 & \varepsilon(h)\\
    \end{pmatrix}
    \]
	Докажем, что в таком случае модульная структура $\psi$ эквивалентна структуре \ref{eps=b^2}.

    Пусть $\psi_1: \mathbb{K}\langle c\rangle_2 \otimes A \to A$ "--- модульная структура \ref{eps=b^2}.
	Обозначим структуру коносителя как $\zeta_1:\mathbb{K}\langle c\rangle_2 \to M_3(\mathbb{K})$. Тогда получаем \[
    \zeta_1(c) = \begin{pmatrix}
        1 & 0 & 0\\
        0 & -1 & 0\\
        0 & 0 & 1\\
    \end{pmatrix}
    \]

    Выберем в $\mathbb{K}\langle c\rangle_3$  базис $(c^k)_{0 \leq k \leq 1}$ и выпишем образ этого базиса:
    \[
    \zeta_1(1) = \begin{pmatrix}
        1 & 0 & 0\\
        0 & 1 & 0\\
        0 & 0 & 1\\
    \end{pmatrix},\
    \zeta_1(c) = \begin{pmatrix}
        1 & 0 & 0\\
        0 & -1 & 0\\
        0 & 0 & 1\\
    \end{pmatrix}
    \]

    Рассмотрим эти матрицы как вектора в базисе из матричных единиц и запишем их координаты построчно в матрицу: \[
        M = \begin{pmatrix}
            1 & 0 & 0 & 0 & 1 & 0 & 0 & 0 & 1\\
            1 & 0 & 0 & 0 & -1 & 0 & 0 & 0 & 1\\
        \end{pmatrix}
    \]
    Нетрудно проверить, что $\mathrm{rank}\ M = 2$, а значит $\dim \mathrm{cosupp}\ \psi_1 = \dim \zeta_1(H) = 2$, следовательно $\mathrm{cosupp}\ \psi_1$ "--- подалгебра диагональных матриц, такая что для любой матрицы $B$, содержащейся в ней, верно $b_{11}=b_{33}$.

	В оставшемся случае $\varepsilon,\ \beta,\ \beta^2$ линейно независимы.
	Пусть $\alpha$ линейно зависима с $\varepsilon,\ \beta,\ \beta^2$. Тогда существуют $\lambda_i$ такие, что 
	\[
	  \alpha = \lambda_1\varepsilon + \lambda_2\beta + \lambda_3\beta^2
	\]
	Значит $\alpha$  и $\beta$ коммутируют и из (\ref{3}) следует, что 
	\[
	  3\alpha\beta^2 = 0
	\]
	Из того, что $\chr \mathbb{K} \neq 3$ и $\beta$ обратим получаем $\alpha = 0$ и 
	\[
	  \zeta(h) = \begin{pmatrix} 
		\varepsilon && 0 && 0\\
		0 && \beta && 0\\
		0 && 0 && \beta^2\\
	  \end{pmatrix}
	\]

	Докажем, что тогда модульная структуре $\psi$ эквивалентна структуре \ref{diagonal}.
    Пусть $\psi_1: \mathbb{K}\langle c\rangle_3 \otimes A \to A$ "--- модульная структура \ref{diagonal}.
	Обозначим структуру коносителя как $\zeta_1:\mathbb{K}\langle c\rangle_3 \to M_3(\mathbb{K})$. Тогда получаем \[
    \zeta_1(c) = \begin{pmatrix}
        1 & 0 & 0\\
        0 & \xi & 0\\
        0 & 0 & \xi^2\\
    \end{pmatrix}
    \]
    Выберем в $\mathbb{K}\langle c\rangle_3$  базис $(c^k)_{0 \leq k \leq 2}$ и выпишем образ этого базиса:
    \[
    \zeta_1(1) = \begin{pmatrix}
        1 & 0 & 0\\
        0 & 1 & 0\\
        0 & 0 & 1\\
    \end{pmatrix},\
    \zeta_1(c) = \begin{pmatrix}
        1 & 0 & 0\\
        0 & \xi & 0\\
        0 & 0 & \xi^2\\
    \end{pmatrix},\ 
    \zeta_1(c^2) = \begin{pmatrix}
        1 & 0 & 0\\
        0 & \xi^2 & 0\\
        0 & 0 & \xi\\
    \end{pmatrix}
    \]

    Рассмотрим эти матрицы как вектора в базисе из матричных единиц и запишем их координаты построчно в матрицу: \[
        M = \begin{pmatrix}
            1 & 0 & 0 & 0 & 1 & 0 & 0 & 0 & 1\\
            1 & 0 & 0 & 0 & \xi & 0 & 0 & 0 & \xi^2\\
            1 & 0 & 0 & 0 & \xi^2 & 0 & 0 & 0 & \xi\\
        \end{pmatrix}
    \]
    Нетрудно проверить, что $\mathrm{rank}\ M = 3$, а значит $\dim \mathrm{cosupp}\ \psi_1 = \dim \zeta_1(H) = 3$, следовательно $\mathrm{cosupp}\ \psi_1$ "--- алгебра диагональных матриц.

	Теперь будем считать функцию $\alpha$ линейно независимой с $\varepsilon,\ \beta,\ \beta^2$
	Тогда применима теорема (\ref{main}). 
	Так как $n$ простое, то собственных делителей у $n$ нет, 
	значит вся матрица является одним блоком и система $\varepsilon,\ \beta,\ \beta^2,\ \alpha,\ \alpha\beta+\beta\alpha,\ \alpha^2$ линейно независима.
	
    \noindentДокажем, что в таком случае модульная структура $\psi$ эквивалентна модульной структуре \ref{all}.
    Пусть $\psi_1:H_9(\xi) \otimes A \to A$ "--- модульная структура \ref{all}.
    Ообозначим структуру коносителя как $\zeta_1:H_9(\xi) \to M_3(\mathbb{K})$.
    Тогда мы получим \[
    \zeta_1(c) = 
    \begin{pmatrix}
        1 & 0 & 0\\
        0 & \xi & 0\\
        0 & 0 & \xi^2\\
    \end{pmatrix},\ 
    \zeta_1(v) = 
    \begin{pmatrix}
        0 & 1 & 0\\
        0 & 0 & 1+\xi\\
        0 & 0 & 0\\
    \end{pmatrix}
    \]

    Выпишем образ базиса $(c^kv^l)_{0 \leq k,l \leq 2}$ в $\zeta(H)$:
    \begin{eqnarray*}
    &\zeta_1(1) = \begin{pmatrix}
    1 & 0 & 0\\
    0 & 1 & 0\\
    0 & 0 & 1\\
    \end{pmatrix},\
    \zeta_1(c) = \begin{pmatrix}
    1 & 0 & 0\\
    0 & \xi & 0\\
    0 & 0 & \xi^2\\
    \end{pmatrix},\
    \zeta_1(c^2) = \begin{pmatrix}
    1 & 0 & 0\\
    0 & \xi^2 & 0\\
    0 & 0 & \xi\\
    \end{pmatrix}\\
    &\zeta_1(v) = \begin{pmatrix}
    0 & 1 & 0\\
    0 & 0 & 1+\xi\\
    0 & 0 & 0\\
    \end{pmatrix},\
    \zeta_1(cv) = \begin{pmatrix}
    0 & 1 & 0\\
    0 & 0 & \xi+\xi^2\\
    0 & 0 & 0\\
    \end{pmatrix} 
    \zeta_1(c^2v) = \begin{pmatrix}
    0 & 1 & 0\\
    0 & 0 & 1+\xi^2\\
    0 & 0 & 0\\
    \end{pmatrix}\\
    &\zeta_1(v^2) = \begin{pmatrix}
    0 & 0 & 1+\xi\\
    0 & 0 & 0\\
    0 & 0 & 0\\
    \end{pmatrix},\
    \zeta_1(cv^2) = \begin{pmatrix}
    0 & 0 & 1+\xi\\
    0 & 0 & 0\\
    0 & 0 & 0\\
    \end{pmatrix} 
    \zeta_1(c^2v^2) = \begin{pmatrix}
    0 & 0 & 1+\xi\\
    0 & 0 & 0\\
    0 & 0 & 0\\
    \end{pmatrix}\\    
    \end{eqnarray*}

    Рассмотрим эти матрицы как вектора в базисе из матричных единиц и запишем их координаты построчно в матрицу: \[ M=
    \begin{pmatrix}
        1 & 0 & 0 & 0 & 1 & 0 & 0 & 0 & 1\\
        1 & 0 & 0 & 0 & \xi & 0 & 0 & 0 & \xi^2\\
        1 & 0 & 0 & 0 & \xi^2 & 0 & 0 & 0 & \xi\\
        0 & 1 & 0 & 0 & 0 & 1+\xi & 0 & 0 & 0\\
        0 & 1 & 0 & 0 & 0 & \xi+\xi^2 & 0 & 0 & 0\\
        0 & 1 & 0 & 0 & 0 & 1+\xi^2 & 0 & 0 & 0\\
        0 & 0 & 1+\xi & 0 & 0 & 0 & 0 & 0 & 0\\
        0 & 0 & 1+\xi & 0 & 0 & 0 & 0 & 0 & 0\\
        0 & 0 & 1+\xi & 0 & 0 & 0 & 0 & 0 & 0\\
    \end{pmatrix}
    \]
    Нетрудно, проверить, что $\mathrm{rank}\ M = 6$, а значит $\dim \mathrm{cosupp} \ \psi_1 = 6$, откуда $\mathrm{cosupp}\ \psi_1$ совпадает c алгеброй верхнетреугольных матриц.
\end{proof}


\newpage
\begin{thebibliography}{99}
	
\normalsize


\bibitem{ArnoldBook} Арнольд В.\,И. Математические методы классической механики. М.: Эдиториал УРСС, 2003, 416~с.

\bibitem{ModernGeometry}Дубровин~Б.\,А., Новиков~С.\,П., Фоменко~А.\,Т. Современная геометрия: в 3 т.
М.: Эдиториал УРСС, 2000.


\bibitem{MurphyBook} Мёрфи~Дж. $C^*$-алгебры и теория операторов. М.:~Факториал, 1997, 336~с.


\bibitem{PolyakovBook} Поляков~А.\,М. Калибровочные поля и струны. Ижевск: Издательский дом <<Удмуртский университет>>, 1999, 312~с.

\bibitem{Abe} Abe, E. Hopf algebras. Cambridge University Press, Cambridge, 1980.

\bibitem{ASGordienko21ALAgoreJVercruysse}
Agore, A.\,L., Gordienko, A.\,S., Vercruysse, J.
$V$-universal Hopf algebras (co)acting on $\Omega$-algebras. \textit{Commun. Contemp. Math.},
\textbf{25}:1 (2023), 2150095-1 -- 2150095-40. 

\bibitem{ConnesMarcolli} Connes, A., Marcolli, M. Quantum fields, noncommutative spaces and motives. (Книга готовится к печати, электронная версия \url{https://www.alainconnes.org/docs/bookwebfinal.pdf})


\bibitem{Danara} D\u asc\u alescu, S., N\u ast\u asescu, C., Raianu, \c S.
Hopf algebras: an introduction. New York, Marcel Dekker, Inc., 2001.


\bibitem{Donatsos} Donatsos, D., Daskaloyannis, C. Quantum groups and their applications in nuclear physics. \textit{Progress in Particle and Nuclear Physics},  \textbf{43} (1999), 537--618.


\bibitem{HaagBook} Haag, R. Local quantum physics: fields, particles, algebras.
Springer-Verlag, Berlin, Heidelberg, 1996.

 
\bibitem{HaagKastler} Haag, R., Kastler, D. An algebraic approach to quantum field theory.
\textit{J. Math. Phys.}, \textbf{5} (1964), 848--861.


\bibitem{KakuBook} Kaku, M. Introduction to superstrings and M-theory. Springer-Verlag, New York, 1999.


\bibitem{MajidBook} Majid, S. Foundations of quantum group theory. Cambridge University Press, 1995.



\bibitem{Montgomery} Montgomery, S. Hopf algebras and their actions on rings. \textit{CBMS Lecture Notes} \textbf{82}, Amer. Math. Soc., Providence, RI, 1993.


\bibitem{Sweedler} Sweedler, M.\,E. Hopf Algebras. W.\,A. Benjamin, New York, 1969.

\end{thebibliography}

\end{document}
