\documentclass[12pt, reqno, a4paper, oneside, notitlepage]{amsart}

\usepackage{color}
\definecolor{darkblue}{rgb}{0,0,0.75}
\definecolor{darkgreen}{rgb}{0,0.75,0}
\usepackage[%pagebackref,
colorlinks, linkcolor=darkblue,
citecolor=darkgreen, urlcolor=blue, bookmarks=false, breaklinks]{hyperref}

\usepackage{cmap}

\usepackage[utf8]{inputenc}
\usepackage[T2A]{fontenc}
\usepackage[english,russian]{babel}

\usepackage{indentfirst}
\usepackage[left=2.5cm, right=1.5cm, top=2cm, bottom=2cm]{geometry}
\setlength\parindent{5ex}

\usepackage{fancyhdr}
\fancypagestyle{plain}{
	\fancyhf{}
	\chead{\thepage}
	\renewcommand{\headrulewidth}{0pt}}
\linespread{1.25}

\usepackage{tocloft}
\usepackage[ddmmyyyy]{datetime}

\usepackage{titlesec}
\titleformat{\section}[block]{\bfseries\large\centering}{\thesection}{1ex}{}
\titleformat{\subsection}[block]{\bfseries\normalsize\centering}{\thesubsection}{1ex}{}

\usepackage{amssymb, amsmath, amsthm, longtable, bbm}
\usepackage[matrix, arrow, curve]{xy}

\usepackage[geometry]{ifsym}

\renewcommand{\square}{\text{\SmallSquare}}

\binoppenalty=\maxdimen
\relpenalty=\maxdimen

\righthyphenmin=2
\sloppy

\newtheoremstyle{mytheoremstyle} % name
{\topsep}                    % Space above
{\topsep}                    % Space below
{\itshape}                   % Body font
{5ex}                           % Indent amount
{\bfseries}                   % Theorem head font
{.}                          % Punctuation after theorem head
{.5em}                       % Space after theorem head
{}  % Theorem head spec (can be left empty, meaning ‘normal’)

\newtheoremstyle{myremarkstyle} % name
{\topsep}                    % Space above
{\topsep}                    % Space below
{}                   % Body font
{5ex}                           % Indent amount
{\bfseries}                   % Theorem head font
{.}                          % Punctuation after theorem head
{.5em}                       % Space after theorem head
{}  % Theorem head spec (can be left empty, meaning ‘normal’)

\theoremstyle{mytheoremstyle}
\newtheorem{theorem}{Теорема}[section]
\newtheorem{proposition}[theorem]{Предложение}
\newtheorem{corollary}[theorem]{Следствие}
\newtheorem{lemma}[theorem]{Лемма}
\newtheorem*{conjecture}{Гипотеза}

\theoremstyle{myremarkstyle}
\newtheorem{remark}[theorem]{Замечание}
\newtheorem{agreement}[theorem]{Соглашение}
\newtheorem{example}[theorem]{Пример}
\newtheorem*{examples}{Примеры}
\newtheorem*{questions}{Вопросы}
\newtheorem*{question}{Вопрос}
\newtheorem{definition}[theorem]{Определение}
\newtheorem{problem}[theorem]{Задача}

\numberwithin{equation}{section}

\makeatletter
\renewenvironment{proof}[1][\proofname]{\par\indent {\bfseries #1\@addpunct{.} }}{\qed}
\makeatother

\DeclareMathOperator{\Id}{Id}
\DeclareMathOperator{\Der}{Der}
\DeclareMathOperator{\id}{id}
\DeclareMathOperator{\chr}{char}
\DeclareMathOperator{\ad}{ad}
\DeclareMathOperator{\tr}{tr}
\DeclareMathOperator{\GL}{GL}
\DeclareMathOperator{\UT}{UT}
\DeclareMathOperator{\SL}{SL}
\DeclareMathOperator{\diff}{diff}
\DeclareMathOperator{\supp}{supp}
\DeclareMathOperator{\cosupp}{cosupp}
\DeclareMathOperator{\PGL}{PGL}
\DeclareMathOperator{\PSL}{PSL}
\DeclareMathOperator{\height}{ht}
\DeclareMathOperator{\End}{End}
\DeclareMathOperator{\Aut}{Aut}
\DeclareMathOperator{\diag}{diag}
\DeclareMathOperator{\rank}{rank}
\DeclareMathOperator{\length}{length}
\DeclareMathOperator{\sign}{sign}
\DeclareMathOperator{\Alt}{Alt}
\DeclareMathOperator{\Ann}{Ann}
\DeclareMathOperator{\Hom}{Hom}
\DeclareMathOperator{\op}{op}
\DeclareMathOperator{\PIexp}{PIexp}
\newcommand{\hatotimes}{\mathbin{\widehat{\otimes}}}
\DeclareMathOperator{\Ker}{Ker}
\DeclareMathOperator{\Coker}{Coker}
\renewcommand{\Im}{\mathop\mathrm{Im}}
\DeclareMathOperator{\Coim}{Coim}
\DeclareMathOperator{\Ext}{Ext}
\DeclareMathOperator{\Tor}{Tor}

%%%%%%%%%%%%%%% Fancy symbol \No %%%%%%%%%%%%%%%%%%%%%%%%%%%%%%%%%%%%%%%%%%%%%%

\DeclareRobustCommand{\No}{\ifmmode{\nfss@text{\textnumero}}\else\textnumero\fi} 


%%%%%%%%%%%%%%% Pullbacks van Tim van der Linden %%%%%%%%%%%%%%%%%%%%%%%%%%%%%

\newbox\skewpullbackbox
\setbox\skewpullbackbox=\hbox{\xy 0;<1mm,0mm>: \POS(4,0)\ar@{-} (-4,0) \ar@{-} (8,4)
	\endxy}
\newcommand{\skewpullback}{\copy\skewpullbackbox}

\newbox\skwepullbackbox
\setbox\skwepullbackbox=\hbox{\xy 0;<1mm,0mm>: \POS(16,0)\ar@{-} (10,0) \ar@{-} (12,4)
	\endxy}
\newcommand{\skwepullback}{\copy\skwepullbackbox}

\newbox\ksewpullbackbox
\setbox\ksewpullbackbox=\hbox{\xy 0;<1mm,0mm>: \POS(0,-8)\ar@{-} (0,-4) \ar@{-} (4,-4)
	\endxy}
\newcommand{\ksewpullback}{\copy\ksewpullbackbox}

\newbox\pullbackbox
\setbox\pullbackbox=\hbox{\xy 0;<1mm,0mm>: \POS(4,0)\ar@{-} (0,0) \ar@{-} (4,4)
	\endxy}
\newcommand{\pullback}{\copy\pullbackbox}

\newbox\pullbackabox
\setbox\pullbackabox=\hbox{\xy 0;<1mm,0mm>: \POS(-4,-6)\ar@{-} (-8,-6) \ar@{-} (-4,-2)
	\endxy}
\newcommand{\pullbacka}{\copy\pullbackabox}

\newbox\pullbackbbox
\setbox\pullbackbbox=\hbox{\xy 0;<1mm,0mm>: \POS(-4,-4)\ar@{-} (-8,-4) \ar@{-} (-4,0)
	\endxy}
\newcommand{\pullbackb}{\copy\pullbackbbox}

\newbox\pullbackcbox
\setbox\pullbackcbox=\hbox{\xy 0;<1mm,0mm>: \POS(4,-7)\ar@{-} (0,-7) \ar@{-} (4,-3)
	\endxy}
\newcommand{\pullbackc}{\copy\pullbackcbox}


\newbox\pullbackdbox
\setbox\pullbackdbox=\hbox{\xy 0;<1mm,0mm>:\POS(-10,-6)\ar@{-} (-14,-6) \ar@{-} (-10,-2)
	\endxy}
\newcommand{\pullbackd}{\copy\pullbackdbox}


\newbox\pushoutbox
\setbox\pushoutbox=\hbox{\xy 0;<1mm,0mm>: \POS(0,4)\ar@{-} (0,0) \ar@{-} (4,4)
	\endxy}
\newcommand{\pushout}{\copy\pushoutbox}

\newbox\pushoutabox
\setbox\pushoutabox=\hbox{\xy 0;<1mm,0mm>: \POS(2,8)\ar@{-} (2,4) \ar@{-} (8,8)
	\endxy}
\newcommand{\pushouta}{\copy\pushoutabox}

\begin{document}

{\fontsize{14pt}{18}\selectfont
  
  \thispagestyle{empty}
  \begin{center}
		
		\vfill\vfill \ \\ {%\Large
			МОСКОВСКИЙ ГОСУДАРСТВЕННЫЙ УНИВЕРСИТЕТ \\
			имени М.В.~ЛОМОНОСОВА
			
			\medskip
			
			МЕХАНИКО-МАТЕМАТИЧЕСКИЙ ФАКУЛЬТЕТ

			\medskip
			
			
			КАФЕДРА ВЫСШЕЙ АЛГЕБРЫ

		}



{%\Large
	\vfill {%\Large
		
		КУРСОВАЯ РАБОТА
		
	}
		
		
		\vfill{\Large
			\textbf{Квантовые симметрии \linebreak в алгебре тройных чисел} 
		}
		
			

			\vfill
			\begin{flushright}
				\begin{tabular}{r}
					Выполнил:\\
студент 211 группы \\
Зазовский Леон Станиславович
					\\
					\\
					Научный руководитель: \\
					доктор физико-математических наук, \\
					профессор Гордиенко Алексей Сергеевич
				\end{tabular}
			\end{flushright}
			
		}
		
		\vfill\vfill\vfill\vfill 
		
		Москва
		
		 2025
	\end{center}
}

\newpage

\thispagestyle{empty}
\tableofcontents

\newpage

\pagestyle{plain}
\section{Введение}

Во многих областях математики и физики (см., например, \cite{ArnoldBook, ModernGeometry, MurphyBook, HaagKastler}) находят своё применение \textit{алгебры}, то есть векторные пространства над некоторым полем $\mathbbm{k}$ (например, $\mathbbm{k}$ может быть полем $\mathbb{R}$ вещественных или $\mathbb{C}$ комплексных чисел), в которых задана бинарная операция внутреннего умножения, линейная по каждому аргументу. 

Часто алгебры, встречающиеся в приложениях, наделены некоторой дополнительной структурой или (обобщёнными) симметриями: действием (полу)группы эндоморфизмами и антиэндоморфизмами, (полу)групповой градуировкой или действием алгебры Ли дифференцированиями (см., например, \cite{PolyakovBook, HaagBook, KakuBook, MajidBook}). Для работы с такими дополнительными структурами оказывается полезным понятие модульной и комодульной алгебры над алгеброй Хопфа и даже более общее понятие алгебры с обобщённым $H$-действием. В частности, понятие (ко)модульной алгебры позволяет изучать различные виды дополнительных структур на алгебрах одновременно.

Кроме того, (ко)модульные алгебры естественным образом возникают в геометрии: если некоторая аффинная алгебраическая группа действует морфизмами на аффинном алгебраическом многообразии (например, рассматриваются симметрии некоторой поверхности, заданной алгебраическими уравнениями), то алгебра регулярных функций (множество функций, которые можно определить при помощи многочленов от координат точки,
с операциями сложения, умножения между собой и на скаляры) будет модульной алгеброй над групповой алгеброй этой группы и над универсальной обёртывающей алгебры Ли этой группы и комодульной алгеброй над алгеброй регулярных функций на аффинной алгебраической группе~\cite{Abe}.

Обратим внимание на то, что в классическом случае алгебры регулярных функций на многообразиях коммутативны, так как для умножения функций выполнен перестановочный закон. Однако в новом направлении, которое получило название некоммутативной геометрии, рассматриваются <<некоммутативные пространства>>, т. е. такие пространства, алгебры регулярных функций которых некоммутативны. Поэтому изучение (ко)действий необязательно коммутативных алгебр Хопфа на необязательно коммутативных алгебрах можно интерпретировать как изучение квантовых симметрий некоммутативных пространств. Последние находят своё применение в теоретической физике (см., например, \cite{ConnesMarcolli, Donatsos}).

В данной работе разрабатываются методы для исследования квантовых симметрий на алгебрах полученных из алгебры многочленов путем факторизования по различным идеалам. Для этого используются понятия эквивалентности модульных структр (введённое в \cite{ASGordienko21ALAgoreJVercruysse}) и структуры коносителя, введённое автором в данной работе. Основная идея исследования заключается в том, чтобы в случае конечномерности алгебры, на которой мы действуем, свести изучение квантовых симметрий к рассмотрению конечномерной подкоалгебры в конечной двойственной к действующей алгебре Хопфа.

\newpage

\section{Основные понятия}

В данном параграфе мы напомним основные понятия, связанные с (ко)алгебрами, алгебрами Ли, биалгебрами и алгебрами Хопфа.
Подробнее с этими понятиями можно познакомиться в монографиях~\cite{HumphreysLieAlg, Danara, Montgomery, Sweedler}.

Прежде всего напомним понятие алгебры над полем в удобной для нас форме, а именно, на языке линейных отображений и коммутативных диаграмм. 

\begin{definition} 
	\textit{Алгеброй над полем $\mathbbm{k}$} называется пара $(A, \mu)$, состоящая из
	векторного пространства $A$ над $\mathbbm{k}$ и линейного отображения
	$\mu \colon A \otimes A \to A$.
\end{definition}

При помощи отображения $\mu$ на векторном пространстве $A$ задаётся операция внутреннего умножения, линейная по каждому аргументу:
$ab := \mu(a\otimes b)$ для всех $a,b\in A$.

Алгебра $(A, \mu)$ называется \textit{ассоциативной}, если
следующая диаграмма коммутативна: 
  \[\xymatrix{ 
	A \otimes A \otimes A \ar[d]_{\mu \otimes \id_A} \ar[rr]^{\id_A \otimes \mu} && A\otimes A
	\ar[d]_\mu \\
	A\otimes A \ar[rr]^\mu && A  
  }\]

\begin{definition}
  Набор $(A, \mu, u)$ называется \textit{алгеброй с единицей над полем $\mathbbm{k}$}, если $(A, \mu)$ --- алгебра над $\mathbbm{k}$, а $u \colon \mathbbm{k} \to A$ --- линейное отображение и, кроме того, следующие диаграммы коммутативны:
  \[\xymatrix{ 
	  A \otimes \mathbbm{k} \ar[rd]^\sim \ar[rr]^{\id_A \otimes u}&& A\otimes A \ar[ld]^\mu \\
	  &A
  }
  \hspace{2em}
  \text{и}
  \hspace{2em}
  \xymatrix{ 
	  \mathbbm{k} \otimes A \ar[rd]^\sim \ar[rr]^{u \otimes \id_A}&& A\otimes A \ar[ld]^\mu \\
	  &A
  }\]
\end{definition}
\noindent(Здесь мы использовали естественные отждествления $A \otimes \mathbbm{k} \cong \mathbbm{k} \otimes A \cong A$.)


Если ввести обозначение $1_A :=u(1_\mathbbm{k})$, то элемент $1_A \in A$ будет удовлетворять классическому определению единицы в алгебре.
\newpage
\section{Конечно двойственная алгебра Хопфа}

\subsection{Структура конечного двойственного пространства}

Пусть у нас задано векторное пространство $V$ над полем $\mathbb{K}$. 

Известно, что если $V$ конечномерное, тогда $V \cong V^*$ и существует канонический изоморфизм между $V$ и $V^{**}$. Оказывается, что похожий результат верен в бесконечномерных пространствах. 

\begin{definition} 
Конечное двойственное пространство $V^\circ$"--- это подпространство $V^*$ такое, что для любой линейной функции $\alpha \in V^\circ$ её ядро имеет конечную коразмерность.
\end{definition}

В случае конечномерных пространств $V^\circ = V^*$.

\begin{definition}
    Пусть в $V$ выбран базис $(e_i)_{i \in \Lambda}$. Тогда 
    двойственная система $(\varepsilon_i)_{i \in \Lambda}$ в $V^\circ$ "--- это система функций  таких, что для всех $i, j \in \Lambda$ верно соотношение $\varepsilon_i(e_j) = \delta^i_j$, где $\delta^i_j$ "--- символ Кронекера.
\end{definition}

В случае конечномерных пространств двойственная система окажется в точности системой координатных функций. 

\begin{theorem}
    Двойственная система является базисом в $V^\circ$
\end{theorem}

\begin{proof}
    
    Сначала докажем, что двойственная система полна в $V^\circ$. Рассмотрим некоторую линейную функцию $f$ из $V^\circ$.
    Тогда существует векторное подпространство $W$ такое, что $V$, раскладывается в прямую сумму ${V = W \oplus \mathrm{Ker} (f)}$.

    Рассмотрим проектор $\pi$ на подпространство $W$ вдоль $\mathrm{Ker}(f)$. Нетрудно, заметить что $\mathrm{Ker}(\pi) = \mathrm{Ker}(f)$ и $\mathrm{Im}(\pi) = W$. Значит $W \cong V/\mathrm{Ker}(f)$.
    Отсюда следует, что $W$ имеет конечную размерность, обозначим её как $m$. Выберем базис $a_1, \dots, a_m$ в $W$.
    Для любого вектора $v$ верно разложение $v = \lambda_1a_1+ \dots +\lambda_ma_m + u$, где $u$ некоторый вектор из $\mathrm{Ker}(f)$. Тогда $f(v) =  f(\lambda_1a_1+ \dots +\lambda_ma_m + u) = \lambda_1f(a_1)+ \dots +\lambda_mf(a_m)$. 
    Мы получили, что значение $f$ на произвольном векторе является линейной комбинацией значений на $(a_i)_{i = 1}^m$

    Так как $(e_i)_{i \in \Lambda}$ "--- базис, значит все $(a_j)_{j = 1}^m$ выражаются с помощью конечной линейной комбинации $(e_i)_{i \in \Lambda}$.
    Отсюда понятно, что значение $f$ на произвольном векторе является конечной линейной комбинацией значений на конечном подмножестве базисных векторов $(e_i)_{i \in \Lambda}$. Обозначим это подмножество $e_{i_1}, \dots, e_{i_n}$.
    Тогда можно выразить $f = f(e_{i_1})\varepsilon_{i_1} + \dots + f(e_{i_n})\varepsilon_{i_n}$.
    А это значит, что двойственная система полна.

    Теперь докажем линейную независимость двойственной системы. Рассмотрим линейную комбинацию некоторой конечной подсистемы $\lambda_{i_1}\varepsilon_{i_1} + \dots + \lambda_{i_n}\varepsilon_{i_n} = 0$. Подставим $e_{i_k}$, получим $\lambda_{i_k}\varepsilon_{i_k}(e_{i_k}) = 0$, а значит $\lambda_{i_k} = 0$. 
    Взяв все возможные $k$ от $1$ до $n$ получим, что все коэффициенты равны нулю, что означает линейную независимость этой подсистемы. Следовательно все конечные подсистемы линейно независимы, а значит и двойственная система линейно независима.

\end{proof}

\begin{remark}
    В силу доказанной выше теоремы будем называть двойственную систему "--- конечно двойственным базисом.
\end{remark}
Для полного соответствия аналогичным результатам в случае конечномерных пространств остатётся доказать две нижеследующие теоремы.

\begin{theorem}
    Пусть задано векторное пространство $V$ над полем $\mathbb{K}$. Тогда $V \cong V^\circ$
\end{theorem}

\begin{proof}
    Выберем в $V$ базис $(e_i)_{i \in \Lambda}$ и соответсвующий ему конечно двойственный базис $(\varepsilon_i)_{i \in \Lambda}$
    Пусть $\varphi: V \to V^\circ$ "--- линейное отображение заданное на базисе $\varphi(e_i) := \varepsilon_i$. Очевидно, что $\varphi$ сюръективно.

    Докажем инъективность. Пусть $\varphi(a) = \varphi(b)$ для некоторых $a, b \in V$. Тогда $\varphi(a-b) = 0$. Разложим $a-b$ по базису $V$: $a-b = \lambda_{i_1}e_{i_1} + \dots + \lambda_{i_n}e_{i_n}$.
    Тогда $\varphi(\lambda_{i_1}e_{i_1} + \dots + \lambda_{i_n}e_{i_n}) = \lambda_{i_1}\varepsilon_{i_1} + \dots + \lambda_{i_n}\varepsilon_{i_n} = 0$. В силу линейной независимости $(\varepsilon_{i_k})_{k=1}^n$, получаем $\lambda_{i_1} =\dots=\lambda_{i_n} = 0$. Значит $a-b =0$, следовательно $\varphi$ инъективно. Тогда $\varphi$ изоморфизм.
\end{proof}

\begin{theorem}
    Существует канонический изоморфизм $\varphi: V \to V^{\circ\circ}$.
\end{theorem}

\begin{proof}
    Определим $\varphi: V \to V^{\circ\circ}$, как $\varphi(u)(\alpha):= \alpha(u)$, для любых $u \in V$ и $\alpha \in V^\circ$. 
    Докажем корректность определения, а именно тот факт, что $\varphi(u) \in V^{\circ\circ}$.
    Ядро $\varphi(u)$ это множество линейных функций $\alpha$ таких, что $\alpha(u) = 0$.
    Выберем базис $(e_i)_{i \in \Lambda}$ в $V$ и соответствующий ему конечно двойственный базис $(\varepsilon_i)_{i \in \Lambda}$ в $V^\circ$. Пусть $u = \lambda_{i_1}e_{i_1}+ \dots +\lambda_{i_m}e_{i_m}$, тогда $\varepsilon_j(u) = \lambda_j$, если для некоторого $k$ верно $j=i_k$, и 
    $\varepsilon_j(u)= 0$ иначе. Отсюда получаем что $V^\circ/\mathrm{Ker}(\varphi(u)) \cong \langle\varepsilon_{i_1}, \dots , \varepsilon_{i_m} \rangle$, откуда следует, что $\mathrm{Ker}(\varphi(u))$ имеет конечную коразмерность.

    Теперь докажем, что $\varphi$ "--- изоморфизм.
    Пусть $\varphi(a)(\alpha) = \varphi(b)(\alpha)$ для любой функции $\alpha \in V^\circ$. 
    Тогда для любого $i \in \Lambda$ верно $\varphi(a)(\varepsilon_i)= \varphi(b)(\varepsilon_i)$, иначе говоря $\varepsilon_i(a) = \varepsilon_i(b)$ для любого  $i \in \Lambda$. Это значит что $a$ и $b$ имеют одинаковые разложения по базису $(e_i)_{i \in \Lambda}$. Получаем, что $a = b$, следовательно $\varphi$ инъективен.

    Пусть $f \in V^{\circ\circ}$. Выберем в $V^{\circ\circ}$ базис $(\epsilon_i)_{i \in \Lambda}$ конечно двойственный к $(\varepsilon_i)_{i \in \Lambda}$. 
    Разложим функцию $f$ по $(\epsilon_i)_{i \in \Lambda}$: $f = \lambda_{i_1}\epsilon_{i_1} + \dots + \lambda_{i_k}\epsilon_{i_k}$. 
    Рассмотрим вектор $u \in V$, такой что $u = \lambda_{i_1}e_{i_1} + \dots + \lambda_{i_k}e_{i_k}$ и докажем, что $\varphi(u) = f$.
    
    $f(\varepsilon_j) = \lambda_j$, если для некоторого $k$ верно, что $j = i_k$, иначе $f(\varepsilon_j) = 0$. $\varphi(u)(\varepsilon_j) = \varepsilon_j(u) = \lambda_j$, если для некоторого $k$ верно, что $j = i_k$, иначе $\varphi(u)(\varepsilon_j) = \varepsilon_j(u) = 0$.
    Значит $f$ и $\varphi(u)$ совпадают на базисе, следовательно $f = \varphi(u)$. Тем самым мы доказали, что $\varphi$ сюръективен.
    
\end{proof}

Для удобства рассуждений введём ещё одно понятие.

\begin{definition}

Пусть задано векторное пространство $V$ над полем $\mathbb{K}$ и система $K \subseteq V^*$.
Единицей линейной функции $\alpha \in K$ по системе $K$~--- будем называть вектор $v\in V$, такой что $\alpha(v) = 1$ и для любой линейной функции $\beta \in K$ отличной от $\alpha$ верно $\beta(v) = 0$
\end{definition}
Будем обозначать единицу линейной функции $\alpha$ как $\alpha_\varepsilon$.
Из рассуждений выше следует, что для любой функции из конечно двойственного базиса существует единица линейной функции.

\newpage

\subsection{Структура конечно двойственной алгебры Хопфа}

Пусть задана $A$ "--- алгебра с единицей над полем $\mathbb{K}$. Рассмотрим подмножество линейных функций $A^\circ \subseteq A^*$, такое что в ядре любой функции $\alpha \in A^\circ$ содержится идеал конечной коразмерности. Тогда $\mu^*(A^\circ) \subseteq A^\circ \otimes A^\circ$, а значит можно определить конечную двойственную коалгебру.

\begin{definition}
    Конечная двойственная коалгебра $A^\circ$ "--- это коалгебра $(A^\circ, \mu^\circ, u^\circ)$, где $\mu^\circ$ и $u^\circ$ это ограничения $\mu^*$ и $u^*$ на $A^\circ$ соответственно.
\end{definition}

Как множество $A^\circ$ является подпространством в конечно двойственном пространстве к векторному пространству $A$. 
Выберем в $A^\circ$ базис $(\varepsilon_i)_{i \in \Lambda}$ и дополним его до базиса всего конечно двойственного пространства.
Рассмотрим конечно двойственный базис в $A$, в нём существует подсистема $(e_i)_{i \in \Lambda}$ из единиц линейных функций составляющих базис $A^\circ$. Она обладает следующим свойством $\varepsilon_i(e_j) = \delta^i_j$, где $\delta^i_j$ "--- символ Кронекера.

Пусть задана $H$ "--- алгебра Хопфа над полем $\mathbb{K}$. Рассмотрим $H^\circ$ "--- конечную двойственную коалгебру. Как множество $H^\circ$ является подалгеброй в $H^*$, поэтому $H^\circ$ является алгеброй Хопфа.

\begin{definition}
    Пусть задана $H$ "--- алгебра Хопфа над полем $\mathbb{K}$. $H^\circ$ называют конечной двойственной алгеброй Хопфа.
\end{definition}

Здесь и далее мы пользуемся обозначениями Свидлера и для удобства опускаем знак суммы: $\bigtriangleup a = a_{(1)} \otimes a_{(2)}$. 
Выпишем некоторые полезные соотношения, следующие из двойственности. Для любых $\alpha, \beta \in H^\circ$ и $h, g \in H$ верно:
\begin{eqnarray}
    \varepsilon\alpha = \alpha\varepsilon = \alpha \label{str1}\\
    (\alpha\beta)(a) = \alpha(a_{(1)})\beta(a_{(2)}) \label{str2}\\
    u^\circ(\alpha) = \alpha(1_H) \label{str3}\\
    \bigtriangleup\alpha(a \otimes b) = \alpha(ab) \label{str4}
\end{eqnarray}

Докажем полезную лемму 
\begin{lemma} \label{Group_lemma}
  Пусть задана алгебра Хопфа $H$ и группоподобный элемент $h \in H$. Если $h$ имеет конечный порядок как элемент группы, тогда его степени $1,\ h,\ \dots,\ h^{k-1}$ линейно независимы, где $k$ --- порядок элемента $h$. Если же $h$ имеет бесконечный порядок, то любая конечная систмеа из степеней $h$ линейно независима. \label{lem}
\end{lemma}

\begin{proof}
  Доказывать будем по индукции.
  База индукции, что система $1$ линейна независима, очевидна.
  Пусть $1,\ h,\ \dots,\ h^n$ линейно независимы. Предположим, что существуют $\lambda_i$ такие, что $h^{n+1} = \lambda_11+\lambda_2h + \dots + \lambda_{n+1}h^n$. Тогда получим, что  \[
  \bigtriangleup h^{n+1} = \bigtriangleup (\lambda_11+\lambda_2h + \dots + \lambda_{n+1}h^n) = \lambda_1 1 \otimes 1 +\lambda_2 h \otimes h + \dots + \lambda_{n+1} h^n \otimes h^n = h^{n+1} \otimes h^{n+1} =  (\lambda_11+\lambda_2h + \dots + \lambda_{n+1}h^n) \otimes (\lambda_11+\lambda_2h + \dots + \lambda_{n+1}h^n)
  \]
А значит для всех $i$ и $j$ верно, что $\lambda_i\lambda_j = \lambda_i \delta_{ij}$, где $\delta_{ij}$ --- символ Кронекера. Значит для любого $i$ $\lambda_i$ либо $0$, либо $1$. 
При этом если существуют $i_1,\ i_2$ такие, что $i_1 \neq i_2$ и $\lambda_{i_1} = \lambda_{i_2} = 1$, то $\lambda_{i_1}\lambda_{i_2} \neq 0$. Значит не более одного $\lambda_i$ может быть ненулевым. Так как \[
  \varepsilon(h^{n+1}) = 1 = \varepsilon(\lambda_11+\lambda_2h + \dots + \lambda_{n+1}h^n) = \lambda_1+\lambda_2 + \dots + \lambda_{n+1}
\]
Значит в точности один коэффицент равен 1. Пусть это $\lambda_{t} = 1$. Тогда $h^{n+1} = h^t$, иначе говоря $h^{n+1-t} = 1$. Если $h$ имеет бесконечный порядок получим противоречие. Если же $h$ имеет конечный порядок, то если $n+1 \leq \mathrm{ord}(h)$, тогда этого неверно, а значит они все $h^k$ вплоть до $h^{\mathrm{ord}-1}$ линейно независимы.

\end{proof}
\newpage

\section{Характеризация \texorpdfstring{$H$}{H}-модульной структуры}

Пусть задана $A$ "--- конечномерная $H$-модульная алгебра с 1 над полем $\mathbb{K}$, где $H$ "--- алгебра Хопфа.

Отображение $\psi: H \otimes A \to A$, такое что $\psi(h \otimes a) := ha$ для всех $h \in H$ и $a \in a$, называется $H$-модульной структурой. Определим гомоморфизм алгебр $\zeta: H \to \End(A)$ равенством $\zeta(h)(a) := \psi (h \otimes a)$ и назовём этот гомоморфизм структурой коносителя.

Выбрав базис в $A$ такой, что его первым элементом будет единица алгебры $A$, отождествим $\End(A)$ с $M_n(\mathbb{K})$.
Тогда существуют такие $(\alpha_{ij})_{ij} \in H^\circ$, что 
\begin{equation}\zeta(h) = 
\begin{pmatrix}
    \varepsilon(h) && \alpha_{12}(h) && \dots  && \alpha_{1n}(h)\\
    0    && \alpha_{22}(h) && \dots  && \alpha_{2n}(h)\\
    \vdots    &&  \vdots       && \ddots  && \vdots       \\
	0    && \alpha_{n2}(h) && \dots  && \alpha_{nn}(h)\\ \label{matrix}
\end{pmatrix}
\end{equation}

Для удобства положим $\alpha_{11} = \varepsilon$ и $\alpha_{21} = \dots = \alpha_{n1}  = 0$ и обозначим за $\mathrm{K}$ базис $\langle\varepsilon, (\alpha_{ij})_{ij}\rangle$.

Будем называть $\zeta(h)$ "--- матрицей модульной структуры $\psi$ . При так заданном отображении $\zeta$, верно $\zeta(H) = \cosupp \ \psi$. Кроме того, рассмотрев ранг матрицы модульной структуры, получаем $\mathrm{dim} \ \zeta(H) = \mathrm{dim}\  \langle K \rangle$. 

Нетрудно заметить, что $\mathrm{Ker}(\zeta)$ совпадает с аннулятором $\langle \mathrm{K} \rangle$.
Из того, что $\mathrm{Ker}(\zeta)$ "--- идеал конечной коразмерности, следует что в ядре любой функции из $\langle\mathrm{K}\rangle$ содержится идеал конечной коразмерности, а значит $\langle\mathrm{K}\rangle \subseteq H^\circ$. 

Так как $\zeta$ гомоморфизм алгебр, для любых $a, b \in H$ верно $\zeta(ab)=\zeta(a)\zeta(b)$. 
Значит для всех $1 \leq i, j \leq n$  верно $$
    \alpha_{ij}(ab) = \sum\limits_{k=1}^{n}\alpha_{ik}(a)\alpha_{kj}(b) \eqno(*)
$$. 

Дополним $K$ до базиса $(\varepsilon_i)_{i \in \Lambda}$ в $H^\circ$, тогда существует система $(e_i)_{i \in \Lambda}$ в $H$, такая что $\varepsilon_i(e_j) = \delta^i_j$, где $\delta^i_j$ "--- символ Кронекера. 
Разложим $\bigtriangleup\alpha_{ij}$ по базису $(\varepsilon_k \otimes \varepsilon_l)_{k, l \in \Lambda}$. Рассматривая $\bigtriangleup\alpha_{ij}$ от всех возможных $e_k \otimes e_l$ и используя (\ref{str4}) и $(*)$, получим 
\begin{equation}\label{coproduct}
  \bigtriangleup\alpha_{ij} = \sum\limits_{k=1}^{n}\alpha_{ik} \otimes \alpha_{kj}
\end{equation}

Кроме того получаем следующее соотношение, которое очевидно следует из (\ref{coproduct})

\begin{equation}\label{pods}
  \alpha_{ij}\left((\alpha_{kl})_\varepsilon(\alpha_{tm})_\varepsilon\right) = \delta^i_k\delta^j_m\delta^l_t
\end{equation}

\newpage

\subsection{Структура некоторых \texorpdfstring{$H$}{H}-модульных алгебр}

Пусть задана алгебра $A = \mathbb{K}[x]/(x^n)$, где поле $\mathbb{K}$, такое что $\chr \mathbb{K}$  не делит $n$ и $A$ является $H$-модульной алгеброй, где $H$ --- алгебра Хопфа.

Пусть $\psi$ --- модульная структура и $\zeta$ --- структура коносителя.
Выберем базис $1,\ \dots,\ x^{n-1}$, тогда матрица модульной структуры будет иметь вид (\ref{matrix}).

Пусть коноситель $\cosupp \psi$ является подалгеброй верхнетреугольных матриц. Тогда, в силу рассуждений выше, матрица модульной структуры верхнетреугольная.

Введём обозначение $\mathrm{P}^i_j(\alpha,\ \beta)$ --- сумма всех возможных произведений, таких что $\alpha$  входит $i$ раз, а $\beta$ --- $j$ раз, где $\alpha,\ \beta \in H^\circ$ и положим $\mathrm{P}^0_0(\alpha,\ \beta) = \varepsilon$.

\begin{remark} \label{reqursive}
  Из этого определения несложно понять, что
  \[
	\mathrm{P}^i_j(\alpha,\ \beta) = \alpha\mathrm{P}^{i-1}_j(\alpha,\ \beta)+\beta\mathrm{P}^i_{j-1}(\alpha,\ \beta)
  \]
\end{remark}

Переобозначим линейные функции из матрицы модульной структуры $\alpha_{12}$ и $\alpha_{22}$ как $\alpha$ и $\beta$ соответственно.
Тогда имеет место следующее утверждение.

\begin{proposition}
  Матрица модульной структуры имеет вид 
  \begin{equation}
	\zeta(h) = 
	\begin{pmatrix}
	  \mathrm{P}^0_0(\alpha,\ \beta) && \mathrm{P}^1_0(\alpha,\ \beta) && \mathrm{P}^2_0(\alpha,\ \beta) && \dots  && \mathrm{P}^{n-1}_0(\alpha,\ \beta) \\
	  0 && \mathrm{P}^0_1(\alpha,\ \beta) && \mathrm{P}^1_1(\alpha,\ \beta) && \dots  && \mathrm{P}^{n-2}_1(\alpha,\ \beta) \\
	  \vdots && 0 && \mathrm{P}^0_2(\alpha,\ \beta) && \dots  && \mathrm{P}^{n-3}_2(\alpha,\ \beta)\\
	  \\
	  \vdots &&  \vdots && \ddots && \ddots && \vdots \\
	  \\
	  0 && 0 && \dots && 0 && \mathrm{P}^0_{n-1}(\alpha,\ \beta) \\ \label{matrixPol}
	\end{pmatrix}
  \end{equation}
\end{proposition}

\begin{proof}
  Доказывать будем индукцией по номеру столбца. 
  База очевидна, ведь по определению $\mathrm{P}^0_0(\alpha,\ \beta) = \varepsilon$, $\mathrm{P}^1_0(\alpha,\ \beta) = \alpha$ и $\mathrm{P}^0_1(\alpha,\ \beta) = \beta$.
  Пусть для столбца с номером $k$ это верно, значит
  \[
	\zeta(h)(\bar x^k) = \sum\limits_{t = 0}^k \mathrm{P}^{k-t}_t(\alpha,\ \beta)(h)(\bar x^t) 
  \]
  Тогда так как $A$ --- $H$-модульная алгебра с $1$, значит
  \[
	\zeta(h)(\bar x^{k+1}) = \left(\zeta(h_{(1)})(\bar x)\right) \left(\zeta(h_{(2)})(\bar x^k)\right)
  \]
  Приравняем коэффиценты при соответствующих степенях $\bar x$. Тогда для всех $i$ от $0$ до $k+1$ 
  \[
	\alpha_{ik+1}(h) = \alpha(h_{(1)})\mathrm{P}^{k-i}_i(\alpha,\ \beta)(h_{(2)})+\beta(h_{(2)})\mathrm{P}^{k+1-i}_{i-1}(\alpha,\ \beta)(h_{(1)})
  \]
  \[
	\alpha_{ik+1} = \alpha\mathrm{P}^{k-i}_i(\alpha,\ \beta)+\beta\mathrm{P}^{k+1-i}_{i-1}(\alpha,\ \beta)
  \]
  Из сделанного замечания следует, что $\alpha_{ik+1} = \mathrm{P}^{k+1-i}_{i}(\alpha,\ \beta)$
  Тем самым шаг индукции доказан.
\end{proof}

Тогда соотношение (\ref{pods}) преобразуется в 
\begin{equation} \label{pods*}
  \mathrm{P}^i_j(\alpha,\ \beta)\left(\left(\mathrm{P}^k_l(\alpha,\ \beta)\right)_\varepsilon\left(\mathrm{P}^t_m(\alpha,\ \beta)\right)_\varepsilon\right) = 
  \delta^j_l\delta^{i+j}_{t+m}\delta^{k+l}_m
  %\delta_{j+1 l+1}\delta_{i+j+1 t+m+1}\delta_{k+l+1 m+1}
\end{equation}

С учётом $\bar x^n = 0$ получаем соотношения
\[
  \zeta(h)(\bar x^n) =  \left(\zeta(h_{(1)})(\bar x)\right) \left(\zeta(h_{(2)})(\bar x^{n-1})\right) = 0
\]
Приравняем все коэффиценты при степенях $\bar x$ к нулю. Тогда для всех $i$ от $0$ до $n-1$ 
\[
  \alpha(h_{(1)})\mathrm{P}^{n-1-i}_i(\alpha,\ \beta)(h_{(2)})+\beta(h_{(2)})\mathrm{P}^{n-i}_{i-1}(\alpha,\ \beta)(h_{(1)}) = 0
\]
\[
  \alpha\mathrm{P}^{n-1-i}_i(\alpha,\ \beta)+\beta\mathrm{P}^{n-i}_{i-1}(\alpha,\ \beta) = 0
\]
Из замечания (\ref{reqursive}) следует, что для всех $i$ от $0$ до $n-1$ 
\begin{equation} \label{null}
  \mathrm{P}^{n-i}_{i}(\alpha,\ \beta) = 0
\end{equation}
Из (\ref{coproduct}) следует, что $\varepsilon,\ \beta,\ \dots,\ \beta^{n-1}$ группоподобные, а значит для них верна лемма (\ref{Group_lemma}).

Обозначим систему $\mathrm{P}^t_0(\alpha,\ \beta),\ \dots,\ \mathrm{P}^t_{n-1-t}(\alpha,\ \beta)$ как $\mathrm{K}_t$.Теперь мы можем написать основную теорему, которую мы позже докажем.

\begin{theorem}\label{main}
  Пусть функции $\mathrm{K}_0 \cup \{\alpha\}$ линейно независимы. Тогда если, $n$ - простое число, то линейно независима система из всех функций $\bigcup\limits_{t = 0}^{n-1}\mathrm{K}_t$.
\end{theorem}

Перед тем, как доказать эту теорему, выведем две необходимые леммы.
\begin{lemma} \label{support1}
  Пусть система $\bigcup\limits_{t = 0}^k\mathrm{K}_t$ линейно независима. Тогда $\mathrm{P}^{k+1}_0(\alpha,\ \beta)$, где $k >0$ линейно независима с $\bigcup\limits_{t = 0}^k\mathrm{K}_t$.
\end{lemma}

\begin{proof}
  Предположим противное, тогда существуют $\lambda^i_j$ такие, что 
  \[
	\mathrm{P}^{k+1}_0(\alpha,\ \beta) = \lambda^i_j\mathrm{P}^{i}_j(\alpha,\ \beta)
  \]
  где $\mathrm{P}^{i}_j(\alpha,\ \beta) \in \bigcup\limits_{t = 0}^k\mathrm{K}_t$.

  Пусть $m \neq 0$ и $\mathrm{P}^{m}_0(\alpha,\ \beta) \in \bigcup\limits_{t = 0}^k\mathrm{K}_t$.
  Тогда \[
	\mathrm{P}^{i}_j(\alpha,\ \beta)\left(\left(\mathrm{P}^m_0(\alpha,\ \beta)\right)_\varepsilon\left(\mathrm{P}^0_m(\alpha,\ \beta)\right)_\varepsilon\right)  = 
	\delta^j_0\delta^i_m
  \]
  Значит  \[
	\mathrm{P}^{k+1}_0(\alpha,\ \beta)\left(\left(\mathrm{P}^m_0(\alpha,\ \beta)\right)_\varepsilon\left(\mathrm{P}^0_m(\alpha,\ \beta)\right)_\varepsilon\right) = \lambda^m_0 = 0
  \]
  Теперь пусть $l \neq 0$, $l$ и $m$ не равны одновременно $k+1$ и $0$ соответственно и $\mathrm{P}^{m}_l(\alpha,\ \beta) \in \bigcup\limits_{t = 0}^k\mathrm{K}_t$.
  Тогда \[
	\mathrm{P}^{i}_j(\alpha,\ \beta)\left(\left(\mathrm{P}^0_l(\alpha,\ \beta)\right)_\varepsilon\left(\mathrm{P}^m_l(\alpha,\ \beta)\right)_\varepsilon\right)  = \delta^j_l\delta^i_m
  \]
  Значит  \[
	\mathrm{P}^{k+1}_0(\alpha,\ \beta)\left(\left(\mathrm{P}^0_l(\alpha,\ \beta)\right)_\varepsilon\left(\mathrm{P}^m_l(\alpha,\ \beta)\right)_\varepsilon\right) = \lambda^m_l = 0
  \]
  Отсюда следует, что 
  \[
	\mathrm{P}^{k+1}_0(\alpha,\ \beta) = \lambda^0_0\mathrm{P}^0_0(\alpha,\ \beta) + \lambda^0_{k+1}\mathrm{P}^0_{k+1}(\alpha,\ \beta)
  \]
  Далее рассмотрим 
  \[
	\mathrm{P}^{k+1}_0(\alpha,\ \beta)\left(\left(\mathrm{P}^1_0(\alpha,\ \beta)\right)_\varepsilon\left(\mathrm{P}^k_1(\alpha,\ \beta)\right)_\varepsilon\right) = 1
  \]
  Однако 
  \begin{eqnarray*}
	\mathrm{P}^0_0(\alpha,\ \beta)\left(\left(\mathrm{P}^1_0(\alpha,\ \beta)\right)_\varepsilon\left(\mathrm{P}^k_1(\alpha,\ \beta)\right)_\varepsilon\right) = 0\\
	\mathrm{P}^0_{k+1}(\alpha,\ \beta)\left(\left(\mathrm{P}^1_0(\alpha,\ \beta)\right)_\varepsilon\left(\mathrm{P}^k_1(\alpha,\ \beta)\right)_\varepsilon\right) = 0\\
  \end{eqnarray*}
  Полученное противоречие и доказывает нашу лемму.
\end{proof}

\begin{lemma} \label{support2}
  Пусть система $\bigcup\limits_{t = 0}^k\mathrm{K}_t$ объединенная с системой $\mathrm{P}^{k+1}_0,\ \dots,\ \mathrm{P}^{k+1}_p$ линейно независима и $n$ --- простое число. Тогда $\mathrm{P}^{k+1}_{p+1}(\alpha,\ \beta)$, где $k >0$ линейно независима с $\bigcup\limits_{t = 0}^k\mathrm{K}_t$, объединенной с системой $\mathrm{P}^{k+1}_0,\ \dots,\ \mathrm{P}^{k+1}_p$.
\end{lemma}

\begin{proof}
  Предположим противное, тогда существуют $\lambda^i_j$ такие, что 
  \[
	\mathrm{P}^{k+1}_{p+1}(\alpha,\ \beta) = \lambda^i_j\mathrm{P}^{i}_j(\alpha,\ \beta)
  \]
  где $\mathrm{P}^{i}_j(\alpha,\ \beta)$ принадлежат или $\bigcup\limits_{t = 0}^k\mathrm{K}_t$, или  $\mathrm{P}^{k+1}_0,\ \dots,\ \mathrm{P}^{k+1}_p$ .

  Пусть $m \neq 0$ и $\mathrm{P}^m_{p+1}(\alpha,\ \beta)$ принадлежат или $\bigcup\limits_{t = 0}^k\mathrm{K}_t$, или  $\mathrm{P}^{k+1}_0,\ \dots,\ \mathrm{P}^{k+1}_p$ .
  Тогда \[
	\mathrm{P}^{i}_j(\alpha,\ \beta)\left(\left(\mathrm{P}^m_{p+1}(\alpha,\ \beta)\right)_\varepsilon\left(\mathrm{P}^0_{m+p+1}(\alpha,\ \beta)\right)_\varepsilon\right)  = 
	\delta^j_{p+1}\delta^i_m
  \]
  Значит  \[
	\mathrm{P}^{k+1}_{p+1}(\alpha,\ \beta)\left(\left(\mathrm{P}^m_0(\alpha,\ \beta)\right)_\varepsilon\left(\mathrm{P}^0_m(\alpha,\ \beta)\right)_\varepsilon\right) = \lambda^m_{p+1} = 0
  \]
  Теперь пусть $l \neq p+1$, $l$ и $m$ не равны одновременно $p+k+2$ и $0$ соответственно и $\mathrm{P}^{m}_l(\alpha,\ \beta)$ принадлежат или $\bigcup\limits_{t = 0}^k\mathrm{K}_t$, или  $\mathrm{P}^{k+1}_0,\ \dots,\ \mathrm{P}^{k+1}_p$ .

  Тогда \[
	\mathrm{P}^{i}_j(\alpha,\ \beta)\left(\left(\mathrm{P}^0_l(\alpha,\ \beta)\right)_\varepsilon\left(\mathrm{P}^m_l(\alpha,\ \beta)\right)_\varepsilon\right)  = \delta^j_l\delta^i_m
  \]
  Значит  \[
	\mathrm{P}^{k+1}_{p+1}(\alpha,\ \beta)\left(\left(\mathrm{P}^0_l(\alpha,\ \beta)\right)_\varepsilon\left(\mathrm{P}^m_l(\alpha,\ \beta)\right)_\varepsilon\right) = \lambda^m_l = 0
  \]
  Отсюда следует, что 
  \[
	\mathrm{P}^{k+1}_{p+1}(\alpha,\ \beta) = \lambda^0_{p+1}\mathrm{P}^0_{p+1}(\alpha,\ \beta) + \lambda^0_{p+k+2}\mathrm{P}^0_{p+k+2}(\alpha,\ \beta)
  \]
  Если $k > 0$, то рассмотрим 
  \[
	\mathrm{P}^{k+1}_{p+1}(\alpha,\ \beta)\left(\left(\mathrm{P}^1_0(\alpha,\ \beta)\right)_\varepsilon\left(\mathrm{P}^k_1(\alpha,\ \beta)\right)_\varepsilon\right) = 1
  \]
  Однако 
  \begin{eqnarray*}
	\mathrm{P}^0_0(\alpha,\ \beta)\left(\left(\mathrm{P}^1_0(\alpha,\ \beta)\right)_\varepsilon\left(\mathrm{P}^k_1(\alpha,\ \beta)\right)_\varepsilon\right) = 0\\
	\mathrm{P}^0_{k+1}(\alpha,\ \beta)\left(\left(\mathrm{P}^1_0(\alpha,\ \beta)\right)_\varepsilon\left(\mathrm{P}^k_1(\alpha,\ \beta)\right)_\varepsilon\right) = 0\\
  \end{eqnarray*}
  Получили противоречие и доказали.

  Если $k = 0$, тогда скажем, что так как $\mathrm{P}^0_t(\alpha,\ \beta)$ это степень $\beta$, то $\mathrm{P}^{1}_{p+1}(\alpha,\ \beta)$ коммутирует с $\beta$.
  
  Если $p+1 = n-2$, тогда 
  \[
	\mathrm{P}^1_{n-1}(\alpha,\ \beta) = \beta\mathrm{P}^1_{p+1}+\alpha\beta^{n-1} = 0
  \]
  Домножим на $\beta^{-p-2}$ и получим нетривиальную линейную комбинацию $\varepsilon,\ \beta,\ \alpha$, что противоречит условиям леммы, значит осталось доказать, для случая $p+1 < n-2$

  Для начала докажем формулу 
  \begin{equation}\label{sup1}
	\mathrm{P}^{1}_{p+2+l}(\alpha,\ \beta) = \mathrm{P}^1_l(\alpha,\ \beta)\beta^{p+2} + \mathrm{P}^1_{p+1}(\alpha,\ \beta)\beta^{l+1}
  \end{equation}
  При $l = 0$
  \begin{eqnarray*}
	\mathrm{P}^1_{p+2}(\alpha,\ \beta) &=& \beta\mathrm{P}^1_{p+1}(\alpha,\ \beta) + \alpha\beta^{p+2} = \\
	  &=& \mathrm{P}^1_0(\alpha,\ \beta)\beta^{p+2} + \mathrm{P}^1_{p+1}(\alpha,\ \beta)\beta
  \end{eqnarray*}

  База индукции доказана.
  Пусть верно при $l$, докажем, что верно при $l+1$
  \begin{eqnarray*}
	\mathrm{P}^{1}_{p+2+l+1}(\alpha,\ \beta)&=& \beta\mathrm{P}^1_{p+1+l}(\alpha,\ \beta) + \alpha\beta^{p+2+l+1} = \\
	&=& \beta(\mathrm{P}^1_l(\alpha,\ \beta)\beta^{p+2} + \mathrm{P}^1_{p+1}(\alpha,\ \beta)\beta^{l+1}) + \alpha\beta^{p+2+l+1} = \\
	&=& (\beta\mathrm{P}^1_l(\alpha,\ \beta) + \alpha\beta^{l+1})\beta^{p+2} + \beta\mathrm{P}^1_{p+1}(\alpha,\ \beta)\beta^{l+1} = \mathrm{P}^1_{l+1}(\alpha,\ \beta)\beta^{p+2} + \mathrm{P}^1_{p+1}(\alpha,\ \beta)\beta^{l+2}
  \end{eqnarray*}

  Тем самым мы доказали (\ref{sup1}).
  Тогда зная $\mathrm{P}^1_{n-1}(\alpha,\ \beta) = 0$, подставим в формулу $l = n-p-3$
  \[
  \mathrm{P}^{1}_{n-1}(\alpha,\ \beta) &=& \mathrm{P}^1_{n-2-(p+1)}(\alpha,\ \beta)\beta^{p+2} + \mathrm{P}^1_{p+1}(\alpha,\ \beta)\beta^{n-2 -p} = \\
  \mathrm{P}^1_{n-2-(p+1)}(\alpha,\ \beta)\beta^{p+2} + (\lambda^0_{p+1}\beta^{p+1}+ \lambda^0_{p+2}\beta^{p+2})\beta^{n-2 -p} = 0
  \]
  Тогда можно выразить 
  \[
	\mathrm{P}^1_{n-2-(p+1)}(\alpha,\ \beta) = -(\lambda^0_{p+1}\varepsilon+ \lambda^0_{p+2}\beta)\beta^{n-2-(p+1)}
  \]
  Благодаря этому и формуле (\ref{sup1}) мы можем получить ещё одну формулу
  \begin{equation}\label{sup2}
	\mathrm{P}^1_{n-1-q(p+2)}(\alpha,\ \beta) = -q(\lambda^0_{p+1}\varepsilon+ \lambda^0_{p+2}\beta)\beta^{n-1-q(p+2)}
  \end{equation}

  База индукции при $q = 1$ была доказана выше.
  Пусть теперь формула верна при $q$, докажем её истинность при $q+1$ пользуясь формулой (\ref{sup1}) при $l = n-1-(q+1)(p+2)$
  \begin{eqnarray*}
  \mathrm{P}^1_{n-1-q(p+2)}(\alpha,\ \beta) = \mathrm{P}^1_{n-1-(q+1)(p+2)}(\alpha,\ \beta)\beta^{p+2} + \mathrm{P}^1_{p+1}(\alpha,\ \beta)\beta^{n-(q+1)(p+2)}\\
	-q(\lambda^0_{p+1}\varepsilon+ \lambda^0_{p+2}\beta)\beta^{n-1-q(p+2)} = \mathrm{P}^1_{n-1-(q+1)(p+2)}(\alpha,\ \beta)\beta^{p+2} + \mathrm{P}^1_{p+1}(\alpha,\ \beta)\beta^{n-(q+1)(p+2)}\\
	-(q+1)(\lambda^0_{p+1}\varepsilon+ \lambda^0_{p+2}\beta)\beta^{n-1-(q+1)(p+2)} = \mathrm{P}^1_{n-1-(q+1)(p+2)}(\alpha,\ \beta)\beta^{p+2}
  \end{eqnarray*}
  Таким образом мы доказали, что формула (\ref{sup2}) верна.

  Повторно применяя формулу (\ref{sup2}) для всё больших $q$, найдётся момент, такой, что $n-1-(q+1)(p+2) < 0 \leq n-1-q(p+2)$. Обозначим такое значение $q$ как $q_0$.
  Тогда $0 \leq n-1-q_0(p+2) \leq p$. Действительно, первое неравенство следует из определения $q_0$.
  Второе неравенство верно, так как если $p+2 \leq n-1-q_0(p+2)$, то $0 \leq n-1-(q_0+1)(p+2)$, что противоречит выбору $q_0$. Если же $n-1-q_0(p+2) = p+1$, тогда $n = (q_0+1)(p+2)$, но $n$ --- простое число и $\chr \mathrm{K}$ не делит $n$, что невозможно.

  Тогда из неравенства следует, что $\mathrm{P}^1_{n-1-q_0(p+2)}(\alpha,\ \beta) \in K_1$, но это противоречит линейной независимости в условии леммы.
\end{proof}

Теперь можем доказать теорему (\ref{main})

\begin{proof}[Доказательство теоремы \ref{main}]
  Применив лемму (\ref{support2}) несколько раз можем докозать, что из того, что система $\bigcup\limits_{t = 0}^k\mathrm{K}_t$, объединенная с $\mathrm{P}^{k+1}_0$, линейно независима, следует, что система $\bigcup\limits_{t = 0}^{k+1}\mathrm{K}_t$ линейно независима. Дальше можно применить лемму (\ref{support1}) получим ситуацию анналогичную исходной, но система включает в себя уже $K_{k+1}$, и мы получаем шаг индукции.
\end{proof}

\newpage
\section{Классификация модульных структур}

\begin{theorem}
	Пусть $\psi:H \otimes A\ \to \ A $~-- структура $H$-модульной алгебры с $1$ на $A = \mathbb{K}[x]/(x^3)$, где $H$~--- некоторая алгебра Хопфа, $\mathrm{char} \ \mathbb{K} \neq 3$ и в поле существует примитивный корень степени 3. Выберем базис в $A: \bar 1, \bar x, \bar x^2$ и отождествим $\End(A)$ с $M_3(\mathbb{K})$. Предположим, что коноситель действия это подалгебра верхнетреугольных матриц. Тогда $\psi$ эквивалентно одной из следующих модульных структур над $A$:
    
    \begin{enumerate}
        \item действие поля $\mathbb{K}$ на алгебре $A$ умножением на скаляры; \label{scalar}
        
        \item действие групповой алгебры $\mathbb{K}\langle c\rangle_2$, заданное равенством 
        ${c\bar{x} = -\bar{x}}$; \label{eps=b^2}
        
        \item действие групповой алгебры $\mathbb{K}\langle c \rangle_3$, заданное равенством $c\bar{x} = \xi \bar{x}$,
        
        где $\xi$ "--- примитивный корень из единицы степени $3$;\label{diagonal} 
        
        \item $H_9(\xi)$-действие, заданное равенствами $c\bar{x}=\xi\bar{x},\ v\bar{x}= \bar{1}$, 
        
        где $\xi$ "--- примитивный корень из единицы степени $3$. \label{all}
        
    \end{enumerate}
\end{theorem}


\begin{proof}
  Обозначим за $\zeta$ структруру коносителя. Тогда $\zeta(h)$ будет матрицей модульной структуры, причем она будет иметь вид (\ref{matrix}), но для удобства работы подставим вместо $\mathrm{P}^i_j(\alpha,\ \beta)$ сами выражения для $\alpha$ и $\beta$
    \[\zeta(h) = 
    \begin{pmatrix}
        \varepsilon(h) & \alpha(h) & \alpha^2(h)\\
         0 & \beta(h) & (\alpha\beta + \beta\alpha)(h) \\
         0 & 0 & \beta^2(h)
    \end{pmatrix}\]
  
  Тогда соотношения (\ref{null}) примут вид
    \begin{align}
        &\alpha^3 = 0 \label{1}\\
        &\alpha^2\beta+\alpha\beta\alpha + \beta\alpha^2 = 0 \label{2}\\
        &\alpha\beta^2+\beta\alpha\beta+\beta^2\alpha = 0 \label{3}
    \end{align}
    
    В предыдущем разделе были доказаны разложения для результатов коумножения на элементах матрицы модульной структуры. Выпишем их для наших функций:
    \begin{eqnarray*}
        \bigtriangleup\varepsilon &=& \varepsilon \otimes \varepsilon\\
		\bigtriangleup\beta &=& \beta \otimes \beta\\
		\bigtriangleup\beta^2 &=& \beta^2 \otimes \beta^2\\
        \bigtriangleup\alpha &=& \varepsilon \otimes \alpha + \alpha \otimes \beta\\
        \bigtriangleup(\alpha\beta+\beta\alpha) &=& \beta \otimes (\alpha\beta+\beta\alpha) + (\alpha\beta+\beta\alpha)\otimes \beta^2\\
        \bigtriangleup\alpha^2 &=& \varepsilon \otimes \alpha^2 + \alpha \otimes (\alpha\beta+\beta\alpha) + \alpha^2 \otimes \beta^2\\
    \end{eqnarray*}
	Значит функции $\varepsilon,\ \beta, \beta^2$ являются группоподобными. Тогда пользуясь леммой (\ref{lem}) можем сказать, что либо $\beta = \varepsilon$, 
	либо $\beta^2 = \varepsilon$ и $\varepsilon,\ \beta$ линейно независимы, либо $\varepsilon,\ \beta,\ \beta^2$ линейно независимы.

	Если $\beta = \varepsilon$, тогда из (\ref{3}) 
	\[
	  3\alpha = 0
	\]
	Так как $\chr \mathbb{K} \neq 3$, то $\alpha = 0$, а значит наша матрица модульной структуры выглядит следующим образом:\[
    \zeta(h) = \begin{pmatrix}
        \varepsilon(h) & 0 & 0\\
        0 & \varepsilon(h) & 0\\
        0 & 0 & \varepsilon(h)\\
    \end{pmatrix}
    \]
    Тогда $\zeta(H)$ --- подалгебра скалярных матриц. Такая модульная структура эквивалентна структуре \ref{scalar}.

	Пусть $\beta^2 = \varepsilon$ и $\varepsilon,\ \beta$ линейно независимы.
	Тогда из (\ref{3}) следует, что
	\[
	  2\alpha +\beta\alpha\beta = 0
	\]
	А значит \[
	  \alpha = \beta^2\alpha\beta^2 = \beta(\beta\alpha\beta)\beta = \beta (-2\alpha)\beta = -2\beta\alpha\beta = 4\alpha
	\]
	То есть $3\alpha = 0$
	Так как $\chr \mathbb{K} \neq 3$, то $\alpha = 0$, а значит наша матрица модульной структуры выглядит следующим образом:\[
    \zeta(h) = \begin{pmatrix}
        \varepsilon(h) & 0 & 0\\
        0 & \beta(h) & 0\\
        0 & 0 & \varepsilon(h)\\
    \end{pmatrix}
    \]
	Докажем, что в таком случае модульная структура $\psi$ эквивалентна структуре \ref{eps=b^2}.

    Пусть $\psi_1: \mathbb{K}\langle c\rangle_2 \otimes A \to A$ "--- модульная структура \ref{eps=b^2}.
	Обозначим структуру коносителя как $\zeta_1:\mathbb{K}\langle c\rangle_2 \to M_3(\mathbb{K})$. Тогда получаем \[
    \zeta_1(c) = \begin{pmatrix}
        1 & 0 & 0\\
        0 & -1 & 0\\
        0 & 0 & 1\\
    \end{pmatrix}
    \]

    Выберем в $\mathbb{K}\langle c\rangle_3$  базис $(c^k)_{0 \leq k \leq 1}$ и выпишем образ этого базиса:
    \[
    \zeta_1(1) = \begin{pmatrix}
        1 & 0 & 0\\
        0 & 1 & 0\\
        0 & 0 & 1\\
    \end{pmatrix},\
    \zeta_1(c) = \begin{pmatrix}
        1 & 0 & 0\\
        0 & -1 & 0\\
        0 & 0 & 1\\
    \end{pmatrix}
    \]

    Рассмотрим эти матрицы как вектора в базисе из матричных единиц и запишем их координаты построчно в матрицу: \[
        M = \begin{pmatrix}
            1 & 0 & 0 & 0 & 1 & 0 & 0 & 0 & 1\\
            1 & 0 & 0 & 0 & -1 & 0 & 0 & 0 & 1\\
        \end{pmatrix}
    \]
    Нетрудно проверить, что $\mathrm{rank}\ M = 2$, а значит $\dim \mathrm{cosupp}\ \psi_1 = \dim \zeta_1(H) = 2$, следовательно $\mathrm{cosupp}\ \psi_1$ "--- подалгебра диагональных матриц, такая что для любой матрицы $B$, содержащейся в ней, верно $b_{11}=b_{33}$.

	В оставшемся случае $\varepsilon,\ \beta,\ \beta^2$ линейно независимы.
	Пусть $\alpha$ линейно зависима с $\varepsilon,\ \beta,\ \beta^2$. Тогда существуют $\lambda_i$ такие, что 
	\[
	  \alpha = \lambda_1\varepsilon + \lambda_2\beta + \lambda_3\beta^2
	\]
	Значит $\alpha$  и $\beta$ коммутируют и из (\ref{3}) следует, что 
	\[
	  3\alpha\beta^2 = 0
	\]
	Из того, что $\chr \mathbb{K} \neq 3$ и $\beta$ обратим получаем $\alpha = 0$ и 
	\[
	  \zeta(h) = \begin{pmatrix} 
		\varepsilon && 0 && 0\\
		0 && \beta && 0\\
		0 && 0 && \beta^2\\
	  \end{pmatrix}
	\]

	Докажем, что тогда модульная структуре $\psi$ эквивалентна структуре \ref{diagonal}.
    Пусть $\psi_1: \mathbb{K}\langle c\rangle_3 \otimes A \to A$ "--- модульная структура \ref{diagonal}.
	Обозначим структуру коносителя как $\zeta_1:\mathbb{K}\langle c\rangle_3 \to M_3(\mathbb{K})$. Тогда получаем \[
    \zeta_1(c) = \begin{pmatrix}
        1 & 0 & 0\\
        0 & \xi & 0\\
        0 & 0 & \xi^2\\
    \end{pmatrix}
    \]
    Выберем в $\mathbb{K}\langle c\rangle_3$  базис $(c^k)_{0 \leq k \leq 2}$ и выпишем образ этого базиса:
    \[
    \zeta_1(1) = \begin{pmatrix}
        1 & 0 & 0\\
        0 & 1 & 0\\
        0 & 0 & 1\\
    \end{pmatrix},\
    \zeta_1(c) = \begin{pmatrix}
        1 & 0 & 0\\
        0 & \xi & 0\\
        0 & 0 & \xi^2\\
    \end{pmatrix},\ 
    \zeta_1(c^2) = \begin{pmatrix}
        1 & 0 & 0\\
        0 & \xi^2 & 0\\
        0 & 0 & \xi\\
    \end{pmatrix}
    \]

    Рассмотрим эти матрицы как вектора в базисе из матричных единиц и запишем их координаты построчно в матрицу: \[
        M = \begin{pmatrix}
            1 & 0 & 0 & 0 & 1 & 0 & 0 & 0 & 1\\
            1 & 0 & 0 & 0 & \xi & 0 & 0 & 0 & \xi^2\\
            1 & 0 & 0 & 0 & \xi^2 & 0 & 0 & 0 & \xi\\
        \end{pmatrix}
    \]
    Нетрудно проверить, что $\mathrm{rank}\ M = 3$, а значит $\dim \mathrm{cosupp}\ \psi_1 = \dim \zeta_1(H) = 3$, следовательно $\mathrm{cosupp}\ \psi_1$ "--- алгебра диагональных матриц.

	Теперь будем считать функцию $\alpha$ линейно независимой с $\varepsilon,\ \beta,\ \beta^2$
	Тогда применима теорема (\ref{main}). 
	Значит система $\varepsilon,\ \beta,\ \beta^2,\ \alpha,\ \alpha\beta+\beta\alpha,\ \alpha^2$ линейно независима.
	
    \noindentДокажем, что в таком случае модульная структура $\psi$ эквивалентна модульной структуре \ref{all}.
    Пусть $\psi_1:H_9(\xi) \otimes A \to A$ "--- модульная структура \ref{all}.
    Ообозначим структуру коносителя как $\zeta_1:H_9(\xi) \to M_3(\mathbb{K})$.
    Тогда мы получим \[
    \zeta_1(c) = 
    \begin{pmatrix}
        1 & 0 & 0\\
        0 & \xi & 0\\
        0 & 0 & \xi^2\\
    \end{pmatrix},\ 
    \zeta_1(v) = 
    \begin{pmatrix}
        0 & 1 & 0\\
        0 & 0 & 1+\xi\\
        0 & 0 & 0\\
    \end{pmatrix}
    \]

    Выпишем образ базиса $(c^kv^l)_{0 \leq k,l \leq 2}$ в $\zeta(H)$:
    \begin{eqnarray*}
    &\zeta_1(1) = \begin{pmatrix}
    1 & 0 & 0\\
    0 & 1 & 0\\
    0 & 0 & 1\\
    \end{pmatrix},\
    \zeta_1(c) = \begin{pmatrix}
    1 & 0 & 0\\
    0 & \xi & 0\\
    0 & 0 & \xi^2\\
    \end{pmatrix},\
    \zeta_1(c^2) = \begin{pmatrix}
    1 & 0 & 0\\
    0 & \xi^2 & 0\\
    0 & 0 & \xi\\
    \end{pmatrix}\\
    &\zeta_1(v) = \begin{pmatrix}
    0 & 1 & 0\\
    0 & 0 & 1+\xi\\
    0 & 0 & 0\\
    \end{pmatrix},\
    \zeta_1(cv) = \begin{pmatrix}
    0 & 1 & 0\\
    0 & 0 & \xi+\xi^2\\
    0 & 0 & 0\\
    \end{pmatrix} 
    \zeta_1(c^2v) = \begin{pmatrix}
    0 & 1 & 0\\
    0 & 0 & 1+\xi^2\\
    0 & 0 & 0\\
    \end{pmatrix}\\
    &\zeta_1(v^2) = \begin{pmatrix}
    0 & 0 & 1+\xi\\
    0 & 0 & 0\\
    0 & 0 & 0\\
    \end{pmatrix},\
    \zeta_1(cv^2) = \begin{pmatrix}
    0 & 0 & 1+\xi\\
    0 & 0 & 0\\
    0 & 0 & 0\\
    \end{pmatrix} 
    \zeta_1(c^2v^2) = \begin{pmatrix}
    0 & 0 & 1+\xi\\
    0 & 0 & 0\\
    0 & 0 & 0\\
    \end{pmatrix}\\    
    \end{eqnarray*}

    Рассмотрим эти матрицы как вектора в базисе из матричных единиц и запишем их координаты построчно в матрицу: \[ M=
    \begin{pmatrix}
        1 & 0 & 0 & 0 & 1 & 0 & 0 & 0 & 1\\
        1 & 0 & 0 & 0 & \xi & 0 & 0 & 0 & \xi^2\\
        1 & 0 & 0 & 0 & \xi^2 & 0 & 0 & 0 & \xi\\
        0 & 1 & 0 & 0 & 0 & 1+\xi & 0 & 0 & 0\\
        0 & 1 & 0 & 0 & 0 & \xi+\xi^2 & 0 & 0 & 0\\
        0 & 1 & 0 & 0 & 0 & 1+\xi^2 & 0 & 0 & 0\\
        0 & 0 & 1+\xi & 0 & 0 & 0 & 0 & 0 & 0\\
        0 & 0 & 1+\xi & 0 & 0 & 0 & 0 & 0 & 0\\
        0 & 0 & 1+\xi & 0 & 0 & 0 & 0 & 0 & 0\\
    \end{pmatrix}
    \]
    Нетрудно, проверить, что $\mathrm{rank}\ M = 6$, а значит $\dim \mathrm{cosupp} \ \psi_1 = 6$, откуда $\mathrm{cosupp}\ \psi_1$ совпадает c алгеброй верхнетреугольных матриц.
\end{proof}


\newpage
\begin{thebibliography}{99}
	
\normalsize


\bibitem{ArnoldBook} Арнольд В.\,И. Математические методы классической механики. М.: Эдиториал УРСС, 2003, 416~с.

\bibitem{BahturinZaicevSegal} Бахтурин Ю.\,А., Зайцев М.\,В., Сегал С.\,К.
Конечномерные простые градуированные алгебры. \textit{Матем. сб.}, \textbf{199}:7 (2008), 21--40.


\bibitem{ModernGeometry}Дубровин~Б.\,А., Новиков~С.\,П., Фоменко~А.\,Т. Современная геометрия: в 3 т.
М.: Эдиториал УРСС, 2000.


\bibitem{MurphyBook} Мёрфи~Дж. $C^*$-алгебры и теория операторов. М.:~Факториал, 1997, 336~с.


\bibitem{PolyakovBook} Поляков~А.\,М. Калибровочные поля и струны. Ижевск: Издательский дом <<Удмуртский университет>>, 1999, 312~с.


\bibitem{HumphreysLieAlg}  Хамфриc Дж. Введение в теорию алгебр Ли и их представлений. М.: МЦНМО, 2003, 216 с. 

\bibitem{Herstein} Херстейн~И. Некоммутативные кольца.
М.: Мир, 1972, 192~с.

\bibitem{Abe} Abe, E. Hopf algebras. Cambridge University Press, Cambridge, 1980.

\bibitem{ASGordienko21ALAgoreJVercruysse}
Agore, A.\,L., Gordienko, A.\,S., Vercruysse, J.
$V$-universal Hopf algebras (co)acting on $\Omega$-algebras. \textit{Commun. Contemp. Math.},
\textbf{25}:1 (2023), 2150095-1 -- 2150095-40. 

\bibitem{ASGordienko24ALAgoreJVercruysse}
Agore, A.\,L., Gordienko, A.\,S., Vercruysse, J.
Lifting of locally initial objects and universal (co)acting Hopf algebras.
\texttt{arXiv:2406.17677 [math.CT] 25 Jun 2024}.


\bibitem{ConnesMarcolli} Connes, A., Marcolli, M. Quantum fields, noncommutative spaces and motives. (Книга готовится к печати, электронная версия \url{https://www.alainconnes.org/docs/bookwebfinal.pdf})


\bibitem{Danara} D\u asc\u alescu, S., N\u ast\u asescu, C., Raianu, \c S.
Hopf algebras: an introduction. New York, Marcel Dekker, Inc., 2001.


\bibitem{Donatsos} Donatsos, D., Daskaloyannis, C. Quantum groups and their applications in nuclear physics. \textit{Progress in Particle and Nuclear Physics},  \textbf{43} (1999), 537--618.


     \bibitem{ZaiGia} Giambruno, A., Zaicev, M.\,V.
Polynomial identities and asymptotic methods.
\textit{AMS Mathematical Surveys and Monographs.} \textbf{122},
Providence, R.I., 2005, 352~pp.


\bibitem{ASGordienko23}
Gordienko, A.\,S.
On a general notion of a polynomial identity and codimensions.
\textit{J. Pure and Appl. Algebra}, \textbf{229}:1 (2025), 107814-1 -- 107814-27.
%\texttt{arXiv:2404.15868 [math.RA] 24 Apr 2024}. 


\bibitem{HaagBook} Haag, R. Local quantum physics: fields, particles, algebras.
Springer-Verlag, Berlin, Heidelberg, 1996.

 
\bibitem{HaagKastler} Haag, R., Kastler, D. An algebraic approach to quantum field theory.
\textit{J. Math. Phys.}, \textbf{5} (1964), 848--861.


\bibitem{KakuBook} Kaku, M. Introduction to superstrings and M-theory. Springer-Verlag, New York, 1999.


\bibitem{MajidBook} Majid, S. Foundations of quantum group theory. Cambridge University Press, 1995.



\bibitem{Montgomery} Montgomery, S. Hopf algebras and their actions on rings. \textit{CBMS Lecture Notes} \textbf{82}, Amer. Math. Soc., Providence, RI, 1993.


\bibitem{Sweedler} Sweedler, M.\,E. Hopf Algebras. W.\,A. Benjamin, New York, 1969.


\end{thebibliography}

\end{document}
